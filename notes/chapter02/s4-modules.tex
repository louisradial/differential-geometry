\section{Tensors over a ring}

\begin{definition}{Ring}{ring}
    A ring \((R, +, \cdot)\) is a set \(R\) equipped with two maps \(+, \cdot : R \times R \to R\) called addition and multiplication that satisfy
    \begin{enumerate}[label=(\alph*)]
        \item Associativity of addition and multiplication: For all \(a,b,c \in R\), \(a + (b + c) = (a + b) + c\) and \(a \cdot (b\cdot c) = (a\cdot b) \cdot c\);
        \item Commutativity of addition: For all \(a,b \in R\), \(a + b = b + a\);
        \item Additive and multiplicative identity: There exists two distinct elements \(0\) and \(1\) in \(R\) such that for all \(a \in R\), \(a + 0 = a\) and \(a \cdot 1 = a\);
        \item Additive inverse: For every \(a \in R\) there exists an element in \(-a \in R\), called the additive inverse of \(a\), such that \(a + (-a) = 0\);
        \item Distributivity of multiplication over addition: For all \(a, b, c \in R\), \(a \cdot (b + c) = (a \cdot b) + (a\cdot c)\).
    \end{enumerate}
    Usually the multiplication \(a \cdot b\) is denoted by \(ab\).
\end{definition}

\begin{theorem}{Existence of Hamel basis}{existence_of_basis}
    Every module over a division ring has a Hamel basis.
\end{theorem}
\begin{corollary}
    Every vector field has a Hamel basis.
\end{corollary}
