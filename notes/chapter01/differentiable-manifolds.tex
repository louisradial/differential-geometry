The theory of smooth manifolds is a very useful generalization of the differential calculus on \(\mathbb{R}^n\). Namely, a smooth manifold is a topological space endowed with a differentiable structure such that it locally resembles Euclidean space.

% \begin{definition}{Differentiable Manifolds}{manifold}
%     A \emph{differentiable manifold} of dimension \(n\) is a set \(M\) and \emph{system of coordinates}, a family \(\set{(U_\alpha, \varphi_\alpha)}_{\alpha\in J}\) of injective mappings \(\varphi_\alpha: U_\alpha \to M\) of open sets \(U_\alpha\) of \(\mathbb{R}^n\) into \(M\), such that
%     \begin{enumerate}
%         \item the union of the \emph{coordinate neighborhoods} \(\varphi_\alpha(U_\alpha)\) cover the set \(M\), that is, \[\bigcup_{\alpha\in J} \varphi_\alpha(U_\alpha) = M;\]
%         \item for any pair \(\alpha, \beta\), with \(\varphi_\alpha(U_\alpha) \cap \varphi_\beta(U_\beta) = W \neq \emptyset\), the sets \(\varphi_\alpha ^{-1}(W)\) and \(\varphi_\beta^{-1}(W)\) are open sets in \(\mathbb{R}^n\) and the mappings \(\varphi_\beta^{-1}\circ\varphi_\alpha\) are differentiable;
%         \item the \emph{system of coordinates} \(\set{U_\alpha, \varphi_\alpha}_{\alpha\in J}\) is maximal relative to the conditions above.
%     \end{enumerate}
% \end{definition}

\section{Topology}
In order to define the notion of smooth manifolds, we must first begin with some building blocks, such as topology and topological manifolds.

\begin{definition}{Topology}{topology}
    A \emph{topology} on the set \(M\) is a family \(\mathcal{O}\) of subsets of \(M\) satisfying
    \begin{enumerate}[label=(\alph*)]
        \item the empty set and the set \(M\) belong to \(\mathcal{O}\);
        \item a finite intersection of elements of \(\mathcal{O}\) is a member of \(\mathcal{O}\); and
        \item an arbitrary union of members of \(\mathcal{O}\) belongs to \(\mathcal{O}\).
    \end{enumerate}

    The pair \topology{M} is named a \emph{topological space}, elements of \(\mathcal{O}\) are called \emph{open sets} and elements of \(M\smallsetminus\mathcal{O}\) are called \emph{closed sets}. Additionally, given an element \(p \in M\) an open set \(U\) that contains \(p\) is called a \emph{neighborhood} of \(p\).
\end{definition}

In \cref{prop:standard_topology,prop:subspace_topology,prop:product_topology} we show a couple of important examples that illustrate how the axioms of topological spaces given in \cref{def:topology} are used.

\begin{proposition}{Standard topology in \(\mathbb{R}^n\)}{standard_topology}
    We define the \emph{open ball} \(B_n(r,p) \subset \mathbb{R}^n\) of radius \(r > 0\) centered at \(p = (p^1, \dots, p^n)\) as the set
    \begin{equation*}
        B_n(r, p) = \set*{q = (q^1, \dots, q^n) \in \mathbb{R}^n : \sum_{i=1}^{n}{(q^i - p^i)^2} < r^2}.
    \end{equation*}
    Next, we define the \emph{standard topology} \(\mathcal{O}_\text{standard}\) of \(\mathbb{R}^n\). A subset \(U \subset \mathbb{R}^n\) is an open set if for every point \(p \in U\) there exists \(r > 0\) such that \(B_n(r, p) \subset U\). Then, \((\mathbb{R}^n, \mathcal{O}_\text{standard})\) is a topological space.
\end{proposition}
\begin{proof}
    It is easy to see \(\mathbb{R}^n\in\mathcal{O}_{\text{standard}}\) and \(\emptyset \in \mathcal{O}_{\text{standard}}\).

    Suppose \(U, V \in \mathcal{O}_{\text{standard}}\) and let \(p \in U \cap V \neq \emptyset\).Then, there exists \(r_U > 0\) and \(r_V > 0\) such that \(B_n(r_U, p) \subset U\) and \(B_n(r_V, p) \subset V\). Setting \(r = \min\set{r_U, r_V} > 0\) we have \(B_n(r, p)\) as subset of both \(U\) and \(V\), that is, \(B_n(r, p) \subset U\cap V\). It follows that \(U\cap V\in\mathcal{O}_\text{standard}\).

    Let \family{U_\alpha}{\alpha\in J} be a family of sets in \(\mathcal{O}_\text{standard}\). Let \(p \in \bigcup_{\alpha\in J}U_\alpha\), that is, there exists \(\beta \in J\) such that \(p \in U_\beta\). Since \(U_\beta \in \mathcal{O}_\text{standard}\), there exists \(r_\beta > 0\) such that \(B_n(r, p) \subset U_\beta \subset \bigcup_{\alpha \in J} U_\alpha\).
\end{proof}

\begin{proposition}{Subspace topology is a topology}{subspace_topology}
    Given a topological space \topology{M} and a subset \(S\) of \(M\), we define the \emph{subspace topology} \restrict{\mathcal{O}_M}{S} as
    \begin{equation*}
        \restrict{\mathcal{O}_M}{S} = \set{U \cap S : U \in \mathcal{O}_M}.
    \end{equation*}
    Then \((S, \restrict{\mathcal{O}_M}{S})\) is a topological space.
\end{proposition}
\begin{proof}
    We must show the conditions (a), (b), and (c) of \cref{def:topology} are satisfied.
    \begin{enumerate}[label=(\alph*)]
        \item Since \(S = M \cap S\) and \(\emptyset = \emptyset \cap S\), we have \(S \in \restrict{\mathcal{O}_M}{S}\) and \(\emptyset \in \restrict{\mathcal{O}_M}{S}\).
        \item Let \(U, V \in \restrict{\mathcal{O}_M}{S}\). Then, there exists \(\tilde{U}, \tilde{V} \in \mathcal{O}_M\) such that \(U = \tilde{U} \cap S\) and \(V = \tilde{V} \cap S\).Then, \(U \cap V = (\tilde{U}\cap S) \cap (\tilde{V} \cap S) = (\tilde{U}\cap\tilde{V})\cap S\). Since \(\tilde{U} \cap \tilde{V} \in \mathcal{O}_M\), we have \(U \cap V \in \restrict{\mathcal{O}_M}{S}\).
        \item Let \family{U_\alpha}{\alpha \in J} be a family of open sets in \(\restrict{\mathcal{O}_M}{S}\). For each \(\alpha \in J\), there exists a \(\tilde{U}_\alpha\in\mathcal{O}_M\) such that \(U_\alpha = \tilde{U}_\alpha \cap S\). Then
            \begin{align*}
                \bigcup_{\alpha \in J} U_\alpha &= \bigcup_{\alpha \in J} \tilde{U}_\alpha \cap S\\
                                                &= \set{m \in S : \exists \alpha \in J \text{ such that } m \in \tilde{U}_\alpha}\\
                                                &= \set{m \in M : \exists \alpha \in J \text{ such that } m \in \tilde{U}_\alpha} \cap S\\
                                                &= S\cap\bigcup_{\alpha\in J}\tilde{U}_\alpha.
            \end{align*}
        Since arbitrary unions of open sets is an open set, it follows that \(\bigcup_{\alpha\in J}U_\alpha \in \restrict{\mathcal{O}_M}{S}\).
    \end{enumerate}
\end{proof}

\begin{proposition}{Product topology}{product_topology}
    Let \topology{M} and \topology{N} be topological spaces. Define the \emph{product topology} \(\mathcal{O}_{M\times N}\) as the collection of subsets \(U \subset M \times N\) such that for all \((m,n) \in U\), there exists neighborhoods \(S \subset M\) and \(T \subset N\) of \(m \in M\) and \(n\in N\) such that \(S \times T \subset U\). Then \topology{M\times N} is a topological space.
\end{proposition}
\begin{proof}
    Clearly, \(M\times N\) and \(\emptyset\) are open sets in the product topology.

    Next, we consider open sets \(U, V \in \mathcal{O}_{M\times N}\) and an element \(p \in U \cap V\). Let \(p = (m, n) \in M \times N\), then there exists neighborhoods \(S_U, S_V\subset M\) of \(m\) and \(T_U, T_V \subset N\) of \(n\) such that \(S_U \times T_U \subset U\) and \(S_V \times T_V \subset V\). Let \(S = S_U \cap S_V\) and \(T = T_U \cap T_V\), then \(S \in \mathcal{O}_M\) and \(T \in \mathcal{O}_N\) are neighborhoods of \(m\) and \(n\), respectively. Moreover, \(S \times T \subset U \cap V\) is a neighborhood of \(p\), from which follows \(U \cap V \in \mathcal{O}_{M\times N}\).

    Let \family{U_\alpha}{\alpha\in J} be a family of open sets in the product topology. Let \(p\in \bigcup_{\alpha\in J}U_\alpha\), then there exists \(\beta \in J\) such that \(p \in U_{\beta}\). By definition, there exists open sets \(S \in \mathcal{O}_M\) and \(T \in \mathcal{O}_N\) such that \(S \times T \subset U_\beta \subset \bigcup_{\alpha\in J} U_\alpha\). Therefore, \(\bigcup_{\alpha\in J}U_\alpha\) is an open set.
\end{proof}

Along with the axioms of topological spaces described in \cref{def:topology} one might add further restrictions to specify the space considered. Some common restrictions are called the \emph{separation axioms}. Among these, we will make use of the T2 axiom, namely the Hausdorff property. Historically, Felix Hausdorff used this axiom in his original definition of a topological space, although the formulation of his other axioms was not exactly as those of \cref{def:topology}, but an equivalent one.
\begin{definition}{Hausdorff space}{hausdorff}
    A topological space \topology{M} is called a \emph{Hausdorff space} if for any \(p,q\in M\) with \(p\neq q\), there exists a neighborhood \(U\) of \(p\), i.e. \(p \in U \in \mathcal{O}_M\), and a neighborhood \(V\) of \(q\) such that \(U \cap V = \emptyset\).
\end{definition}

\section{Homeomorphisms}

With the notion of topological spaces, we may ask ourselves whether certain maps between topological spaces can preserve the topology. That is, a map that takes open sets in the domain topology into open sets in the target topology. To define such a map we define \emph{continuity}.

\begin{definition}{Continuous map}{continuity}
    Let \topology{M} and \topology{N} be topological spaces. Then a map \(f : M \to N\) is \emph{continuous} (with respect to \(\mathcal{O}_M\) and \(\mathcal{O}_N\)) if, for all \(V \in \mathcal{O}_N\), the preimage \(f^{-1}(V)\) is an open set in \(\mathcal{O}_M\).
\end{definition}

In short, a map is continuous if and only the preimages of (all) open sets are open sets. Now a map that preserves the topology is called a \emph{homeomorphism}, which is defined as a continuous bijection with continuous inverse. We now prove such a map satisfies the condition required.

\begin{proposition}{Homeomorphism maps open sets to open sets}{homeomorphism}
    Let \topology{M} and \topology{N} be topological spaces. Suppose a map \(f : M \to N\) is a homeomorphism, then \(f\) maps open sets in \(\mathcal{O}_M\) into open sets in \(\mathcal{O}_N\).
\end{proposition}
\begin{proof}
    Given a subset \(U \in \mathcal{O}_M\), we must show the image \(V = f(U)\) is open in \topology{N}. Taking our attention to the inverse map \(g = f^{-1} : N \to M\), we see the preimage \(g^{-1}(U) = V\) must be open in \topology{N}, due to continuity.
\end{proof}

If there exists a homeomorphism between two topological spaces, they are said to be homeomorphic to each other. This begs the question: if \topology{M} is homeomorphic to \topology{N} and \topology{N} is homeomorphic to \topology{P}, are \topology{M} and \topology{P} homeomorphic? To answer this we must show whether the composition of continuous maps is itself continuous.

\begin{theorem}{Composition of continuous maps}{continuous_composition}
    Let \topology{M}, \topology{N}, and \topology{P} be topological spaces. If the maps \(f: M \to N\) and \(g : N \to P\) are continuous (with respect to the appropriate topologies), then the map \(g \circ f : M \to P\) is continuous with respect to \(\mathcal{O_M}\) and \(\mathcal{O_P}\).
\end{theorem}
\begin{proof}
    Let \(V\) be an open set of \topology{P}. We must show the preimage \((g \circ f)^{-1}(V)\) is an open set of \topology{M}. We have
    \begin{align*}
        (g\circ f)^{-1}(V) &= \set{m \in M : g\circ f(m) \in V}\\
                           &= \set{m \in M : f(m) \in g^{-1}(V)}\\
                           &= f^{-1}\left(g^{-1}(V)\right).
    \end{align*}
    Since the map \(g\) is continuous and \(V\) is an open set in \topology{P}, it follows that \(g^{-1}(V)\) is open in \topology{N}. By the same argument, \(f^{-1}\left(g^{-1}(V)\right)\) is an open set in \topology{M}.
\end{proof}

\begin{corollary}
    If \topology{M} is homeomorphic to \topology{N} and \topology{N} is homeomorphic to \topology{P}, then \topology{M} is homeomorphic to \topology{P}.
\end{corollary}
\begin{proof}
    Let \(f : M \to N\) and \(g : N \to P\) be homeomorphisms from \topology{M} to \topology{N} and \topology{N} to \topology{P}, respectively. Consider the composition \(g\circ f : M \to P\).
    \[
    \begin{tikzcd}
        M \arrow{r}{f} \arrow[swap]{dr}{g\circ f} & N \arrow{d}{g} \\
                                            & P
    \end{tikzcd}
    \]
    By \cref{thm:continuous_composition}, the map \(g\circ f\) is a homeomorphism from \topology{M} to \topology{P}.
\end{proof}

As was done for the subspace topology, we prove a similar result for continuous maps.

\begin{proposition}{Restriction of a continuous map}{restriction_map}
    Let \topology{M} and \topology{N} be topological spaces and let \(f : M \to N\) be a continuous map. Let \(S\) be a subset of \(M\) and let \topology{S} be the subspace topology, then \(\restrict{f}{S} : S \to N\) is a continuous map with respect to \(\mathcal{O}_S\) and \(\mathcal{O}_N\).
\end{proposition}
\begin{proof}
    Let \(V \in \mathcal{O}_N\). Then, by the definition of preimage, we have
    \begin{align*}
        \restrict{f}{S}^{-1}(V) &= \set{s \in S : \restrict{f}{S}(s) \in V}\\
                                &= \set{s \in S : f(s) \in V}\\
                                &= f^{-1}(V) \cap S.
    \end{align*}
    By hypothesis, the preimage \(f^{-1}(V)\) is an open set in \topology{M}, so \(\restrict{f}{S}^{-1}(V)\) is an open set in the subspace topology.
\end{proof}

We can now define the notion of a topological space locally resembling Euclidean space.
\begin{definition}{Locally Euclidean topological space}{locally_euclidean}
    A topological space \topology{M} is \emph{locally Euclidean} of dimension \(n\) if for all \(m \in M\) there exists an open subset \(U \in \mathcal{O}_M\) about \(m\) that is homeomorphic to \(\mathbb{R}^n\) with respect to the subspace topology and the standard topology of \(\mathbb{R}^n\).
\end{definition}
It is sufficient to show the subspace topology \(\topology{U}\) is homeomorphic to an open ball in \(\mathbb{R}^n\), due to \cref{prop:ball_homeomorphic_euclidean}.
\begin{proposition}{Open ball is homeomorphic to the Euclidean space}{ball_homeomorphic_euclidean}
    Let \(r > 0\), then the map \(f : B_n(r, 0)\subset\mathbb{R}^n\to\mathbb{R}^n\) given by
    \[f(x) = \frac{x}{r - \norm{x}}\]
    is a homeomorphism with respect to the standard topology.
\end{proposition}
\begin{proof}
    We begin by checking \(f\) is one-to-one and onto.

    Suppose there exists \(x_1, x_2 \in B_n(r, 0)\) such that \(f(x_1) = f(x_2)\). It follows from
    \begin{align*}
        f(x_2) - f(x_1) &= \frac{x_2}{r - \norm{x_2}} - \frac{x_1}{r - \norm{x_1}}\\
                        &= \frac{\left(r - \norm{x_1}\right)x_2 - \left(r - \norm{x_2}\right)x_1}{\left(r - \norm{x_2}\right)\left(r - \norm{x_1}\right)}
    \end{align*}
    that \(\left(r - \norm{x_1}\right)x_2 = \left(r - \norm{x_2}\right)x_1\). Applying the norm to both sides, we have \(\norm{x_1} = \norm{x_2}\). Substituting back, we have \(x_1 = x_2\), proving \(f\) is injective.

    Suppose \(y \in \mathbb{R}^n\) and consider \(\xi = \frac{ry}{1 + \norm{y}}\). Clearly, \(\xi \in B_n(r,0)\). We have
    \begin{align*}
        f(\xi) &= f\left(\frac{ry}{1 + \norm{y}}\right)\\
               &= \frac{ry}{1 + \norm{y}} \frac{1}{r - \norm*{\frac{ry}{1 + \norm{y}}}}\\
               &= \frac{1}{\left(1 + \norm{y}\right)\left(1 - \frac{\norm{y}}{1 + \norm{y}}\right)} y\\
               &= y,
    \end{align*}
    so \(f\) is onto.

    We have shown \(f\) is a bijection with inverse \(f^{-1} : \mathbb{R}^n \to B_n(r, 0)\) defined by
    \begin{equation}
        f^{-1}(x) = \frac{rx}{1 + \norm{x}}.
    \end{equation}
    With the standard topology, continuity of \(f\) and \(f^{-1}\) follows from techniques of elementary calculus, and we conclude \(f\) is a homeomorphism.
\end{proof}

\section{Compactness and paracompactness}

\begin{definition}{Hausdorff space}{hausdorff}
    A topological space \topology{M} is called a \emph{Hausdorff space} if for any \(p,q\in M\) with \(p\neq q\), there exists a neighborhood \(U\) of \(p\), i.e. \(p \in U \in \mathcal{O}_M\), and a neighborhood \(V\) of \(q\) such that \(U \cap V = \emptyset\).
\end{definition}
\begin{remark}
    The Hausdorff property is one of the \emph{separation axioms} of topological spaces. Namely, a Hausdorff space is also called a \emph{T2 space}.
\end{remark}

\begin{definition}{Compactness}{compact}
    A topological space \topology{M} is \emph{compact} if every \emph{open cover} of \(M\) has a finite subcover. That is, the topological space is compact if for every family of open sets \(C\) that covers \(M\), i.e. \(\bigcup_{U \in C}U = M\) with \(U \in \mathcal{O}_M\), there exists a finite family of open sets \(F \subset C\) such that \(\bigcup_{U\in F} U = M\).

    Additionally, in a topological space \topology{N}, a subset \(S\subset N\) is called compact if the subspace topology is compact.
\end{definition}

\begin{theorem}{Heine-Borel theorem}{heine_borel}
    A subset \(S\subset\mathbb{R}^n\) with the standard topology is compact if it is closed and bounded.
\end{theorem}
\begin{proof}
    Refer to \cite{babyrudin}.
\end{proof}

\begin{definition}{Locally finite collection}{locally_finite}
    A collection of subsets \(C\) of a topological space \topology{M} is called \emph{locally finite} if each point in the space has a neighborhood that intersects only finitely many sets in \(C\). More precisely, for all \(p \in M\) there exists a neighborhood \(U \in \mathcal{O}_M\) about \(p\) such that \(U \cap V \neq \emptyset\) only for finitely many \(V \in C\).
\end{definition}

\begin{definition}{Refinement}{refinement}
    A \emph{refinement} of a cover \(C\) of a topological space \topology{M} is a cover \(D\) such that every set in \(D\) is contained in some set in \(C\). Precisely, let \(C = \family{U_\alpha}{\alpha \in A}\) and \(D = \family{V_\beta}{\beta \in B}\) such that \(\bigcup_{\alpha \in A} U_\alpha = M\) and \(\bigcup_{\beta \in B} V_\beta = M\), then \(D\) is a refinement of \(C\) if for all \(\beta \in B\) there exists \(\alpha \in A\) such that \(V_\beta \subset U_\alpha\).
\end{definition}

\begin{definition}{Paracompactness}{paracompact}
    A topological space \topology{M} is called \emph{paracompact} if every open cover \(C\) has an \emph{open refinement} \(\tilde{C}\) that is \emph{locally finite}.
\end{definition}

\begin{definition}{Partition of unity}{partition_of_unity}
    A \emph{partition of unity} of a topological space \topology{M} is a set \(\mathcal{F}\) of continuous functions from \(M\) to \([0,1]\subset\mathbb{R}\) such that for every point \(p \in M\)
    \begin{enumerate}[label=(\alph*)]
        \item there exists a neighborhood \(U \in \mathcal{O}_M\) about \(p\) where all but finitely many functions of \(\mathcal{F}\) vanish on \(U\);
        \item the sum of all function values at \(p\) is 1, that is, \(\sum_{f \in \mathcal{F}} f(p) = 1\).
    \end{enumerate}

    Moreover, let \(C = \family{U_\alpha}{\alpha \in J}\) be an open cover of \(M\). A \emph{partition of unity subordinate to the open cover \(C\)} is a family \(\mathcal{F}_C\) of continuous maps \(f_\alpha : p \to [0,1] \subset\mathbb{R}\) indexed over the same set \(J\) such that the support of \(f_\alpha\) is contained in \(U_\alpha\), for all \(\alpha \in J\). That is, for every \(f \in \mathcal{F}_C\) there exists an open set \(U \in C\) such that \(f(p) \neq 0 \implies p \in U\).
\end{definition}

\begin{theorem}{Paracompactness and partitions of unity}{hausdorff_paracompact}
    Let \topology{M} be a Hausdorff space. Then it is paracompact if and only if every open cover \(C\) admits a partition of unity subordinate to that cover.
\end{theorem}
\begin{proof}
    Refer to \cite{munkres_topology}.
\end{proof}

\input{chapter01/s4-connectedness}

