\section{Topology}
In order to define the notion of smooth manifolds, we must first begin with some building blocks, such as topology and topological manifolds.

\begin{definition}{Topology}{topology}
    A \emph{topology} on the set \(M\) is a family \(\mathcal{O}\) of subsets of \(M\) satisfying
    \begin{enumerate}[label=(\alph*)]
        \item the empty set and the set \(M\) belong to \(\mathcal{O}\);
        \item a finite intersection of elements of \(\mathcal{O}\) is a member of \(\mathcal{O}\); and
        \item an arbitrary union of members of \(\mathcal{O}\) belongs to \(\mathcal{O}\).
    \end{enumerate}

    The pair \topology{M} is named a \emph{topological space}, elements of \(\mathcal{O}\) are called \emph{open sets} and elements of \(M\smallsetminus\mathcal{O}\) are called \emph{closed sets}. Additionally, given an element \(p \in M\) an open set \(U\) that contains \(p\) is called a \emph{neighborhood} of \(p\).
\end{definition}

In \cref{prop:standard_topology,prop:subspace_topology,prop:product_topology} we show a couple of important examples that illustrate how the axioms of topological spaces given in \cref{def:topology} are used.

\begin{proposition}{Standard topology in \(\mathbb{R}^n\)}{standard_topology}
    We define the \emph{open ball} \(B_n(r,p) \subset \mathbb{R}^n\) of radius \(r > 0\) centered in the point \(p = (p^1, \dots, p^n)\) as the set
    \begin{equation*}
        B_n(r, p) = \set*{q = (q^1, \dots, q^n) \in \mathbb{R}^n : \sum_{i=1}^{n}{(q^i - p^i)^2} < r^2}.
    \end{equation*}
    Next, we define the \emph{standard topology} \(\mathcal{O}_\text{standard}\) of \(\mathbb{R}^n\). A subset \(U \subset \mathbb{R}^n\) is an open set if for every point \(p \in U\) there exists \(r > 0\) such that \(B_n(r, p) \subset U\). Then, \((\mathbb{R}^n, \mathcal{O}_\text{standard})\) is a topological space.
\end{proposition}
\begin{proof}
    It is easy to see \(\mathbb{R}^n\in\mathcal{O}_{\text{standard}}\) and \(\emptyset \in \mathcal{O}_{\text{standard}}\).

    Suppose \(U, V \in \mathcal{O}_{\text{standard}}\) and let \(p \in U \cap V \neq \emptyset\).Then, there exists \(r_U > 0\) and \(r_V > 0\) such that \(B_n(r_U, p) \subset U\) and \(B_n(r_V, p) \subset V\). Setting \(r = \min\set{r_U, r_V} > 0\) we have \(B_n(r, p)\) as subset of both \(U\) and \(V\), that is, \(B_n(r, p) \subset U\cap V\). It follows that \(U\cap V\in\mathcal{O}_\text{standard}\).

    Let \family{U_\alpha}{\alpha\in J} be a family of sets in \(\mathcal{O}_\text{standard}\). Let \(p \in \bigcup_{\alpha\in J}U_\alpha\), that is, there exists \(\beta \in J\) such that \(p \in U_\beta\). Since \(U_\beta \in \mathcal{O}_\text{standard}\), there exists \(r_\beta > 0\) such that \(B_n(r, p) \subset U_\beta \subset \bigcup_{\alpha \in J} U_\alpha\).
\end{proof}

\begin{proposition}{Subspace topology is a topology}{subspace_topology}
    Given a topological space \topology{M} and a subset \(S\) of \(M\), we define the \emph{subspace topology} \restrict{\mathcal{O}_M}{S} as
    \begin{equation*}
        \restrict{\mathcal{O}_M}{S} = \set{U \cap S : U \in \mathcal{O}_M}.
    \end{equation*}
    Then \((S, \restrict{\mathcal{O}_M}{S})\) is a topological space.
\end{proposition}
\begin{proof}
    We must show the conditions (a), (b), and (c) of \cref{def:topology} are satisfied.
    \begin{enumerate}[label=(\alph*)]
        \item Since \(S = M \cap S\) and \(\emptyset = \emptyset \cap S\), we have \(S \in \restrict{\mathcal{O}_M}{S}\) and \(\emptyset \in \restrict{\mathcal{O}_M}{S}\).
        \item Let \(U, V \in \restrict{\mathcal{O}_M}{S}\). Then, there exists \(\tilde{U}, \tilde{V} \in \mathcal{O}_M\) such that \(U = \tilde{U} \cap S\) and \(V = \tilde{V} \cap S\).Then, \(U \cap V = (\tilde{U}\cap S) \cap (\tilde{V} \cap S) = (\tilde{U}\cap\tilde{V})\cap S\). Since \(\tilde{U} \cap \tilde{V} \in \mathcal{O}_M\), we have \(U \cap V \in \restrict{\mathcal{O}_M}{S}\).
        \item Let \family{U_\alpha}{\alpha \in J} be a family of open sets in \(\restrict{\mathcal{O}_M}{S}\). For each \(\alpha \in J\), there exists a \(\tilde{U}_\alpha\in\mathcal{O}_M\) such that \(U_\alpha = \tilde{U}_\alpha \cap S\). Then
            \begin{align*}
                \bigcup_{\alpha \in J} U_\alpha &= \bigcup_{\alpha \in J} \tilde{U}_\alpha \cap S\\
                                                &= \set{m \in S : \exists \alpha \in J \text{ such that } m \in \tilde{U}_\alpha}\\
                                                &= \set{m \in M : \exists \alpha \in J \text{ such that } m \in \tilde{U}_\alpha} \cap S\\
                                                &= S\cap\bigcup_{\alpha\in J}\tilde{U}_\alpha.
            \end{align*}
        Since arbitrary unions of open sets is an open set, it follows that \(\bigcup_{\alpha\in J}U_\alpha \in \restrict{\mathcal{O}_M}{S}\).
    \end{enumerate}
\end{proof}

\begin{proposition}{Product topology}{product_topology}
    Let \topology{M} and \topology{N} be topological spaces. Define the \emph{product topology} \(\mathcal{O}_{M\times N}\) as the collection of subsets \(U \subset M \times N\) such that for all \((m,n) \in U\), there exists neighborhoods \(S \subset M\) and \(T \subset N\) of \(m \in M\) and \(n\in N\) such that \(S \times T \subset U\). Then \topology{M\times N} is a topological space.
\end{proposition}
\begin{proof}
    Clearly, \(M\times N\) and \(\emptyset\) are open sets in the product topology.

    Next, we consider open sets \(U, V \in \mathcal{O}_{M\times N}\) and an element \(p \in U \cap V\). Let \(p = (m, n) \in M \times N\), then there exists neighborhoods \(S_U, S_V\subset M\) of \(m\) and \(T_U, T_V \subset N\) of \(n\) such that \(S_U \times T_U \subset U\) and \(S_V \times T_V \subset V\). Let \(S = S_U \cap S_V\) and \(T = T_U \cap T_V\), then \(S \in \mathcal{O}_M\) and \(T \in \mathcal{O}_N\) are neighborhoods of \(m\) and \(n\), respectively. Moreover, \(S \times T \subset U \cap V\) is a neighborhood of \(p\), from which follows \(U \cap V \in \mathcal{O}_{M\times N}\).

    Let \family{U_\alpha}{\alpha\in J} be a family of open sets in the product topology. Let \(p\in \bigcup_{\alpha\in J}U_\alpha\), then there exists \(\beta \in J\) such that \(p \in U_{\beta}\). By definition, there exists open sets \(S \in \mathcal{O}_M\) and \(T \in \mathcal{O}_N\) such that \(S \times T \subset U_\beta \subset \bigcup_{\alpha\in J} U_\alpha\). Therefore, \(\bigcup_{\alpha\in J}U_\alpha\) is an open set.
\end{proof}

Along with the axioms of topological spaces described in \cref{def:topology} one might add further restrictions to specify the space considered. Some common restrictions are called the \emph{separation axioms}. Among these, we will make use of the T2 axiom, namely the Hausdorff property. Historically, Felix Hausdorff used this axiom in his original definition of a topological space, although the formulation of his other axioms were not exactly those of \cref{def:topology}, but an equivalent one.
\begin{definition}{Hausdorff space}{hausdorff}
    A topological space \topology{M} is called a \emph{Hausdorff space} if for any \(p,q\in M\) with \(p\neq q\), there exists a neighborhood \(U\) of \(p\), i.e. \(p \in U \in \mathcal{O}_M\), and a neighborhood \(V\) of \(q\) such that \(U \cap V = \emptyset\).
\end{definition}
