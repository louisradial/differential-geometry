\section{Topological manifolds}

\begin{definition}{Local chart}{local_chart}
    Let \topology{M} be a topological space and let \(U \in\mathcal{O}_M\) be an open set locally Euclidean of dimension \(n\). The homeomorphism \(x : U \to \mathbb{R}^n\) is called a \emph{coordinate chart on \(U\)}, which is called a \emph{local chart}. The pair \((U, x)\) is called a \emph{local chart of coordinates on \(U\)}, which will be abbreviated simply by \emph{chart}.
\end{definition}

\begin{definition}{Atlas}{atlas}
    Let \topology{M} be a locally Euclidean space of dimension \(n\). The family of charts
    \begin{equation*}
        \mathscr{A} = \family{(U_\alpha, x_\alpha)}{\alpha \in J}
    \end{equation*}
    is called an \emph{atlas} if the family \family{U_\alpha}{\alpha\in J} covers \(M\), i.e. \(\bigcup_{\alpha \in J} U_\alpha = M\), and if, for all \(\alpha \in J\), the map \(x_\alpha : U_\alpha \to \mathbb{R}^n\) is a homeomorphism from \(U_\alpha\) to \(\mathbb{R}^n\).
\end{definition}

\begin{definition}{Topological Manifolds}{topological_manifolds}
    A topological space \topology{M} is an \emph{\(n\)-dimensional topological manifold} if for every point \(p \in M\) there is a \emph{chart} \((U, x)\), where \(U\in\mathcal{O}_M\) is an open subset that contains \(p\) and \(x : U \to x(U) \subset \mathbb{R}^n\) is a \emph{homeomorphism} with respect to the subspace topology and the standard topology of \(\mathbb{R}^n\).
\end{definition}
