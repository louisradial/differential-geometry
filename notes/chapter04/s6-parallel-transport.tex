\section{Parallel transport}
Let \bundle{P}{\pi}{M} be a principal \(G\)-bundle equipped with a connection one-form \(\omega\). The key idea of parallel transport is to use a curve \(\gamma : [0,1]\to M\) on the base manifold and the connection one-form \(\omega\) to associate points in the fiber \(\preim{\pi}{\set{\gamma(0)}}\)  with the fiber \(\preim{\pi}{\set{\gamma(1)}}\). In order to do this, we construct a curve \(\gamma^\uparrow : [0,1] \to P\) in the total space which projects to the original curve \(\gamma\) and whose tangent vectors all lie in the horizontal spaces.

\begin{definition}{Horizontal lift of a curve to the principal bundle}{horizontal_lift}
    Let \(\gamma : [0,1] \to M\) be a smooth curve and let \(p \in \preim{\pi}{\set{\gamma(0)}}\) be a point in the fiber of the initial point of the curve. The \emph{horizontal lift of the curve \(\gamma\) through the point \(p\)} is the unique curve
    \begin{equation*}
        \gamma^\uparrow : [0,1] \to P
    \end{equation*}
    with \(\gamma^\uparrow(0) = p\) which satisfies
    \begin{enumerate}[label=(\alph*)]
        \item \(\pi \circ \gamma^\uparrow = \gamma\);
        \item \(\ver\left(X_{\gamma^\uparrow, \gamma^\uparrow(\lambda)}\right) = 0\), for all \(\lambda \in [0,1]\);
        \item \(\pf{\pi}{X_{\gamma^\uparrow, \gamma^\uparrow(\lambda)}} = X_{\gamma, \gamma(\lambda)}\), for all \(\lambda \in [0,1]\).
    \end{enumerate}
\end{definition}
\begin{remark}
    The horizontal lift is only unique due to the choice of the point in the fiber.
\end{remark}

Our strategy to write down an explicit expression for the horizontal lift is to proceed in two steps.
\begin{enumerate}[label=(\alph*)]
    \item "Generate" the horizontal lift by starting from some arbitrary smooth curve \(\delta : [0,1] \to P\) projecting to \(\gamma = \pi \circ \delta\) by action of a suitable smooth curve \(g : [0,1] \to G\), such that \(\gamma^\uparrow(\gamma) = \delta(\lambda) \ract g(\lambda)\), for \(\lambda \in [0,1]\).

    The suitable curve \(g : [0,1] \to G\) will be the solution to an ordinary differential equation with the initial condition \(g(0) = g_0\), where \(g_0\) is the unique group element for which \(\delta(0) \ract g_0 = p \in P\).
    \item We will locally explicitly solve the differential equation for \(g : [0,1]\to G\) by a path-ordered integral over the local Yang-Mills field.
\end{enumerate}

\begin{theorem}{Ordinary differential equation to determine the horizontal lift}{lift_ode}
    Let \bundle{P}{\pi}{M} be a principal \(G\)-bundle equipped with a connection one-form \(\omega\), let \(\gamma : [0,1] \to M\) and \(\delta : [0,1] \to P\) be smooth curves satisfying \(\pi \circ \delta = \gamma\), and let \(p \in P\) be a point in the total space. A smooth curve \(g : [0,1] \to G\) satisfies the ordinary differential equation
    \begin{equation*}
        \Ad{g(\lambda)^{-1}}\left(\omega_{\delta(\lambda)}X_{\delta, \delta(\lambda)}\right) + \Xi_{g(\lambda)}X_{g, g(\lambda)} = 0
    \end{equation*}
    with initial condition \(g(0) = g_0\), where \(\delta(0) \ract g_0 = p\), if and only if the curve \(\delta \ract g\) is the horizontal lift of \(\gamma\) through \(p\).
\end{theorem}
\begin{proof}
    Let \(g : [0,1] \to G\) be any smooth curve on \(G\) with \(g(0) = g_0\) and define the smooth curve
    \begin{align*}
        \eta : [0,1] &\to P\\
             \lambda &\mapsto \delta(\lambda) \ract g(\lambda)
    \end{align*}
    with \(\eta(0) = p\).

    If \(\delta(\lambda) \in \preim{\pi}{\set{\gamma(\lambda)}}\), then \(\delta(\lambda) \ract g(\lambda) \in \preim{\pi}{\set{\gamma(\lambda)}}\), that is, \(\pi \circ \eta = \gamma\). Recall the pushforward of a tangent vector to a curve is the tangent vector to the image curve. Indeed, for any smooth map \(f \in \smooth{M}\),
    \begin{align*}
        \left(\pf{\pi}{X_{\eta,\eta(\lambda)}}\right)f &= X_{\eta,\eta(\lambda)}(f \circ \pi)\\
                                                       &= \diff*{f \circ \pi \circ \eta(\lambda + t)}{t}[t=0]\\
                                                       &= \diff*{f \circ \gamma (\lambda + t)}{t}[t=0]\\
                                                       &= X_{\gamma,\gamma(\lambda)}f,
    \end{align*}
    hence \(\pf{\pi}{X_{\eta,\eta(\lambda)}}=X_{\gamma,\gamma(\lambda)}\). Thus, the curve \(\eta\) is the horizontal lift of \(\gamma\) through \(p\) if and only if
    \begin{equation*}
        \omega_{\eta(\lambda)}X_{\eta, \eta(\lambda)} = 0
    \end{equation*}
    for all \(\lambda \in [0,1].\)

    Notice the curve \(\eta\) is the image of the curve \((\delta, g) : [0,1] \to P \times G\) under the action \(\ract : P \times G \to P\), that is
    \begin{align*}
        X_{\eta, \eta(\lambda)} &= \pf{\ract}{\left(X_{\delta,\delta(\lambda)}, X_{g, g(\lambda)}\right)}\\
                                &= \pf{(\ract g(\lambda))}{X_{\delta,\delta(\lambda)}} + \pf{(\delta(\lambda)\ract)}{X_{g,g(\lambda)}}.
    \end{align*}
    Consider \(\omega_{\eta(\lambda)}X_{\eta,\eta(\lambda)}\). Then
    \begin{align*}
        \omega_{\eta(\lambda)}X_{\eta, \eta(\lambda)} &= \omega_{\delta(\lambda) \ract g(\lambda)}\left(\pf{(\ract g(\lambda))}{X_{\delta,\delta(\lambda)}}\right) + \omega_{\delta(\lambda) \ract g(\lambda)}\left(\pf{(\delta(\lambda)\ract)}{X_{g,g(\lambda)}}\right)\\
                                                      &= \left(\pb{(\ract g(\lambda))}\omega\right)_{\delta(\lambda)}X_{\delta,\delta(\lambda)} + \Xi_{g(\lambda)}X_{g,g(\lambda)}\\
                                                      &= \Ad{g(\lambda)^{-1}}\left(\omega_{\delta(\lambda)}X_{\delta,\delta(\lambda)}\right)+ \Xi_{g(\lambda)}X_{g,g(\lambda)}
    \end{align*}
    due to \cref{thm:connection_one_form_properties,lem:maurercartan_connection}. Hence, \(g\) satisfies the differential equation
    \begin{equation*}
        \Ad{g(\lambda)^{-1}}\left(\omega_{\delta(\lambda)}X_{\delta, \delta(\lambda)}\right) + \Xi_{g(\lambda)}X_{g, g(\lambda)} = 0
    \end{equation*}
    if and only if \(\eta\) is the horizontal lift of the curve \(\gamma\) through \(p\).
\end{proof}
\begin{corollary}
    If \(G\) is a matrix group, then the differential equation takes the form
    \begin{equation*}
        g(\lambda)^{-1} \cdot \omega_{\delta(\lambda)} X_{\delta,\delta(\lambda)} \cdot g(\lambda) + g(\lambda)^{-1} \cdot \dot{g}(\lambda) = 0,
    \end{equation*}
    or equivalently,
    \begin{equation*}
        \dot{g}(\lambda) = - \omega_{\delta(\lambda)}X_{\delta,\delta(\lambda)} g(\lambda).
    \end{equation*}
\end{corollary}

In order to manipulate this differential equation, we focus our attention to a local chart \((U,x) \in \mathscr{A}_M\) of the base manifold. In addition, we choose a local section \(\sigma : U \to P\), with which we consider the curve \(\delta = \sigma \circ \gamma\) and the Yang-Mills field \(\omega^U = \pb{\sigma}\omega\).

Recall the pushforward of a map between manifolds associates a tangent vector to a curve to the tangent vector to the image of the curve under the map, then \(\pf{\sigma}{X_{\gamma,\gamma(\lambda)}} = X_{\delta,\delta(\lambda)}\) for all \(\lambda \in [0,1]\). With this, we have
\begin{align*}
    \omega_{\delta(\lambda)}X_{\delta, \delta(\lambda)} &= \omega_{\delta(\lambda)}\left(\pf{\sigma}{X_{\gamma,\gamma(\lambda)}}\right)\\
                                                        &= (\pb{\sigma}{\omega})_{\gamma(\lambda)}X_{\gamma,\gamma(\lambda)}\\
                                                        &= \omega^U_{\gamma(\lambda)}X_{\gamma,\gamma(\lambda)}\\
                                                        &= \left(\omega^U_{\gamma(\lambda)}\right)_\mu (dx_{\gamma(\lambda)})^\mu \left(\left(X_{\gamma, \gamma(\lambda)}\right)^\nu\bvec{x^\nu}{\gamma(\lambda)}\right)\\
                                                        &= \left(\omega^U_{\gamma(\lambda)}\right)_\mu \left(X_{\gamma,\gamma(\lambda)}\right)^\mu.
\end{align*}
\begin{corollary}
    If \(G\) is a matrix group, then the differential equation is locally expressed as
    \begin{equation*}
        \dot{g}(\lambda) = - \left(\omega_{\gamma(\lambda)}^{U}\right)_\mu \dot{\gamma}^\mu(\lambda) g(\lambda),
    \end{equation*}
    with initial condition \(\gamma(0) = g_0\).
\end{corollary}

\begin{theorem}{Local solution in the case of a matrix Lie group}{}
    Let \(G\) be a matrix Lie group, let \bundle{P}{\pi}{M} be a principal \(G\)-bundle equipped with a connection one-form \(\omega\), and let \((U,x) \in \mathscr{A}_M\) be a chart in the base manifold. The horizontal lift of a curve \(\gamma [0,1] \to U \subset M\) through a point \(p \in P\) is given by the explicit expression
    \begin{equation*}
        \gamma^\uparrow(t) = (\sigma \circ \gamma)(t) \ract \left[\mathrm{P}\exp\left(-\int_{0}^{t} \dli{\lambda}\left(\omega_{\gamma(\lambda)}^{U}\right)_\mu \dot{\gamma}^\mu(\lambda)\right)\right]g_0,
    \end{equation*}
    for all \(t \in [0,1]\), where \(\sigma : U \to P\) is a smooth local section, \(\omega^U = \pb{\sigma}{\omega}\) is aYang-Mills field, and \(g_0 \in G\) is the unique element such that \((\sigma\circ\gamma)(0) \ract g_0 = p\).
\end{theorem}
\begin{proof}
    We solve the differential equation in the particular case of a matrix group. To simplify notation, we define a map
    \begin{align*}
        \Gamma : [0,1] &\to T_eG\\
               \lambda &\mapsto \left(\omega_{\gamma(\lambda)}^{U}\right)_\mu \dot{\gamma}^\mu(\lambda),
    \end{align*}
    then our differential equation becomes
    \begin{equation*}
        \dot{g}(\lambda) = - \Gamma(\lambda) g(\lambda),
    \end{equation*}
    for all \(\lambda \in [0,1]\), with initial condition \(g(0) = g_0\). \todo[Introduce a uniformly converging sequence of curves.] Naively, we consider
    \begin{equation*}
        g(t) = g_0 - \int_{0}^{t} \dli{\lambda} \Gamma(\lambda) g(\lambda)
    \end{equation*}
    for some \(t \in [0,1]\).

    We may recursively substitute this expression in the integrand, obtaining, after \(k\) steps,
    \begin{equation*}
        \begin{aligned}
            g(t) = g_0 &- \int_{0}^{t} \dli{\lambda_1} \Gamma(\lambda_1) g_0\\
                       &+ \int_{0}^t \dli{\lambda_1} \int_{0}^{\lambda_1} \dli{\lambda_2} \Gamma(\lambda_1)\Gamma(\lambda_2) g_0\\
                       &\,\vdots\\
                       &+ (-1)^{k} \int_{0}^{t} \dli{\lambda_1} \dots \int_{0}^{\lambda_{k-1}} \dli{\lambda_k} \Gamma(\lambda_1) \dots \Gamma(\lambda_k) g(\lambda_k).
        \end{aligned}
    \end{equation*}

    Notice the first \(k\) terms are possible to compute, since there is no dependence on the map \(g\), therefore arriving at an approximation to the desired map \(g\). Generically, as elements of the matrix Lie algebra \(T_eG\) do not commute, we express the limit \(k \to \infty\) with the \emph{path-ordered exponential}, namely
    \begin{equation*}
        g(t) =\mathrm{P}\exp\left(-\int_{0}^{t} \dli{\lambda}\Gamma(\lambda)\right)g_0.
\end{equation*}
    By \cref{thm:lift_ode}, the horizontal lift is given by \(\gamma^\uparrow = \sigma \circ \gamma \ract g\).
\end{proof}

\begin{definition}{Parallel transport map}{parallel_transport_map}
    Let \bundle{P}{\pi}{M} be a principal \(G\)-bundle equipped with a connection one-form \(\omega\), and let \(\gamma : [0,1] \to M\) be a smooth curve. The \emph{parallel transport map along \(\gamma\)} is defined by
    \begin{align*}
        T_{\gamma} : \preim{\pi}{\set{\gamma(0)}} &\to \preim{\pi}{\set{\gamma(1)}}\\
                                                p &\mapsto \gamma_p^\uparrow(1),
    \end{align*}
    where \(\gamma_p^\uparrow : [0,1] \to P\) is the horizontal lift of \(\gamma\) through the point \(p \in \preim{\pi}{\set{\gamma(0)}}\).
\end{definition}

\begin{proposition}{Parallel transport map is a bijection}{parallel_transport_bijection}
    Under the above assumptions, the parallel transport map \(T_{\gamma}\) is a bijection between the fibers \(\preim{\pi}{\set{\gamma(0)}}\) and  \(\preim{\pi}{\set{\gamma(1)}}\).
\end{proposition}
\begin{proof}
    Recall that \(\pf{(\ract g)}H_pP = H_{p\ract g}P\) for all \(p \in P\) and \(g \in G\). Then,
    for a horizontal lift \(\gamma_p^\uparrow\) through \(p\) it follows that
    \begin{equation*}
        \gamma_p^\uparrow(\lambda) \ract g = \gamma_{p \ract g}^\uparrow(\lambda),
    \end{equation*}
    hence \(T_{\gamma} (p \ract g) = T_{\gamma}(p) \ract g\).

    Suppose there exists \(p_1, p_2 \in \preim{\pi}{\set{\gamma(0)}}\) such that \(T_{\gamma}(p_1) = T_{\gamma}(p_2)\). Since both belong to the same fiber, there exists \(g \in G\) such that \(p_2 = p_1 \ract g\), then
    \begin{align*}
        T_{\gamma}(p_1) &= T_{\gamma}(p_1 \ract g)\\
                        &= T_{\gamma}(p_1) \ract g.
    \end{align*}
    Since the action is free, \(g = e\), hence the map is injective.

    Let \(q \in \preim{\pi}{\set{\gamma(1)}}\). Let \(\tilde{p} \in \preim{\pi}{\set{\gamma(0)}}\), then \(T_{\gamma}(\tilde{p}) \in \preim{\pi}{\set{\gamma(1)}}\). There exists \(\tilde{g} \in G\) such that \(q = T_{\gamma}(\tilde{p}) \ract \tilde{g}\). That is, \(p = \tilde{p} \ract \tilde{g} \in \preim{\pi}{\set{\gamma(0)}}\) satisfies \(T_{\gamma}(p) = q\). Therefore, \(T_{\gamma}\) is surjective.
\end{proof}

We consider the special case of closed curves, that is \(\gamma : [0,1] \to M\) with \(\gamma(0) = \gamma(1) = a.\) For each \(p \in \preim{\pi}{\set{a}}\), there exists a unique \(g^p_{\gamma} \in G\) such that \(p \ract g^p_{\gamma} = T_{\gamma}(p).\)

\begin{definition}{Holonomy group of a connection in a principal bundle}{holonomy_group}
    Let \bundle{P}{\pi}{M} be a principal \(G\)-bundle with a connection one-form \(\omega\). The \emph{holonomy group of \(\omega\) at a point \(p \in P\)} is the subgroup of \(G\) defined by
    \begin{equation*}
        \mathrm{Hol}_p(\omega) = \set*{g^p_{\gamma} \in G : \gamma \in \mathscr{L}_{\pi(p)}},
    \end{equation*}
    where \(\mathscr{L}_{\pi(p)}\) is the space of loops at \(\pi(p).\)
\end{definition}

\subsection{Parallel transport in an associated bundle}
We may naturally define horizontal lift to an associated bundle with the construction on the principal bundle.
\begin{definition}{Horizontal lift of a curve to the associated bundle}{horizontal_lift_associated}
    Let \bundle{P}{\pi}{M} be a principal \(G\)-bundle equipped with a connection one-form \(\omega\), and let the smooth manifold \(F\) be a left \(G\)-space, with which we consider the associated bundle \bundle{P_F}{\pi_F}{M}. Let \(\gamma : [0,1] \to M\) be a smooth curve and let \(\gamma^\uparrow_p : [0,1] \to P\) be the horizontal lift of \(\gamma\) through \(p \in \preim{\pi}{\set{\gamma(0)}}\). The \emph{horizontal lift of \(\gamma\) to the associated bundle through \([p,f] \in P_F\)} is the curve
    \begin{align*}
        \gamma^{\uparrow^{P_F}}_{[p,f]} : [0,1] &\to P_F\\
                                        \lambda &\mapsto [\gamma_p^\uparrow(\lambda), f].
    \end{align*}
\end{definition}

Similarly, the parallel transport map is defined with the horizontal lift.
\begin{definition}{Parallel transport map on the associated bundle}{parallel_transport_map_associated}
    The \emph{parallel transport map along \(\gamma\) on the associated bundle \bundle{P_F}{\pi_F}{M}} is defined by
    \begin{align*}
        T^{P_F}_{\gamma} : \preim{\pi_F}{\set{\gamma(0)}} &\to \preim{\pi_F}{\set{\gamma(1)}}\\
                                                    [p,f] &\mapsto \gamma_{[p,f]}^{\uparrow^{P_F}}(1),
    \end{align*}
    where \(\gamma_{[p,f]}^{\uparrow^{P_F}} : [0,1] \to P_F\) is the horizontal lift of \(\gamma\) to the associated bundle through the point \([p,f] \in \preim{\pi_F}{\set{\gamma(0)}}\).
\end{definition}

If \(F\) is a \(\mathbb{R}\)-vector space and the left action \(\lact : G \times F \to F\) is linear with respect to the fiber, then \bundle{P_F}{\pi_F}{M} is a \emph{vector bundle} associated to the principal bundle \bundle{P}{\pi}{M}. Let \(\psi : U \to P_F\) be a local section of the associated bundle. With a connection on the principal bundle and the vector space structure of the fibers, we may compare the values of \(\psi\) for neighboring points in \(U\) along a certain direction. This is the geometrical idea behind the \emph{covariant derivative} of a section. More precisely, let \(\gamma : [0, \varepsilon] \to U\) be a curve in \(M\) such that \(\gamma(0) = x\), then the \emph{covariant derivative of \(\psi\) in the direction \(\gamma\) at \(x\)} is the difference quotient
\begin{equation*}
    \nabla_{\gamma}\psi = \lim_{t \to 0}\left(\frac{(T_{\gamma}^{P_F}\circ \psi \circ \gamma)(t) - \psi(x)}{t}\right) \in \preim{\pi_F}{\set{x}},
\end{equation*}
where the parallel transport maps the fiber \(\preim{\pi_F}{\set{\gamma(t)}}\) to the fiber \(\preim{\pi_F}{\set{x}}\).
