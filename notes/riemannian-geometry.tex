\documentclass[12pt,oneside,a4paper]{book}

% Language and formatting
\usepackage{polyglossia}
\usepackage[strict=false,autostyle=true,english=american]{csquotes} %fvextra to avoid warning?
\setmainlanguage[variant=us]{english}

\usepackage[backend=biber]{biblatex}
\addbibresource{bibliography.bib}

% \setmainfont{Palatino Linotype}
% \setmathfont{Palatino Linotype}
\usepackage[a4paper, margin=2cm]{geometry}
\usepackage{booktabs}

% title header
\usepackage{titleps}% http://ctan.org/pkg/titleps
\makeatletter
\newpagestyle{main}{% Define page style main
    \sethead%
    [\textbf\thepage][][\thechapter.\ \chaptertitle]% [<even-left>][<even-center>][<even-right>]
    {\thesection.\ \sectiontitle}{}{\textbf\thepage}% {<odd-left>}{<odd-center>}{<odd-right>}
    \setfoot{}{}{}% {<left>}{<center>}{<right>}
}
\pagestyle{main}% Use page style main

% Images
\usepackage{tikz}
\usetikzlibrary{cd}
\usepackage{graphicx, caption, subcaption}
\usepackage{float}

% Math tools
\usepackage{amsfonts, mathtools, amssymb, amsmath, amsthm, enumitem}
\usepackage{newpxtext, newpxmath}
\numberwithin{equation}{section}
\usepackage[ISO]{diffcoeff}
\usepackage{tensor}
\usepackage{siunitx}

% Misc
\usepackage{xcolor}
\usepackage[breakable]{tcolorbox}

\DeclareMathOperator\sech{sech}
\DeclarePairedDelimiter\abs{\lvert}{\rvert}
\DeclarePairedDelimiter\norm{\lVert}{\rVert}
\DeclarePairedDelimiterX\inner[2]{\langle}{\rangle}{#1,\mathopen{}#2}
\DeclarePairedDelimiter\set{\{}{\}}
\newcommand\family[2]{\ensuremath{\set{#1}_{#2}}}
\newcommand\vetor[1]{\ensuremath{\boldsymbol{#1}}}
\newcommand\topology[1]{\ensuremath{\left(#1, \mathcal{O}_{#1}\right)}}
\newcommand\manifold[1]{\ensuremath{\left(#1, \mathcal{O}_{#1}, \mathscr{A}_{#1}\right)}}
\newcommand\restrict[2]{\ensuremath{\left.#1\right\rvert_{#2}}}

% catppuccin (latte)
\definecolor{Rosewater}{RGB}{220,138,120}
\definecolor{Flamingo}{RGB}{221,120,120}
\definecolor{Pink}{RGB}{234,118,203}
\definecolor{Mauve}{RGB}{136,57,239}
\definecolor{Red}{RGB}{210,15,57}
\definecolor{Maroon}{RGB}{230,69,83}
\definecolor{Peach}{RGB}{254,100,11}
\definecolor{Yellow}{RGB}{223,142,29}
\definecolor{Green}{RGB}{64,160,43}
\definecolor{Teal}{RGB}{23,146,153}
\definecolor{Sky}{RGB}{4,165,229}
\definecolor{Sapphire}{RGB}{32,159,181}
\definecolor{Blue}{RGB}{30,102,245}
\definecolor{Lavender}{RGB}{114,135,253}
\definecolor{Text}{RGB}{76,79,105}
\definecolor{Subtext1}{RGB}{92,95,119}
\definecolor{Subtext0}{RGB}{108,111,133}
\definecolor{Overlay2}{RGB}{124,127,147}
\definecolor{Overlay1}{RGB}{140,143,161}
\definecolor{Overlay0}{RGB}{156,160,176}
\definecolor{Surface2}{RGB}{172,176,190}
\definecolor{Surface1}{RGB}{188,192,204}
\definecolor{Surface0}{RGB}{204,208,218}
\definecolor{Base}{RGB}{239,241,245}
\definecolor{Mantle}{RGB}{230,233,239}
\definecolor{Crust}{RGB}{220,224,232}

% References
\usepackage{hyperref}
\usepackage[capitalize, nameinlink, noabbrev]{cleveref}
\makeatletter
\hypersetup{
    pdftitle=\@title,
    pdfauthor=\@author,
    colorlinks=true,
    linkcolor=Mauve,
    citecolor=pink,
    filecolor=red,
    urlcolor=blue,
    bookmarksdepth=4
}
\makeatother

% tcolorbox environments
\tcbuselibrary{theorems}
% theorem
\newtcbtheorem[number within=chapter]{theorem}{Theorem}%
{breakable,colback=Mauve!5,colframe=Mauve!95!black,fonttitle=\bfseries}{thm}
\crefname{tcb@cnt@theorem}{Theorem}{Theorems}

% definition
\newtcbtheorem[number within=chapter]{definition}{Definition}%
{breakable, colback=Pink!5,colframe=Pink!95!black,fonttitle=\bfseries}{def}
\crefname{tcb@cnt@definition}{Definition}{Definitions}

% proposition
\newtcbtheorem[number within=chapter]{proposition}{Proposition}%
{breakable,colback=Rosewater!5,colframe=Rosewater!95!black,fonttitle=\bfseries}{prop}
\crefname{tcb@cnt@proposition}{Proposition}{Propositions}

% lemma
\newtcbtheorem[number within=chapter]{lemma}{Lemma}%
{breakable,colback=Flamingo!5,colframe=Flamingo!95!black,fonttitle=\bfseries}{lem}
\crefname{tcb@cnt@lemma}{Lemma}{Lemmas}

% example
\newtheorem{example}{Example}[chapter]

% amsthm environments
% \newtheorem{definition}{Definition}[chapter]
% \newtheorem{theorem}{Theorem}[chapter]
% \newtheorem{proposition}{Proposition}[chapter]
\newtheorem{remark}{Remark}[chapter]
% \newtheorem{lemma}{Lemma}[chapter]
\newtheorem{corollary}{Corollary}[chapter]

\title{Notes on \textit{Riemannian Geometry}}
\author{Louis Bergamo Radial}

\setcounter{chapter}{0}

\begin{document}
\maketitle

\tableofcontents

\chapter{Topological Manifolds}
The theory of smooth manifolds is a very useful generalization of the differential calculus on \(\mathbb{R}^n\). Namely, a smooth manifold is a topological space endowed with a differentiable structure such that it locally resembles Euclidean space.

% \begin{definition}{Differentiable Manifolds}{manifold}
%     A \emph{differentiable manifold} of dimension \(n\) is a set \(M\) and \emph{system of coordinates}, a family \(\set{(U_\alpha, \varphi_\alpha)}_{\alpha\in J}\) of injective mappings \(\varphi_\alpha: U_\alpha \to M\) of open sets \(U_\alpha\) of \(\mathbb{R}^n\) into \(M\), such that
%     \begin{enumerate}
%         \item the union of the \emph{coordinate neighborhoods} \(\varphi_\alpha(U_\alpha)\) cover the set \(M\), that is, \[\bigcup_{\alpha\in J} \varphi_\alpha(U_\alpha) = M;\]
%         \item for any pair \(\alpha, \beta\), with \(\varphi_\alpha(U_\alpha) \cap \varphi_\beta(U_\beta) = W \neq \emptyset\), the sets \(\varphi_\alpha ^{-1}(W)\) and \(\varphi_\beta^{-1}(W)\) are open sets in \(\mathbb{R}^n\) and the mappings \(\varphi_\beta^{-1}\circ\varphi_\alpha\) are differentiable;
%         \item the \emph{system of coordinates} \(\set{U_\alpha, \varphi_\alpha}_{\alpha\in J}\) is maximal relative to the conditions above.
%     \end{enumerate}
% \end{definition}

\section{Topology}
In order to define the notion of smooth manifolds, we must first begin with some building blocks, such as topology and topological manifolds.

\begin{definition}{Topology}{topology}
    A \emph{topology} on the set \(M\) is a family \(\mathcal{O}\) of subsets of \(M\) satisfying
    \begin{enumerate}[label=(\alph*)]
        \item the empty set and the set \(M\) belong to \(\mathcal{O}\);
        \item a finite intersection of elements of \(\mathcal{O}\) is a member of \(\mathcal{O}\); and
        \item an arbitrary union of members of \(\mathcal{O}\) belongs to \(\mathcal{O}\).
    \end{enumerate}

    The pair \topology{M} is named a \emph{topological space}, elements of \(\mathcal{O}\) are called \emph{open sets} and elements of \(M\smallsetminus\mathcal{O}\) are called \emph{closed sets}. Additionally, given an element \(p \in M\) an open set \(U\) that contains \(p\) is called a \emph{neighborhood} of \(p\).
\end{definition}

In \cref{prop:standard_topology,prop:subspace_topology,prop:product_topology} we show a couple of important examples that illustrate how the axioms of topological spaces given in \cref{def:topology} are used.

\begin{proposition}{Standard topology in \(\mathbb{R}^n\)}{standard_topology}
    We define the \emph{open ball} \(B_n(r,p) \subset \mathbb{R}^n\) of radius \(r > 0\) centered at \(p = (p^1, \dots, p^n)\) as the set
    \begin{equation*}
        B_n(r, p) = \set*{q = (q^1, \dots, q^n) \in \mathbb{R}^n : \sum_{i=1}^{n}{(q^i - p^i)^2} < r^2}.
    \end{equation*}
    Next, we define the \emph{standard topology} \(\mathcal{O}_\text{standard}\) of \(\mathbb{R}^n\). A subset \(U \subset \mathbb{R}^n\) is an open set if for every point \(p \in U\) there exists \(r > 0\) such that \(B_n(r, p) \subset U\). Then, \((\mathbb{R}^n, \mathcal{O}_\text{standard})\) is a topological space.
\end{proposition}
\begin{proof}
    It is easy to see \(\mathbb{R}^n\in\mathcal{O}_{\text{standard}}\) and \(\emptyset \in \mathcal{O}_{\text{standard}}\).

    Suppose \(U, V \in \mathcal{O}_{\text{standard}}\) and let \(p \in U \cap V \neq \emptyset\).Then, there exists \(r_U > 0\) and \(r_V > 0\) such that \(B_n(r_U, p) \subset U\) and \(B_n(r_V, p) \subset V\). Setting \(r = \min\set{r_U, r_V} > 0\) we have \(B_n(r, p)\) as subset of both \(U\) and \(V\), that is, \(B_n(r, p) \subset U\cap V\). It follows that \(U\cap V\in\mathcal{O}_\text{standard}\).

    Let \family{U_\alpha}{\alpha\in J} be a family of sets in \(\mathcal{O}_\text{standard}\). Let \(p \in \bigcup_{\alpha\in J}U_\alpha\), that is, there exists \(\beta \in J\) such that \(p \in U_\beta\). Since \(U_\beta \in \mathcal{O}_\text{standard}\), there exists \(r_\beta > 0\) such that \(B_n(r, p) \subset U_\beta \subset \bigcup_{\alpha \in J} U_\alpha\).
\end{proof}

\begin{proposition}{Subspace topology is a topology}{subspace_topology}
    Given a topological space \topology{M} and a subset \(S\) of \(M\), we define the \emph{subspace topology} \restrict{\mathcal{O}_M}{S} as
    \begin{equation*}
        \restrict{\mathcal{O}_M}{S} = \set{U \cap S : U \in \mathcal{O}_M}.
    \end{equation*}
    Then \((S, \restrict{\mathcal{O}_M}{S})\) is a topological space.
\end{proposition}
\begin{proof}
    We must show the conditions (a), (b), and (c) of \cref{def:topology} are satisfied.
    \begin{enumerate}[label=(\alph*)]
        \item Since \(S = M \cap S\) and \(\emptyset = \emptyset \cap S\), we have \(S \in \restrict{\mathcal{O}_M}{S}\) and \(\emptyset \in \restrict{\mathcal{O}_M}{S}\).
        \item Let \(U, V \in \restrict{\mathcal{O}_M}{S}\). Then, there exists \(\tilde{U}, \tilde{V} \in \mathcal{O}_M\) such that \(U = \tilde{U} \cap S\) and \(V = \tilde{V} \cap S\).Then, \(U \cap V = (\tilde{U}\cap S) \cap (\tilde{V} \cap S) = (\tilde{U}\cap\tilde{V})\cap S\). Since \(\tilde{U} \cap \tilde{V} \in \mathcal{O}_M\), we have \(U \cap V \in \restrict{\mathcal{O}_M}{S}\).
        \item Let \family{U_\alpha}{\alpha \in J} be a family of open sets in \(\restrict{\mathcal{O}_M}{S}\). For each \(\alpha \in J\), there exists a \(\tilde{U}_\alpha\in\mathcal{O}_M\) such that \(U_\alpha = \tilde{U}_\alpha \cap S\). Then
            \begin{align*}
                \bigcup_{\alpha \in J} U_\alpha &= \bigcup_{\alpha \in J} \tilde{U}_\alpha \cap S\\
                                                &= \set{m \in S : \exists \alpha \in J \text{ such that } m \in \tilde{U}_\alpha}\\
                                                &= \set{m \in M : \exists \alpha \in J \text{ such that } m \in \tilde{U}_\alpha} \cap S\\
                                                &= S\cap\bigcup_{\alpha\in J}\tilde{U}_\alpha.
            \end{align*}
        Since arbitrary unions of open sets is an open set, it follows that \(\bigcup_{\alpha\in J}U_\alpha \in \restrict{\mathcal{O}_M}{S}\).
    \end{enumerate}
\end{proof}

\begin{proposition}{Product topology}{product_topology}
    Let \topology{M} and \topology{N} be topological spaces. Define the \emph{product topology} \(\mathcal{O}_{M\times N}\) as the collection of subsets \(U \subset M \times N\) such that for all \((m,n) \in U\), there exists neighborhoods \(S \subset M\) and \(T \subset N\) of \(m \in M\) and \(n\in N\) such that \(S \times T \subset U\). Then \topology{M\times N} is a topological space.
\end{proposition}
\begin{proof}
    Clearly, \(M\times N\) and \(\emptyset\) are open sets in the product topology.

    Next, we consider open sets \(U, V \in \mathcal{O}_{M\times N}\) and an element \(p \in U \cap V\). Let \(p = (m, n) \in M \times N\), then there exists neighborhoods \(S_U, S_V\subset M\) of \(m\) and \(T_U, T_V \subset N\) of \(n\) such that \(S_U \times T_U \subset U\) and \(S_V \times T_V \subset V\). Let \(S = S_U \cap S_V\) and \(T = T_U \cap T_V\), then \(S \in \mathcal{O}_M\) and \(T \in \mathcal{O}_N\) are neighborhoods of \(m\) and \(n\), respectively. Moreover, \(S \times T \subset U \cap V\) is a neighborhood of \(p\), from which follows \(U \cap V \in \mathcal{O}_{M\times N}\).

    Let \family{U_\alpha}{\alpha\in J} be a family of open sets in the product topology. Let \(p\in \bigcup_{\alpha\in J}U_\alpha\), then there exists \(\beta \in J\) such that \(p \in U_{\beta}\). By definition, there exists open sets \(S \in \mathcal{O}_M\) and \(T \in \mathcal{O}_N\) such that \(S \times T \subset U_\beta \subset \bigcup_{\alpha\in J} U_\alpha\). Therefore, \(\bigcup_{\alpha\in J}U_\alpha\) is an open set.
\end{proof}

Along with the axioms of topological spaces described in \cref{def:topology} one might add further restrictions to specify the space considered. Some common restrictions are called the \emph{separation axioms}. Among these, we will make use of the T2 axiom, namely the Hausdorff property. Historically, Felix Hausdorff used this axiom in his original definition of a topological space, although the formulation of his other axioms was not exactly as those of \cref{def:topology}, but an equivalent one.
\begin{definition}{Hausdorff space}{hausdorff}
    A topological space \topology{M} is called a \emph{Hausdorff space} if for any \(p,q\in M\) with \(p\neq q\), there exists a neighborhood \(U\) of \(p\), i.e. \(p \in U \in \mathcal{O}_M\), and a neighborhood \(V\) of \(q\) such that \(U \cap V = \emptyset\).
\end{definition}

\section{Convergence}

In analysis on \(\mathbb{R}^n\) with the standard topology, we often consider sequences \(x : \mathbb{N} \to \mathbb{R}^n\) and study whether it converges to a value. We say the sequence \(x\) converges to \(y \in \mathbb{R}^n\) if for all \(\varepsilon > 0\) there exists \(N \in \mathbb{N}\) such that \(x(i) - y \in B_n(\varepsilon, 0)\) for all \(i > N\). We generalize the notion of a convergent sequence to any topological space in \cref{def:convergence}.

\begin{definition}{Convergence of a sequence}{convergence}
    A sequence \(x : \mathbb{N} \to M\) on a topological space \topology{M} is said to \emph{converge} to a \emph{limit point} \(p \in M\) if for every neighborhood \(U \in \mathcal{O}_M\) of \(p\) there exists \(N \in \mathbb{N}\) such that \(x(n) \in U\) for all \(n > N\).
\end{definition}

\begin{theorem}{Unique limit on Hausdorff spaces}{unique_limit}
    Let \topology{M} be a Hausdorff space. If a sequence \(x\) converges on \(M\), its limit point is unique.
\end{theorem}
\begin{proof}
    Let \(p, q \in M\) be limit points of the sequence \(x\). Suppose, by contradiction, that \(p \neq q\). By the Hausdorff property, there exists neighborhoods \(U, V \in \mathcal{O}_M\) of \(p\) and \(q\), respectively, such that \(U \cap V = \emptyset\). From the definition of convergence, there exists \(N_p, N_q \in \mathbb{N}\) such that \(x(n) \in U\) for all \(n > N_p\) and \(x(n) \in V\) for all \(n > N_q\). Let \(N = \mathrm{min}\set{N_p, N_q}\), then for all \(n > N\), \(x(n) \in U\) and \(x(n) \in V\), that is, \(x(n) \in U \cap V = \emptyset\). This contradiction proves the statement.
\end{proof}

% open/closed in terms of sequences, closure

\input{chapter01/s3-homeomorphisms}
\input{chapter01/s4-compactness}
\section{Connectedness and path-connectedness}
\begin{definition}{Connectedness}{connectedness}
    A topological space \topology{M} is \emph{connected} unless there exists two non-empty, non-intersecting open sets \(A, B \in \mathcal{O}_M\) such that \(M = A \cup B\).
\end{definition}

\begin{theorem}{Interval is connected}{interval}
    Every interval \(I \subset \mathbb{R}\) is connected with respect to the standard topology.
\end{theorem}
\begin{proof}
    Suppose \(I\) is not connected, then \(I = A \cup B\), where \(A, B \subset I\) are non-empty, non-intersecting open sets. Let \(a \in A\) and \(b \in B\). Without loss of generality, we assume \(a < b\).

    Consider \(\alpha = \sup\set{x \in \mathbb{R} : [a, x) \cap I \subset A}\). Then \(a \leq \alpha \leq b\), so \(\alpha \in I\). Since \(B = I \smallsetminus A\) is open, we have \(A\) closed, hence \(\alpha \in A\). Since \(A\) is open, there exists \(r > 0\) such that \((\alpha - r, \alpha + r)\cap I \subset A\). We conclude \((a, \alpha+r)\cap I \subset A\), which is a contradiction.
\end{proof}

\begin{theorem}{Open and closed subsets in a connected space}{clopen}
    A topological space \topology{M} is connected if and only if \(\emptyset\) and \(M\) are the only subsets that are both open and closed.
\end{theorem}
\begin{proof}
    Suppose the topological space is connected. Suppose, by contradiction, the non-empty subset \(U \subsetneq M\) is open and closed. It follows from \(M = U \cup (M\smallsetminus U)\) that the topological space is not connected, a contradiction.

    Suppose the empty set and \(M\) are the only subsets that are both open and closed. Suppose, by contradiction, that the topological space is not connected. Then there exists two non-empty non-intersecting open sets \(A,  B \in \mathcal{O}_M\) such that \(M = A \cup B\). Clearly, \(B = M\smallsetminus A\) is closed, and likewise, \(A\) is closed. By hypothesis, \(A = B = M\) since they are both open and closed and are non-empty. We have thus arrived at a contradiction, since \(A\cap B = M\) is non-empty, which proves the topological space is connected.
\end{proof}

\begin{definition}{Path-connectedness}{pathconnected}
    A topological space \topology{M} is called \emph{path-connected} if for every pair of points \(p, q\in M\) there exists a continuous curve \(\gamma : [0, 1] \to M\) such that \(\gamma (0) = p\) and \(\gamma(1) = q\).
\end{definition}

\begin{theorem}{Path-connectedness implies connectedness}{pathconnected_implies_connectedness}
    A path-connected topological space is connected.
\end{theorem}
\begin{proof}
    Suppose, by contradiction, a topological space \topology{M} is path-connected but not connected. Then, there exists non-empty, non-intersecting open sets \(A, B \in \mathcal{O}_M\) such that \(M = A \cup B\). Choose \(a \in A\) and \(b \in B\). Since \topology{M} is path-connected, there exists a continuous curve \(\gamma: [0,1] \to M\) such that \(\gamma(0) = a\) and \(\gamma(1) = b\). Consider the preimage \([0,1] = \gamma^{-1}(M)\). If follows from
    \begin{align*}
        \gamma^{-1}(M) &= \gamma^{-1}(A\cup B)\\
                       &= \set*{x \in [0,1] : \gamma(x) \in A \cup B}\\
                       &= \gamma^{-1}(A) \cup \gamma^{-1}(B)
    \end{align*}
    that \([0,1]\) is the union of two non-empty sets, since \(0 \in \gamma^{-1}(A)\) and \(1 \in \gamma^{-1}(B)\). Suppose \(\gamma^{-1}(A) \cap \gamma^{-1}(B)\) is non-empty. Then there exists \(t \in [0,1]\) such that \(\gamma(t) \in A\) and \(\gamma(t) \in B\), that is, \(\gamma(t) \in A \cap B\). This contradiction shows \(\gamma^{-1}(A)\) and \(\gamma^{-1}(B)\) are non-intersecting. By continuity, these sets are both open and closed. Therefore, \([0,1]\) is not connected. By \cref{thm:interval}, this is a contradiction, and the theorem follows.
\end{proof}

\section{Homotopic curves and the fundamental group}

\begin{definition}{Homotopic curves}{homotopy}
    Let \topology{M} be a topological space and let \(p, q \in M\). Two curves \(\gamma,\eta: [0,1]\to M\) from \(p\) to \(q\), i.e. \(\gamma(0) = \eta(0) = p\) and \(\gamma(1) = \eta(1) = q\), are \emph{homotopic} if there exists a continuous map \(h : [0,1] \times [0,1] \to M\) such that \(h(0, \lambda) = \gamma(\lambda)\) and \(h(1, \lambda) = \eta(\lambda)\), for all \(\lambda \in [0,1]\).
\end{definition}

It is easy to check that homotopic curves form a equivalence relation. (do that)

\begin{definition}{Loops at a point}{loops}
    Let \topology{M} be a topological space and let \(p \in M\). We define the \emph{space of loops at \(p\)} as the set \(\mathscr{L}_p\) of continuous curves \(\gamma : [0,1] \to M\) such that \(\gamma(0) = \gamma(1) = p\).

    A \emph{concatenation} is a binary operation \(\ast_p : \mathscr{L}_p \times \mathscr{L}_p \to \mathscr{L}_p\) defined by
    \begin{equation*}
        (\gamma \ast_p \eta)(\lambda) = \begin{cases}\gamma(2\lambda), & \text{ for } \lambda \in \left[0,\frac12\right]\\\eta(2\lambda -1), &\text{ for }\lambda \in \left(\frac12, 1\right]\end{cases}
    \end{equation*}
    for all \(\gamma,\eta \in \mathscr{L}_p\).
\end{definition}

\begin{definition}{Fundamental group}{fundamental_group}
    The \emph{fundamental group \((\pi_{1,p}, \cdot)\)} of a topological space is the set
    \begin{equation*}
        \pi_{1,p} = \mathscr{L}_p / \mathrm{homotopy} = \set{[\gamma]_{\mathrm{homotopy}} : \gamma \in \mathscr{L}_p}
    \end{equation*}
    together with the product \(\cdot : \pi_{1,p} \times \pi_{1,p} \to \pi_{1,p}\) defined by
    \begin{equation*}
        [\gamma] \cdot [\eta] = [\gamma \ast_p \eta].
    \end{equation*}
\end{definition}

\begin{example}
    \begin{enumerate}[label=(\alph*)]
        \item On the two-sphere \(S^2\), the fundamental group has a single element, represented by the constant loop.
        \item For the cylinder \(C = \mathbb{R}\times S^1\), the fundamental group is homomorphic to the group \((\mathbb{Z}, +)\). There exists a bijection \(f : \pi_1 \to \mathbb{Z}\) with the property \(f(\alpha\cdot\beta) = f(\alpha) + f(\beta)\).
        \item On the torus \(T^2 = S^1 \times S^1\), we have \(\pi_1\) isomorphic to \(\mathbb{Z}\times\mathbb{Z}\).
    \end{enumerate}
\end{example}

\begin{definition}{Simply connected}{simply_connected}
    A topological space \topology{M} is \emph{simply connected} if it is path-connected and if for every point \(p \in M\) the fundamental group \((\pi_{1,p}, \cdot)\) is the trivial group.
\end{definition}
\begin{remark}
    Building up on the previous examples, we see that the two-sphere is simply connected, while the cylinder and the torus are not, although path-connected.
\end{remark}

\section{Topological manifolds}

We now finally define the notion of a manifold, which is a "well-behaving" topological space that \emph{locally} looks like Euclidean space.

% second countable?
\begin{definition}{Topological manifold}{topological_manifold}
    A paracompact Hausdorff space \topology{M} locally Euclidean of dimension \(n\) is called a \emph{\(n\)-dimensional topological manifold}.
\end{definition}

\begin{example}
    We list a few examples of constructions of new manifolds from other manifolds.
    \begin{enumerate}[label=(\alph*)]
        \item As with topological spaces, it is clear that the subspace topology \topology{N} is in its own right a manifold. This construction takes the name of \emph{submanifold}.
        \item Let \topology{M} be an \(m\)-dimensional manifold and \topology{N} be an \(n\)-dimensional manifold. Then the product topology \topology{M\times N} is an \((n+m)\)-dimensional manifold. As an example, the circle \(S^1\) is a 1-dimensional manifold, so the torus \(T^2 = S^1 \times S^1\) is a 2-dimensional manifold.
    \end{enumerate}
\end{example}

\subsection{Bundles}

It is clear we may not always construct a manifold as a product manifold. To see this, take the Möbius strip. %I have no idea what I'm doing, one day I'll learn this properly.
We generalize this concept with \emph{bundles}.

\begin{definition}{Bundles of topological manifolds}{bundle}
    A \emph{bundle of topological manifolds} is a triple \((E, \pi, M)\), where \topology{E} is a topological manifold called the \emph{total space}, \topology{M} is a topological manifold called the \emph{base space}, and \(\pi : E \to M\) is a continuous surjective map called the \emph{projection}. Let \(p \in M\), then \(\pi^{-1}(\set{p}) = F_p\) is the \emph{fiber at \(p\).}
\end{definition}
\begin{example}
    \begin{enumerate}[label=(\alph*)]
        \item The first example is when the total space \(E = M \times F\) is a product manifold of the base space \(M\) and a fiber \(F\). We take the projection \(\pi : M \times F \to M\) as the continuous map \((p, f) \mapsto p\).
        \item We now take the Möbius strip as the total space and the base space as \(S^1\). For any point in \(S^1\), the preimage of the projection is some interval on the real line.  % I'm not sure I understood this properly.
        \item We consider \(M = \mathbb{R}\) as the base space. We begin construct the total space by attaching to each point of \(M\) a circle. Then, we continuously deform the circles such that they collapse to a point on the positive numbers of the real line. Finally, we continuously stretch the point on that half of the line to increasing intervals. The fiber at any given point may be homeomorphic to a circle or to a point, or to an interval.
        % I don't see how this total space is a manifold
    \end{enumerate}
\end{example}

In the last example, a bundle was constructed with a fiber space that is not the same at different points. This motivates the following definition.

\begin{definition}{Fiber bundle}{fiber_bundle}
    A bundle \((E, \pi, M)\) is a \emph{fiber bundle with typical fiber \(F\)} if for all \(p \in M\), the preimage \(\pi^{-1}(\{p\})\) is homeomorphic to the manifold \(F\). The diagram
    \begin{equation*}
        \begin{tikzcd}[row sep = large, column sep = normal]
            F \arrow{r}{} & E \arrow{r}{\pi} & M
        \end{tikzcd}
    \end{equation*}
    denotes a fiber bundle.
\end{definition}

As an example, we consider the \emph{\(\mathbb{C}\)-line bundle over \(M\)} as the bundle \((E, \pi, M)\) with typical fiber \(\mathbb{C}\).

\begin{definition}{Section of the bundle}{bundle_section}
    Let \((E, \pi, M)\) be a bundle. A map \(\sigma : M \to E\) is called a section of the bundle if \(\pi \circ \sigma = \mathrm{id}_M\).
\end{definition}

As a special case we consider the "product bundle"
\begin{equation*}
    \begin{tikzcd}[column sep = normal, row sep = large]
        F \arrow{r}{} & M \times F \arrow{r}{\pi} & M,
    \end{tikzcd}
\end{equation*}
where \(\pi : M \times F \to M\) is the projection \((p, f) \mapsto p\). Given any function \(s : M \to F\), we may construct a section \(\sigma : M \to M \times F\) defined by \(p\mapsto (p, s(p))\). % As an example, we consider the \(\mathbb{C}\)-line bundle over \(M\) in quantum mechanics, and notice a wave function is a section of the \(\mathbb{C}\)-line bundle over the physical space.

Given a bundle \((E, \pi, M)\), we may consider the submanifolds \(E' \subset E\) and \(M' \subset M\) and the restriction \(\pi' = \restrict{\pi}{E'}\), then we call \((E', \pi', M')\) a \emph{subbundle}. Similarly, we consider another submanifold \(N \subset M\) and the preimage \(G = \pi^{-1}(N)\), then \((G, \restrict{\pi}{G}, N)\) is called the \emph{restricted bundle.}

Given two bundles \((E, \pi, M)\) and \((E', \pi', M')\) and a pair of maps \(u: E \to E'\) and \(f: M \to M'\). Then the pair \((u,f)\) is called a \emph{bundle morphism} if the diagram
\begin{equation*}
    \begin{tikzcd}[column sep = normal, row sep = large]
        E \arrow{r}{u} \arrow{d}{\pi} & E' \arrow{d}{\pi'}\\
        M \arrow{r}{f} & M'
    \end{tikzcd}
\end{equation*}
commutes. The bundles are called \emph{isomorphic as bundles} if there exists bundle morphisms \((u, f)\) and \((u^{-1}, f^{-1})\). In this case \((u,f)\) are called \emph{bundle isomorphisms} and are the structure-preserving maps for bundles.

We may weaken this condition by not requiring it globally. Two bundles \((E, \pi, M)\) and \((E', \pi', M')\) are \emph{locally isomorphic as bundles} if for every \(p \in M\) there exists a neighborhood \(U \in \mathcal{O}_M\) such that the restricted bundle \((\pi^{-1}(U), \restrict{\pi}{\pi^{-1}(U)}, U)\) is isomorphic to \((E', \pi', M')\).

A bundle is called \emph{(locally) trivial} if it is (locally) isomorphic to a product bundle. As an example, the cylinder is trivial and thus locally trivial and a Möbius strip is not trivial, but it is locally trivial. From now on we will disregard bundles that are not locally trivial. Then, locally, any section can be represented as a map from the base space to fiber.

\begin{definition}{Pullback bundle}{pullback_bundle}
    Let \((E, \pi, M)\) be a bundle, \(M'\) be a manifold and \(f : M' \to M\) be a continuous map.
    \begin{equation*}
        \begin{tikzcd}[column sep = normal, row sep = large]
            f^\ast E \arrow{d}{\pi'} \arrow{r}{u} & E \arrow{d}{\pi} \\
            M' \arrow{r}{f} & M
        \end{tikzcd}
    \end{equation*}
    Let \(f^\ast E = \set{(m', e) \in M'\times E : \pi(e) = f(m')} \subset M' \times E\) equipped with the subspace topology, define the map \(u : f^\ast E \to E\) by \( (m', e) \mapsto e\) and the projection \(\pi' : f^\ast E \to M'\) by \((m',e)\mapsto m'\) such that the diagram above commutes. The bundle \((f^\ast E, \pi', M')\) is called the \emph{pullback bundle by \(f\)} or the \emph{bundle induced by \(f\)}.
\end{definition}

Given a bundle \((E, \pi, M)\) with section \(\sigma : M \to E\) and a continuous map \(f : M' \to M\), where \(M'\) is a manifold, we may define the pullback section \(f^\ast \sigma : M' \to f^\ast E\) on the bundle induced by \(f\) by the map \( m' \mapsto (m', \sigma\circ f(m'))\). Since \(\pi \circ \sigma = \mathrm{id}_M\), it is clear the image of the pullback section is contained in \(f^\ast E\), so the map is well-defined.

\subsection{Atlas}

Since a topological manifold is locally Euclidean, for any point there is a neighborhood homeomorphic to Euclidean space. That is, there exists a homeomorphism from one such neighborhood to Euclidean space. Moreover, the collection of all such neighborhoods must cover the manifold. These remarks motivate the \cref{def:chart,def:atlas}.

\begin{definition}{Chart of a manifold}{chart}
    Let \topology{M} be a topological manifold of dimension \(n\). Then a pair \((U, x)\) where \(U \in \mathcal{O}_M\) and \(x: U \to x(U) \subset \mathbb{R}^n\) is called a \emph{chart} of the manifold.

    The component functions of x, the maps \(x^i : U \to \mathbb{R}\) defined by \(p \mapsto \mathrm{proj}_i(x(p))\), are called the \emph{coordinates of the point \(p \in U\) with respect to the chart \((U, x)\).}
\end{definition}

\begin{definition}{Atlas of a manifold}{atlas}
    Let \topology{M} be a topological manifold. The \emph{atlas} \(\mathscr{A} = \family{(U_\alpha, x_\alpha)}{\alpha \in J}\) is a family of charts of the manifold such that \(\bigcup_{\alpha\in J}{U_{\alpha}} = M\).
\end{definition}

It is easy to see that the domains of different charts may overlap. Let \((U, x)\) and \((V, y)\) be charts of a manifold \topology{M} with \(U\cap V \neq \emptyset\). Since \(U \cap V \in \mathcal{O}_M\) is a non-empty open set, the pairs \((U\cap V, x)\) and \((U \cap V, y)\) are charts on the manifold.
\begin{equation*}
    \begin{tikzcd}[column sep = normal, row sep = large]
        & U\cap V \arrow[swap]{ld}{x} \arrow{rd}{y} & \\
        x(U\cap V) \arrow{rr}{y\circ x^{-1}} && y(U\cap V)
    \end{tikzcd}
\end{equation*}
As the maps \(x\) and \(y\) are bijections, the map \(y \circ x^{-1} : x(U\cap V) \to y(U \cap V)\) is well defined and it is called the \emph{chart transition map}. By \cref{thm:continuous_composition}, the transition map is a homeomorphism.

The following definitions will seem redundant, but will serve as a framework of definitions as we require more and more structure on manifolds. Namely, later on we will replace the continuity requirement with a differentiability class.

\begin{definition}{\(\mathcal{C}^0\)-compatible charts}{c0_compatible}
    Two charts \((U, x)\) and \((V, y)\) of a \(n\)-dimensional manifold \topology{M} are \emph{\(\mathcal{C}^0\)-compatible} if either
    \begin{enumerate}[label=(\alph*)]
        \item \(U \cap V = \emptyset\); or
        \item \(U \cap V \neq \emptyset\) and the transition map \(y \circ x^{-1}\) is continuous as a map \(\mathbb{R}^n \to \mathbb{R}^n\).
    \end{enumerate}
\end{definition}

As discussed above, it is clear that any two charts on a manifold are \(\mathcal{C}^0\)-compatible. However, as we can study, for example, the differentiability class of the transition map as a map \(\mathbb{R}^n \to \mathbb{R}^n\), we may not conclude any such thing from the chart maps before adding structure to the manifold.

\begin{definition}{\(\mathcal{C}^0\)-atlas}{c0_atlas}
    A \emph{\(\mathcal{C}^0\)-atlas} \(\mathscr{A}\) is an atlas whose charts are pairwise \(\mathcal{C}^0\)-compatible.
\end{definition}

\begin{definition}{Maximal \(\mathcal{C}^0\)-atlas}{max_c0_atlas}
    A \(\mathcal{C}^0\)-atlas \(\mathscr{A}\) is \emph{maximal} if any chart \((U, x)\) that is \(\mathcal{C}^0\)-compatible with any \((V, y) \in \mathscr{A}\) is already contained in \(\mathscr{A}\).
\end{definition}

Even though every atlas is a \(\mathcal{C}^0\)-atlas, not every atlas is maximal. As an example we consider \topology{M} as the real line equipped with the standard topology. Then, the family \(\set{(\mathbb{R}, \mathrm{id}_{\mathbb{R}})}\) is an atlas, but the chart \(((-\infty, 0), \mathrm{id}_{\mathbb{R}})\) is not contained in it.

With an atlas and charts, one may study objects on an \(n\)-dimensional topological manifold \topology{M} from different points of view. As an example, we consider a curve \(\gamma : \mathbb{R} \to M\) and ask ourselves whether the curve is continuous. Clearly, by definition, we can check whether the preimage of open sets in \(M\) are open. From another perspective, we may employ charts.
\begin{equation*}
    \begin{tikzcd}[column sep = normal, row sep = large]
        & y(U) \subset \mathbb{R}^n \\
        \gamma^{-1}(U) \subset \mathbb{R} \arrow{r}{\gamma} \arrow{ur}{y\circ \gamma} \arrow[swap]{dr}{x\circ \gamma} &  U \subset M\arrow[swap]{u}{y} \arrow{d}{x}\\
        & x(U) \subset \mathbb{R}^n \arrow[bend right = 60, swap]{uu}{y\circ x^{-1}}
    \end{tikzcd}
\end{equation*}
Let \(U \subset M\) be an open set that contains the image of the curve, and let \((U, x)\) be a chart. The \emph{expression} of \(\gamma\) in this chart is the map \(x\circ \gamma: \gamma^{-1}(U) \subset \mathbb{R} \to x(U) \subset \mathbb{R}^n\). Then, the curve is continuous if its expression is continuous as a function from \(\mathbb{R} \to \mathbb{R}^n\), that is, if its components are continuous real-valued functions of a single variable. Additionally, if \((U, y)\) is another chart, then the expression of the curve in this chart \(y \circ \gamma : \gamma^{-1}(U) \to y(U)\) is continuous if and only if the expression in the other chart \(x\circ \gamma\) is continuous, because the chart transition map \(y\circ x^{-1} : x(U) \to y(U)\) is continuous. In this perspective, it is possible to ignore altogether the inner workings of the manifolds and use only the coordinate systems given by the charts.

Analogously, a map \(\phi : M \to N\), where \topology{M} is an \(m\)-dimensional topological manifold and \topology{N} is an \(n\)-dimensional topological manifold, is continuous if preimages of open sets in \(N\) are open sets in \(M\). By employing charts \((U, x)\) on \(M\) and \((V, y)\) on \(N\), the expression of the map is \(y \circ \phi \circ x^{-1} : x(U) \subset \mathbb{R}^n \to y(V) \subset \mathbb{R}^n\), and its continuity can be determined as a function of \(\mathbb{R}^m \to \mathbb{R}^n\).
\begin{equation*}
    \begin{tikzcd}[column sep = large, row sep = large]
        \tilde{x}(U) \subset \mathbb{R}^m \arrow{r}{\tilde{y}\circ \phi \circ \tilde{x}^{-1}} & \tilde{y}(V) \subset \mathbb{R}^n\\
        U \subset M \arrow{d}{x} \arrow[swap]{u}{\tilde{x}} \arrow{r}{\phi} & V \subset N \arrow{d}{y} \arrow[swap]{u}{\tilde{y}}\\
        x(U) \subset \mathbb{R}^m \arrow{r}{y\circ \phi \circ x^{-1}} \arrow[bend left=60]{uu}{\tilde{x} \circ x^{-1}} & y(V) \subset \mathbb{R}^n \arrow[bend right=60, swap]{uu}{\tilde{y}\circ y^{-1}}
    \end{tikzcd}
\end{equation*}
Taking another pair of charts \((U, \tilde{x})\) and \((V, \tilde{y})\), it follows from the continuity of the chart transition maps that the expression of \(\phi\) in these charts \(\tilde{y} \circ \phi \circ \tilde{x}^{-1} : \tilde{x}(U) \subset \mathbb{R}^n \to \tilde{y}(V) \subset \mathbb{R}^n\) is a continuous function if and only if the expression \(y \circ \phi\circ{x}^{-1}\) is continuous.



\chapter{Differentiable Manifolds}
\section{Smooth Atlas and Differentiable Manifolds}

Even though for topological manifolds the atlas was mostly redundant, although useful, we may add a differentiable structure to a topological manifold by refining the maximal \(\mathcal{C}^0\)-atlas. The following definitions are almost the same as \cref{def:c0_compatible,def:c0_atlas,def:max_c0_atlas}, substituting the continuity trivial requirement to the differentiability class \(\mathcal{C}^k\), with \(k \geq 1\).
\begin{definition}{Differentiability class on Euclidean spaces}{diff_class}
    A function \(f : U \subset \mathbb{R}^n \to \mathbb{R}\) defined on an open set \(U\subset \mathbb{R}^n\) (with respect to the standard topology) is of \emph{differentiability class \(\mathcal{C}^k\) on \(U\)} if all partial derivatives \(\partial_\alpha f\) on \(U\) exist and are continuous, where \(\alpha\) is a multi-index with \(|\alpha| \leq k\). We denote the set of all functions from \(U\) to \(\mathbb{R}\) that are of differentiability class \(\mathcal{C}^k\) on \(U\) by \(\mathcal{C}^k(U)\) and we say \(f \in \mathcal{C}^k(U)\) if \(f\) is of that differentiability class.

    Similarly, a function \(f : U \subset \mathbb{R}^n \to \mathbb{R}^m\) is of \emph{differentiability class \(\mathcal{C}^k\) on \(U\)} if its component functions \(f_i = \pi_i \circ f\) are of that differentiability class, where \(\pi_i : \mathbb{R}^m \to \mathbb{R}\) are the projections \((x^1, \dots, x^m) \mapsto x^i\). Likewise, \(f \in \mathcal{C}^k(U, \mathbb{R}^m)\) if \(f : U \to \mathbb{R}^m\) if \(f\) is of differentiability class \(\mathcal{C}^k\) on \(U\).
\end{definition}
Analogously, we may change the differentiability class \(\mathcal{C}^k\) to the smoothness condition \(\mathcal{C}^\infty\) or to real \(\mathcal{C}^\omega\) or complex analytic functions.

\begin{definition}{\(\mathcal{C}^k\)-compatible charts}{compatible_charts}
    Let \topology{M} be an \(n\)-dimensional topological manifold. Two charts \((U, x)\) and \((U, y)\) are \(\mathcal{C}^k\)-compatible if
    \begin{enumerate}[label=(\alph*)]
        \item \(U \cap V = \emptyset\); or
        \item \(U \cap V \neq \emptyset\) and the transition map \(y \circ x^{-1}\) is of class \(\mathcal{C}^k\) as a map \(\mathbb{R}^n \to \mathbb{R}^n\).
    \end{enumerate}
\end{definition}

\begin{definition}{\(\mathcal{C}^k\)-atlas}{compatible_atlas}
    A \emph{\(\mathcal{C}^k\)-compatible atlas} \(\mathscr{A}\) is an atlas whose charts are pairwise \(\mathcal{C}^k\)-compatible.
\end{definition}

\begin{definition}{Maximal \(\mathcal{C}^k\)-atlas}{maximal_atlas}
    A \(\mathcal{C}^k\)-atlas \(\mathscr{A}\) is \emph{maximal} if any chart \((U, x)\) that is \(\mathcal{C}^k\)-compatible with any \((V, y) \in \mathscr{A}\) is already contained in \(\mathscr{A}\).
\end{definition}

We now state a theorem \cite{hirsch} that allows us to consider either maximal \(\mathcal{C}^0\)-atlas or a maximal smooth atlas. That is, the construction of a maximal \(\mathcal{C}^1\)-atlas is not weaker than the construction of a maximal \(\mathcal{C}^\infty\)-atlas.
\begin{theorem}{Any maximal differentiable atlas contains a smooth atlas}{whitney_atlas}
    Any maximal \(\mathcal{C}^k\)-atlas with \(k \geq 1\) contains a smooth atlas. And two maximal \(\mathcal{C}^k\)-atlases that contain the same smooth atlas are already identical.
\end{theorem}

\begin{definition}{\(\mathcal{C}^k\)-manifold}{manifold}
    A \(\mathcal{C}^k\)-manifold is a triple \manifold{M} when \topology{M} is a topological manifold and \(\mathscr{A}_M\) is a maximal \(\mathcal{C}^k\)-atlas.
\end{definition}

\begin{remark}
    A given topological manifold can carry different incompatible atlases. As an example, take the manifold as the real line equipped with the standard topology. We consider two atlases \(\mathscr{A}_1 = \set{(\mathbb{R}, \mathrm{id}_{\mathbb{R}})}\) and \(\mathscr{A}_2 = \set{(\mathbb{R}, x)}\) where \(x : \mathbb{R} \to \mathbb{R}\) is the map \(p \mapsto p^{\frac13}\). Clearly, \(\mathscr{A}_1\) can be completed to be a maximal smooth atlas. The other atlas is a smooth atlas, as there is only one chart, so the chart transition map is the identity map, which is smooth. As such, it is also possible to complete this atlas to a maximal smooth atlas. We observe however that the atlas \(\mathscr{A}_1 \cup \mathscr{A}_2\) is not even \(\mathcal{C}^1\)-compatible, as the transition map is not differentiable at \(p = 0\).
\end{remark}

\begin{definition}{Differentiable map between manifolds}{differentiable_map}
    Let \(\phi : M \to N\) be a map, where \manifold{M} and \manifold{N} are \(\mathcal{C}^k\)-manifolds of dimensions \(m\) and \(n\) respectively.

    \begin{equation*}
        \begin{tikzcd}[column sep = large, row sep = large]
            U \subset M \arrow{d}{x}  \arrow{r}{\phi} & V \subset N \arrow{d}{y} \\
            x(U) \subset \mathbb{R}^m \arrow{r}{y\circ \phi \circ x^{-1}} & y(V) \subset \mathbb{R}^n
        \end{tikzcd}
    \end{equation*}

    The map \(\phi\) is \emph{differentiable} at \(p \in M\) if there exists charts \((U, x) \in \mathscr{A}_M\) and \((V, y) \in \mathscr{A}_N\), where \(U\) and \(V\) are neighborhoods of \(p\) and \(\phi(p)\), such that the expression of \(\phi\) in these charts, that is the map \(y \circ \phi \circ x^{-1} : x(U) \subset \mathbb{R}^m \to y(V) \subset \mathbb{R}^n\) is a \(\mathcal{C}^k\) map from \(\mathbb{R}^m\to \mathbb{R}^n\).
\end{definition}

Although the definition relies on the existence of charts, we must show the differentiability of a map between manifolds is independent of the choice of charts. Without loss of generality, we suppose there are another pair of charts \((U, \tilde{x})\in\mathscr{A}_M\) and \((V, \tilde{y})\in\mathscr{A}_N\).

\begin{equation*}
    \begin{tikzcd}[column sep = large, row sep = large]
        \tilde{x}(U) \subset \mathbb{R}^m \arrow{r}{\tilde{y}\circ \phi \circ \tilde{x}^{-1}} & \tilde{y}(V) \subset \mathbb{R}^n\\
        U \subset M \arrow{d}{x} \arrow[swap]{u}{\tilde{x}} \arrow{r}{\phi} & V \subset N \arrow{d}{y} \arrow[swap]{u}{\tilde{y}}\\
        x(U) \subset \mathbb{R}^m \arrow{r}{y\circ \phi \circ x^{-1}} \arrow[bend left=60]{uu}{\tilde{x} \circ x^{-1}} & y(V) \subset \mathbb{R}^n \arrow[bend right=60, swap]{uu}{\tilde{y}\circ y^{-1}}
    \end{tikzcd}
\end{equation*}
Because the atlases are \(\mathcal{C}^k\)-compatible, the chart transition maps \(\tilde{x}\circ x^{-1}\) and \(\tilde{y} \circ y^{-1}\) are \(\mathcal{C}^k\) maps. Therefore, the expression \(\tilde{y}\circ \phi \circ\tilde{x}^{-1} : \tilde{x}(U) \to \tilde{y}(V)\) is a \(\mathcal{C}^k\) map if and only if \(y\circ \phi\circ x^{-1} : x(U) \to y(V)\) is a \(\mathcal{C}^k\). This shows the definition is independent of the choice of charts.

We now define the maps that preserve the differentiable structure on manifolds.
\begin{definition}{Diffeomorphism}{diffeomorphism}
    The map \(\phi : M \to N\) is a \emph{diffeomorphism} if it is is bijective and the maps \(\phi\) and \(\phi^{-1}\) are smooth.
\end{definition}

\begin{definition}{Diffeomorphic manifolds}{diffeomorphic}
    Two manifolds \manifold{M} and \manifold{N} are \emph{diffeomorphic} if there exists a diffeomorphism between them.
\end{definition}

It is custom to regard diffeomorphic manifolds to be the same up to diffeomorphism. A natural question arises: how many different differentiable structures can one add on a given \(n\)-dimensional topological manifold up to diffeomorphism? The answer differs for the dimension of the manifold.
\begin{enumerate}[label=(\alph*)]
    \item The case \(1 \leq n \leq 3\). Radó-Moise theorems. There is a unique smooth manifold one can construct of a given topological manifold.
    \item The case \(n = 4\). Depending on the structure of the topological manifold, there are possibly uncountably many different smooth manifolds.
    \item The case \(n > 4\). Surgery theory (1960s). For compact manifolds, there are finitely many smooth manifolds one can make from a given topological manifold.
\end{enumerate}

\section{Tensors over a field}

Before defining tensors, we first review vector spaces.

\begin{definition}{Field}{field}
    A \emph{field} \((\mathbb{K}, +, \cdot)\) is a set \(\mathbb{K}\) equipped with two maps \(+, \cdot : \mathbb{K} \times \mathbb{K} \to \mathbb{K}\) called addition and multiplication that satisfy
    \begin{enumerate}[label=(\alph*)]
        \item Associativity of addition and multiplication: For all \(a,b,c \in \mathbb{K}\), \(a + (b + c) = (a + b) + c\) and \(a \cdot (b\cdot c) = (a\cdot b) \cdot c\);
        \item Commutativity of addition and multiplication: For all \(a,b \in \mathbb{K}\), \(a + b = b + a\) and \(a\cdot b = b\cdot a\);
        \item Additive and multiplicative identity: There exists two distinct elements \(0\) and \(1\) in \(\mathbb{K}\) such that for all \(a \in \mathbb{K}\), \(a + 0 = a\) and \(a \cdot 1 = a\);
        \item Additive inverse: For every \(a \in \mathbb{K}\) there exists an element in \(-a \in \mathbb{K}\), called the additive inverse of \(a\), such that \(a + (-a) = 0\);
        \item Multiplicative inverse: For every \(a \in \mathbb{K} \smallsetminus \set{0}\), there exists an element in \(a^{-1} \in \mathbb{K}\), called the multiplicative inverse of \(a\), such that \(a \cdot a^{-1} = 1\); and
        \item Distributivity of multiplication over addition: For all \(a, b, c \in \mathbb{K}\), \(a \cdot (b + c) = (a \cdot b) + (a\cdot c)\).
    \end{enumerate}
    Usually the multiplication \(a \cdot b\) is denoted by \(ab\).
\end{definition}
\begin{remark}
    A field is a group under addition with 0 as the additive identity, and the nonzero elements are a group under multiplication with 1 as the multiplicative identity.
\end{remark}

\begin{definition}{Vector space over a field}{vector_space}
    A \emph{vector space \((V, +, \cdot)\) over a field \(\mathbb{K}\)} is a set \(V\) equipped with two maps \(+: V \times V \to V\), called vector addition, and \(\cdot : \mathbb{K} \times V \to V\), called scalar multiplication, which satisfy
    \begin{enumerate}[label=(\alph*)]
        \item Associativity of vector addition: For all \(u,v,w \in V\), \(u + (v + w) = (u + v) + w\);
        \item Commutativity of vector addition: For all \(u,v \in V\), \(u + v = v + u\);
        \item Identity element of vector addition: There exists an element \(0 \in V\), called the zero vector, such that \(v + 0 = v\) for all \(v \in V\).
        \item Additive inverse: For every \(v \in V\) there exists an element in \(-v \in V\), called the additive inverse of \(v\), such that \(v + (-v) = 0\);
        \item Compatibility of scalar multiplication with field multiplication: For every \(a,b\in \mathbb{K}\) and \(v \in V\), \(a\cdot(b\cdot v) = (ab) \cdot v\);
        \item Identity element of scalar multiplication: For all \(v \in V,\) \(1\cdot v = v\), where 1 is the multiplicative identity of \(\mathbb{K}\);
        \item Distributivity of scalar multiplication with respect to vector addition: For all \(u, v \in V\) and  \(a\in \mathbb{K}\), \(a\cdot (u+v) = (a\cdot u) + (a\cdot v)\);
        \item Distributivity of scalar multiplication with respect to field addition: For all \(a,b \in \mathbb{K}\) and  \(v\in V\), \((a+b)\cdot v = (a\cdot v) + (b\cdot v)\).
    \end{enumerate}
    Usually the scalar multiplication \(a \cdot v\) is denoted by \(av\) and it is clear from context that it is the scalar multiplication. A vector space over a field \(\mathbb{K}\) may also be referred to a \(\mathbb{K}\)-vector space.
\end{definition}
\begin{remark}
    It is easy to verify the field \(\mathbb{K}\) is a vector space over \(\mathbb{K}\).
\end{remark}
\begin{remark}
    A vector space is an abelian additive group under vector addition, with the extra structure of the scalar multiplication.
\end{remark}

\begin{definition}{Vector subspace}{vector_subspace}
    A subset \(U \subset V\) is a \emph{vector subspace} if the vector addition and scalar multiplication are closed in \(U\). That is, for all \(u, u_1, u_2 \in U\) and \(\lambda \in \mathbb{K}\), we have \(u_1 \restrict{+}{U} u_2 \in U\) and \(\lambda \restrict{\cdot}{U} u \in U\).
\end{definition}

\begin{definition}{Linear map}{linear_map}
    Let \(V, W\) be vector spaces over a field \(\mathbb{K}\). Then a map \(f : V \to W\) is a \emph{linear map} if for all \(v, v_1, v_2 \in V\) and \(\lambda\in \mathbb{K}\), it satisfies
    \begin{enumerate}[label=(\alph*)]
        \item \(f(v_1 + v_2) = f(v_1) + f(v_2)\); and
        \item \(f(\lambda v) = \lambda f(v)\).
    \end{enumerate}
    As a shorthand, we denote \(f : V \linear M\) if \(f\) is a linear map. The map may also be referred to a \(\mathbb{K}\)-linear map, if one wants to specify the underlying field of a vector space.
\end{definition}
\begin{definition}{Vector space isomorphism}{vector_isomorphism}
    A \emph{vector space isomorphism} is a bijective linear map \(f : V \linear W\) from vector spaces \(V, W\) over a field \(\mathbb{K}\). If such a map exists, \(V\) and \(W\) are \emph{isomorphic vector spaces}.
\end{definition}

\begin{definition}{Vector space homomorphisms}{homvw}
    The set of all linear maps between vector spaces \(V, W\) over a field \(\mathbb{K}\) is denoted by \(\Hom[\mathbb{K}]{V,W},\) called the \emph{vector space homomorphisms}.
\end{definition}

\begin{proposition}{The set of vector space homomorphisms has a canonical a vector space structure}{homvw_vector_space}
    Defining the operations
    \begin{enumerate}[label=(\alph*)]
        \item \(+: \Hom[\mathbb{K}]{V,W} \times \Hom[\mathbb{K}]{V,W} \to \Hom[\mathbb{K}]{V,W}\) by the map \((f,g) \mapsto f + g\), where \(f+g : V \linear W\) is defined by \((f+g)(v) = f(v) + g(v)\) for all \(v \in V\), and
        \item \(\cdot : \mathbb{K} \times \Hom[\mathbb{K}]{V, W} \to \Hom[\mathbb{K}]{V, W}\) by the map \((\lambda, f) \mapsto \lambda f\), where \(\lambda f : V \linear W\) is defined by \((\lambda f)(v) =\lambda\cdot (f(v)) \),
    \end{enumerate}
    Then \((\Hom[\mathbb{K}]{V,W}, +, \cdot)\) is a vector space.
\end{proposition}
\begin{proof}
    We check these operations are indeed linear functions. For all \(u, v \in V\) and \(\lambda \in \mathbb{K}\), we have
    \begin{align*}
        (f+g)(u + \lambda v) &= f(u + \lambda v) + g(u + \lambda v)\\
                             &= f(u) + \lambda f(v) + g(u) + \lambda g(v)\\
                             &= (f+g)(u) + \lambda (f+g)(v),
    \end{align*}
    and
    \begin{align*}
        (\mu h)(u + \lambda v) &= \mu \cdot ( h(u + \lambda v))\\
                               &= \mu \cdot ( h(u) + \lambda h(v))\\
                               &= (\mu h)(u) + (\mu \lambda) h(v)\\
                               &= (\mu h)(u) + \lambda (\mu h(v))(u),
    \end{align*}
    where in this last step the commutativity of the field multiplication was used. That is, these operations are indeed closed in \(\Hom[\mathbb{K}]{V, W}\). The vector space axioms are then verified by computing the linear maps of \(\Hom[\mathbb{K}]{V,W}\) on arbitrary vectors of \(V\) and using the vector space axioms of \(W\).
\end{proof}

\begin{definition}{Kernel of a linear map}{kernel}
    Let \(V, W\) be vector spaces over a field \(\mathbb{K}\) and let \(T \in \Hom[\mathbb{K}]{V,W}\). The \emph{kernel} of the linear map \(T\) is the set \(\ker T = \set{v \in V : T(v) = 0}\).
\end{definition}

\begin{proposition}{Kernel and image of a linear map are vector subspaces}{kernel_image}
    Let \(V, W\) be vector spaces over a field \(\mathbb{K}\) and let \(T \in \Hom[\mathbb{K}]{V,W}\). Then the kernel and the image of \(T\) are vector subspaces of \(V\) and \(W\), respectively.
\end{proposition}
\begin{proof}
    Let \(u, v \in \ker T\) and let \(\lambda \in \mathbb{K}\). Then, by linearity, \(T(u + \lambda v) = T(u) + \lambda T(v) = 0,\) therefore \(u + \lambda v \in \ker T\). This shows the kernel is a vector subspace of \(V\).

    Let \(x, y \in T(V)\). Then, there exists \(u, v \in V\) such that \(T(u) = x\) and \(T(v) = y\). By linearity, if \(\mu \in \mathbb{K}\), we have \(T(u + \mu v) = T(u) + \mu T(v) = x + \mu y\). This shows the image is a vector subspace of \(W\).
\end{proof}

We now prove a result that will be needed in the next section.
\begin{lemma}{Trivial kernel and injective map}{trivial_kernel}
    Let \(V, W\) be vector spaces over a field \(\mathbb{K}\) and let \(T\in\Hom[\mathbb{K}]{V,W}\) be a linear map. The map \(T\) is injective if and only if the kernel is the trivial vector subspace, that is \(\ker T = \set{0} \subset V\).
\end{lemma}
\begin{proof}
    Suppose the map is injective. Then, for all \(u, v \in V\), we have \(T(u) = T(v) \implies u = v\). Clearly \(T(0) = 0\), so \(T(v) = 0 = T(0) \implies v = 0\) for all \(v \in V\). It follows that \(\ker T = \set{0}.\)

    Suppose \(\ker T = \set{0}.\) Suppose \(T(u) = T(v)\) for some \(u, v \in V\). Then \(T(u) - T(v) = T(u-v) = 0\). It follows that \(u - v \in \ker T\), so \(u = v\). Then \(T\) is injective.
\end{proof}

\begin{definition}{Endomorphism}{endomorphism}
    Let \(V\) be a vector space. An \emph{endomorphism} is a map \(f : V \linear V\) and it is an \emph{automorphism} if it is a vector space isomorphism. The set of all endomorphisms on \(V\) is denoted by \(\End(V)\), while the set of all automorphisms on \(V\) is denoted by \(\mathrm{Aut}(V)\).
\end{definition}
\begin{remark}
    Clearly, \(\End(V) = \Hom[\mathbb{K}]{V,V}\) and \(\mathrm{Aut}(V) \subset \End(V)\). By \cref{prop:homvw_vector_space}, the set of endomorphisms on \(V\) also has a canonical vector space structure. However, the set of automorphisms on \(V\) is not a vector subspace of the vector space of endomorphisms.
\end{remark}

\begin{definition}{Dual space}{dual_space}
    The vector space \(V^\ast = \mathrm{Hom}(V, \mathbb{K})\) is the \emph{dual} vector space to \(V\).
\end{definition}
\begin{remark}
    Regarding the vector space \(V\) as a base vector space, elements of \(V\) may be called vectors, while the dual vector space \(V ^{\ast}\) elements may be called covectors or linear functionals on \(V\).
\end{remark}

\subsection{Basis and Dimension}

\begin{definition}{Hamel Basis}{hamel_basis}
    Let \(V\) be a vector space. Then a subset \(\mathcal{B} \subset V\) is a \emph{(Hamel) basis} if
    \begin{enumerate}[label=(\alph*)]
        \item every finite subset \(\set{b_1, \dots b_N} \subset \mathcal{B}\) is \emph{linear independent}, that is,
            \begin{equation*}
                \sum_{i = 1}^N \lambda^i b_i = 0 \implies \lambda^1 = \dots = \lambda^N = 0;
            \end{equation*}
        \item for every vector \(v \in V\), there exists a finite subset \(\set{b_1, \dots, b_M} \subset \mathcal{B}\) and a subset \(\set{v^1, \dots, v^M} \subset \mathbb{K}\) such that \(v\) is a \emph{linear combination} of \(\set{b_1, \dots, b_M}\), that is
        \begin{equation*}
            v = \sum_{i=1}^M v^ib_i.
        \end{equation*}
        We say \(\mathcal{B}\) \emph{spans} \(V\) or \(\mathcal{B}\) is a \emph{generating set} of \(V\).
    \end{enumerate}
\end{definition}

We assume a result that is proven later in a more general setting, namely \nameref{thm:existence_of_basis}.

\begin{definition}{Vector space dimension}{vector_dimension}
    Let \(V\) be a vector space with a basis \(\mathcal{B}\). The \emph{dimension} of V, denoted by \(\dim{V}\), is equal to the cardinality of \(\mathcal{B}\). If a basis has a finite number of elements, we say \(V\) is \emph{finite dimensional}.
\end{definition}
\begin{remark}
    It is not immediate that this is well-defined. In fact, this is motivated by a theorem that  states any two different basis for a vector space have the same cardinality.
\end{remark}

In the following, we will show the relations between a finite-dimensional vector space to its dual and bidual spaces. To show these relations, we prove a lemma and an important theorem of linear maps.

\begin{lemma}{Dimension of a vector subspace}{subspace_dimension}
    Let \(V\) be a finite-dimensional vector space over \(\mathbb{K}\). If \(U \subset V\) is a vector subspace of \(V\), then \(\dim U = \dim V\) if and only if \(U = V\).
\end{lemma}
\begin{proof}
    It is obvious that if \(U = V,\) then \(\dim U = \dim V\). We now show the converse, if \(\dim U = \dim V\), then \(U = V\).

    Let \(\mathcal{B}\) be a basis for \(U\). Then \(\mathcal{B}\) is linear independent and spans \(U\). Suppose, by contradiction, \(\mathcal{B}\) is not a basis for \(V\). This implies there exists \(v \in V\) that is not a linear combination of the elements of \(\mathcal{B}\). As a result, \(\mathcal{B} \cup \set{v}\) is a linearly independent subset of \(V\) with \(1 + \dim U > \dim V\) elements. This contradiction shows that \(\mathcal{B}\) is a basis for \(V\).
\end{proof}
\begin{remark}
    This lemma guarantees that \(U \subset V \implies \dim U \leq \dim V\), with equality implying \(U = V\).
\end{remark}

We may now prove the theorem that relates the dimensions of the domain, kernel and image of a linear map.

\begin{theorem}{Rank-nullity theorem}{rank_nullity}
    Let \(V, W\) be vector spaces over a field \(\mathbb{K}\) and let \(T \in \Hom[\mathbb{K}]{V,W}\). If \(V\) is finite dimensional, we have \(\dim V = \dim \ker T + \dim T(V)\).
\end{theorem}
\begin{proof}
    Since the kernel is a subspace of \(V\), we have \(\dim V \geq \dim \ker T,\) by the lemma. We first consider \(\dim V = \dim \ker T\). This implies \(T(V) = \set {0} \subset W\) and the statement follows. We may now assume \(\dim V > \dim \ker T\).

    Let \(\mathcal{B}_{\ker T} = \set{e_1, \dots, e_n}\) be a basis for \(\ker T,\) where \(n = \dim\ker T\) it the nullity of \(T\). We may complete this basis such that the resulting set is a basis \(\mathcal{B} = \mathcal{B}_{\ker T} \cup \set{e_{n+1}, \dots, e_{n+m}}\) for \(V\), with \(\dim V = m + n\) and \(m \geq 1\). In particular, for all \(v \in V\) there exists a family \ffamily{v^i}{i=1}{n+m} in \(\mathbb{K}\) such that
    \begin{equation*}
        v = \sum_{i=1}^{n+m} v^ie_i,
    \end{equation*}
    since \(\mathcal{B}\) is a generating set for \(V\).

    We consider \(u \in T(V)\). Then there exists \(v \in V\) such that \(T(v) = u\), that is
    \begin{align*}
        u &= T\left(\sum_{i=1}^{n+m} v^ie_i\right)\\
          &= \sum_{i=1}^{n+m} v^i T(e_i)\\
          &= \sum_{i=1}^{m} v^{i+n} T(e_{i+n}),
    \end{align*}
    since \(e_i \in \ker T\) for \(i \leq n\). Equivalently, \(\mathcal{B}_{T(V)} = \set{T(e_{n+1}), \dots, T(e_{n+m})} \subset T(V)\) is a generating set for \(T(V)\).

    Consider \(0 \in T(V) \subset W\). Since \(\mathcal{B}_{T(V)}\) spans \(T(V),\) there exists a family \ffamily{\lambda^i}{i=1}{m} in \(\mathbb{K}\) such that
    \begin{equation*}
        \sum_{i=1}^m \lambda^i T(e_{i+n}) = 0.
    \end{equation*}
    By linearity, this implies \(w = \sum_{i=1}^m \lambda^ie_{i+n} \in \ker T\). Since \(e_{i+n}\) is not a linear combination of \(\mathcal{B}_{\ker T},\) we must have \(\lambda^i = 0\), for all \(1 \leq i \leq m\). Therefore, \(\mathcal{B}_{T(V)}\) is linearly independent and it is a basis of \(T(V)\) with \(m\) elements and the theorem follows.
\end{proof}
\begin{remark}
    The \emph{nullity} and \emph{rank} of a linear map are the dimensions of its kernel and image. We note \(\dim \ker T\) and \(\dim T(V)\) are well-defined due to \cref{prop:kernel_image}.
\end{remark}
\begin{remark}
    It is not immediate it is possible to complete a basis given a linearly independent subset, this is related to \nameref{thm:existence_of_basis}.
\end{remark}
\begin{corollary}
    If both vector spaces are finite-dimensional with \(\dim V = \dim W\) and if \(T\) is injective, then \(T\) is an isomorphism.
\end{corollary}
\begin{proof}
    Since \(T\) is one-to-one, its nullity is zero by \cref{lem:trivial_kernel}. Then, by the \nameref{thm:rank_nullity}, we have \(\dim T(V) = \dim V = \dim W\). Since \(T(V) \subset W\) is a vector subspace of \(W\), it follows from \cref{lem:subspace_dimension} that \(T(V) = W,\) that is, \(T\) is onto.
\end{proof}

\begin{theorem}{Dual vector space dimension}{dual_space_dimension}
    Let \(V\) be a finite-dimensional vector space over a field \(\mathbb{K}\). Then \(\dim V = \dim V ^{\ast}\).
\end{theorem}
\begin{proof}
    Let \(n = \dim V\) and let \(\mathcal{B} = \set*{e_1, \dots, e_n}\) be a basis for \(V\). Then, define  \(\mathcal{B}^{\ast} = \set*{\epsilon^1, \dots, \epsilon^n}\), a subset of maps from \(V \to \mathbb{K}\), by letting \(\epsilon^i\left(\sum_{j=1}^{n} c^je_j\right) = c^i\), where \(c^i \in \mathbb{K}\) for \(i = 1, \dots, n\).

    First, we show that \(\epsilon^i\) is indeed an element of \(V ^{\ast}\). Let \(x, y \in V\) with \(x = \sum_{i=1}^n x^i e_i\) and \(y = \sum_{i=1}^n y^i e_i\) and \(x^i,y^i \in \mathbb{K}\). By definition of the maps \(\epsilon^i\), we have \(\epsilon^i(x) = x^i\) and \(\epsilon^i(y) = y^i\), for \(i = 1, \dots, n\). Let \(\lambda \in \mathbb{K}\), then
    \begin{align*}
        \epsilon^i\left(x + \lambda y\right) &= \epsilon^i\left[\sum_{j = 1}^n (x^j + \lambda y^j)e_j\right]\\
                                             &= x^i + \lambda y^i\\
                                             &= \epsilon^i(x) + \lambda \epsilon^i(y),
    \end{align*}
    that is, \(\epsilon^i\) is a linear map. Therefore, \(\mathcal{B}^{\ast} \subset V ^{\ast}\).

    We consider the linear combination \(\omega = \sum_{i = 1}^n \omega_i \epsilon^i \in V ^{\ast}\), with \(\omega_i \in \mathbb{K}\). The dual vector \(\omega\) is the zero dual vector if \(\omega(v) = 0\) for all \(v\). We may choose \(v\) as each element of the basis \(\mathcal{B}\), that is, if \(v = e_j\), then \(v^i = \delta^i_j,\) where
    \begin{equation*}
        \delta_{j}^{i} = \begin{cases}
            1, & \text{ if } j = i\\
            0, & \text{ if } j\neq i
        \end{cases}
    \end{equation*}
    is the \emph{Kronecker delta}, for \(j = 1, \dots, n\). As consequence, we have \(\omega_j = 0\) for \(j = 1, \dots, n\), therefore \(\mathcal{B}^{\ast}\) is linearly independent.

    We consider a dual vector \(\varphi \in V ^{\ast}\). Then, for all \(u = \sum_{i =1}^n u^ie_i \in V\), we have \(\epsilon^i(u) = u^i\) and
    \begin{align*}
        \varphi(u) &= \varphi\left(\sum_{i=1}^n u^ie_i\right)\\
                   &= \varphi\left(\sum_{i=1}^n \epsilon^i(u) e_i\right)\\
                   &= \sum_{i=1}^n \varphi(e_i)\epsilon^i(u),
    \end{align*}
    that is, \(\varphi = \sum_{i=1}^n \varphi(e_i) \epsilon^i\). Then \(\mathcal{B}^{\ast}\) is a generating set of \(V^{\ast}\).

    We have shown \(\mathcal{B}^{\ast}\) is a basis for \(V ^{\ast}\), therefore \(\dim V ^{\ast}= n\) and the theorem follows.
\end{proof}
\begin{remark}
    The construction used in this proof will be used extensively: in lieu of making arbitrary choices of basis for both \(V\) and \(V ^{\ast}\), only the choice of basis in \(V\) is needed, and we have an induced basis on \(V ^{\ast}\), henceforth named \emph{dual basis}.
\end{remark}
\begin{remark}
    The proof that two finite-dimensional vector spaces are isomorphic if and only if their dimensions are equal is very similar.
\end{remark}

\begin{theorem}{Bidual vector space canonical linear isomorphism}{double_dual_space}
    Let \(V\) be a finite-dimensional vector space over a field \(\mathbb{K}\). Then there exists a canonical linear isomorphism from \(V\) to the bidual vector space \((V^{\ast})^{\ast}\).
\end{theorem}
\begin{proof}
    We remind ourselves the bidual vector space is the set of linear maps from the dual space to the field vector space, that is
    \begin{equation*}
        (V^{\ast})^{\ast} = \set*{\phi : V^{\ast}\to \mathbb{K}\text{ such that }\phi\text{ is linear}}.
    \end{equation*}

    We consider the map
    \begin{align*}
        \psi : V &\to (V^{\ast})^{\ast}\\
        v &\mapsto \psi(v),
    \end{align*}
    where \(\psi(v) \in (V ^{\ast})^{\ast}\) is the linear map
    \begin{align*}
        \psi(v) : V^{\ast} &\linear \mathbb{K}\\
        \omega & \mapsto \omega(v).
    \end{align*}
    Since this definition requires no additional structure, this map is canonically defined: no choice of basis in what follows taints this.

    First, we show the map is linear. Let \(u, v \in V\) and \(\lambda \in \mathbb{K}\), then we let \(\psi(u + \lambda v)\) act on a dual vector \(\omega\in V^{\ast}\)
    \begin{align*}
        \psi(u + \lambda v)(\omega) &= \omega(u + \lambda v)\\
                                    &= \omega(u) + \lambda \omega(v)\\
                                    &= \psi(u)(\omega) + \lambda \psi(v)(\omega),
    \end{align*}
    that is, \(\psi(u + \lambda v) = \psi(u) + \lambda \psi(v)\). Therefore, \(\psi : V \linear (V ^{\ast})^{\ast} \) is a linear map.

    We now show this map is injective. Let \(v \in V\) such that \(\psi(v)\) is the null map. Suppose, by contradiction, \(v \neq 0\). The subset \(\set{v}\subset V\) is linearly independent and we may complete this set to be a basis for \(V\). Then, let \(v ^{\ast} \in V ^{\ast}\) be the element in the dual basis such that \(v ^{\ast}(v) = 1\). We let the null map \(\psi(v)\) act on \(v ^{\ast}\), arriving at at contradiction. This shows \(v = 0\), that is, \(\psi\) is injective.

    Noting \((V ^{\ast})^{\ast}\) is the dual space of \(V ^{\ast}\), we have \(\dim V = \dim V ^{\ast} = \dim (V^{\ast})^{\ast}\) by \cref{thm:dual_space_dimension}. By \cref{thm:rank_nullity}, it follows that \(\psi\) is a bijection.
\end{proof}
\begin{remark}
    Since for a finite-dimensional vector space \(V\) there is a natural isomorphism from the vector space to its bidual, if \(v \in V\) and \(\omega \in V ^{\ast}\), one may write \(\omega(v) = v(\omega)\), where \(v(\omega) = \psi(v)(\omega)\).
\end{remark}

\subsection{Tensor spaces}

With the notion of vector spaces and dual vector spaces, we may construct functions of multiple vector variables, known as tensors.

\begin{definition}{Tensors on a vector space}{tensor_over_field}
    Let \(V\) be a vector space over a field \(\mathbb{K}\). An element of the vector space defined by the set
    \begin{equation*}
        \underbrace{V \otimes \dots \otimes V}_{r \text{ times}} \otimes \underbrace{V^\ast \otimes \dots \otimes V^\ast}_{s \text{ times}} = \set*{\underbrace{V^\ast \times \dots \times V^\ast}_{r \text{ times}} \times \underbrace{V \times \dots \times V}_{s \text{ times}} \to \mathbb{K} : T \text{ is multilinear}}
    \end{equation*}
    is a \emph{\((r,s)\)-tensor on the vector space \(V\)}. A map is \emph{multilinear} if it is linear on each argument. The pair \((r,s)\) is called the \emph{valence} of the tensor. As a shorthand, we denote the set of \((r,s)\)-tensors on the vector space \(V\) as \(T_s^rV\).
\end{definition}

Just as \(\Hom[\mathbb{K}]{V,W}\) had a canonical vector space structure, we show a similar result in \cref{prop:tensor_over_field_vector_space}.
\begin{proposition}{Tensors have a canonical vector space structure}{tensor_over_field_vector_space}
    Let \(V\) be a vector space over a field \(\mathbb{K}\). The set \(T_s^rV\) together with the operations
    \begin{enumerate}[label=(\alph*)]
        \item \(+: T_s^rV \times T_s^rV \to T_s^rV\) defined by \((T,S) \mapsto T+S\) where
            \begin{equation*}
                \hspace{-7pt}%overfull hbox I hate this
                (T+S)(\omega_1, \dots, \omega_r, v_1, \dots, v_s) = T(\omega_1, \dots, \omega_r, v_1, \dots, v_s) + S(\omega_1, \dots, \omega_r, v_1, \dots, v_s),
            \end{equation*}
        \item \(\cdot: \mathbb{K} \times T_s^rV \to T_s^rV\) defined by \((\lambda,T) \mapsto \lambda T\) where
            \begin{equation*}
                (\lambda T)(\omega_1, \dots, \omega_r, v_1, \dots, v_s) = \lambda \cdot \left(T(\omega_1, \dots, \omega_r, v_1, \dots, v_s)\right),
            \end{equation*}
    \end{enumerate}
    is a vector space.
\end{proposition}
\begin{proof}
    We verify the well-definitions of the operations above. Let \(T, S \in T_s^r\) and \(\lambda \in \mathbb{K}\). We consider a family \ffamily{\omega_i}{i=1}{r} of \(r\) covectors, a family of \ffamily{v_i}{i=1}{s} vectors. Without loss of generality, we verify the linearity on an arbitrary argument, say the first argument and we abbreviate \(T(\omega_1, \dots)\) to denote \(T(\omega_1, \dots, \omega_r, v_1, \dots, v_s)\). Let \(\sigma \in V ^{\ast}\) and \(\mu \in \mathbb{K}\). We have
    \begin{align*}
        (T+S)(\omega_1+\mu \sigma, \dots) &= T(\omega_1 + \mu \sigma, \dots) + S(\omega_1 + \mu \sigma, \dots)\\
                                          &= T(\omega_1, \dots) + S(\omega_1, \dots) + \mu T(\sigma, \dots) + \mu S(\sigma, \dots)\\
                                          &= (T+S)(\omega_1, \dots) + \mu (T+S)(\sigma, \dots),
    \end{align*}
    and
    \begin{align*}
        (\lambda T)(\omega_1 + \mu\sigma, \dots) &= \lambda \cdot \left(T(\omega_1 + \mu \sigma, \dots)\right),\\
                                                 &= \lambda \cdot \left(T(\omega_1, \dots) + \mu T(\sigma, \dots)\right)\\
                                                 &= (\lambda T)(\omega_1, \dots) + \mu (\lambda T)(\sigma, \dots),
    \end{align*}
    where we have used the commutativity of field multiplication. This shows the operations yield indeed multilinear maps, and as such are well-defined. We could then verify the vector space axioms by letting \((r,s)\)-tensors act on the families of vectors and covectors and using the vector space axioms on \(\mathbb{K}\).
\end{proof}

We may construct new tensors from given tensors on the same vector space with possibly different valences.
\begin{definition}{Tensor product}{tensor_product}
    Let \(V\) be a vector space over \(\mathbb{K}\). The \emph{tensor product} is the map \(\otimes : T_q^p V \times T_s^rV \to T_{q+s}^{p+r}V\) defined by
    \begin{align*}
        (T \otimes S)&\left(\omega_1, \dots, \omega_p, \omega_{p+1}, \dots, \omega_{p+r}, v_1, \dots, v_q, v_{q+1}, \dots, v_{q+s}\right) \\
        &= T\left(\omega_1, \dots, \omega_p, v_1, \dots, v_q\right) \cdot S\left(\omega_{p+1}, \dots, \omega_{p+r}, v_{q+1}, \dots, v_{q+s}\right).
    \end{align*}
\end{definition}

\begin{example}
    Let \(V\) be a vector space over a field \(\mathbb{K}\).
    \begin{enumerate}[label=(\alph*)]
        \item By convention, we set \(T_0^0 V = \mathbb{K}\).
        \item \(T_1^0V = V^\ast\). That is, \((0,1)\)-tensors are elements of the dual vector space.
        \item \(T_1^1V = V \otimes V^\ast = \set{V^\ast \times V \to \mathbb{K} : T\text{ is multilinear}}\). We will show that this is isomorphic to \(\End(V^\ast)\). That is, given \(T \in V \otimes V^\ast\) we may construct \(\tilde{T} \in \End(V^\ast)\) by setting \(\tilde{T}(\omega) = T(\omega, \cdot)\). Similarly, given \(\tilde{T} \in \End(V^\ast)\) we may reconstruct \(T\) by setting \(T(\omega, v) = \tilde{T}(\omega)(v)\).
        \item Similarly, \(T_1^1V\) is isomorphic to \(\End(V)\) if \(V\) is finite-dimensional, due to \cref{thm:double_dual_space}. In fact, one may check the map \(\Psi : \End(V) \to T_1^1 V\) defined by \((\Psi(\phi))(\omega, v) = \omega(\phi(v))\) is linear and bijective.
        \item By \cref{thm:double_dual_space}, if \(V\) is finite-dimensional, then \(T_0^1 V = \set{V ^{\ast} \to \mathbb{K} : T \text{ is multilinear}} = (V ^{\ast})^{\ast}\) is isomorphic to \(V\).
    \end{enumerate}
\end{example}

\begin{definition}{Components of a tensor}{tensor_components}
    Let \(T\in T_s^r V\), where \(\dim V = n\). Let \(\set{e_1, \dots, e_n}\) be a basis of \(V\) and let \(\set{\epsilon^1, \dots, \epsilon^n}\) be the dual basis of \(V ^{\ast}\). Then the \emph{components of \(T\) with respect to the chosen basis} are
    \begin{equation*}
        T\indices{^{i_1, \dots, i_r}_{j_1, \dots, j_s}} = T\left(\epsilon^{i_1}, \dots, \epsilon^{i_r}, e_{j_1}, \dots, e_{j_r}\right) \in \mathbb{K},
    \end{equation*}
    with indices \(i_k, j_l \in \set{1, \dots, n}\) for all \(k \in \set{1, \dots, r}\) and \(l \in \set{1, \dots, s}\).
\end{definition}

\begin{proposition}{Components determine a tensor}{tensor_components}
    Let \(V\) be an \(n\)-dimensional vector space over a field \(\mathbb{K}\). Let \(\set{e_1, \dots, e_n}\) be a basis of \(V\) and let \(\set{\epsilon^1, \dots, \epsilon^n}\) be the dual basis of \(V ^{\ast}\). The tensor \(T \in T_s^rV\) with components \(T\indices{^{i_1, \dots, i_r}_{j_1, \dots, j_s}}\), with indices \(i_k, j_l \in \set{1, \dots, n}\) for all \(k \in \set{1, \dots, r}\) and \(l \in \set{1, \dots, s}\), is given by
    \begin{equation*}
        T = \sum_{i_1 = 1}^{n} \dots \sum_{i_s = 1}^{n} \sum_{j_1 = 1}^{n}\dots \sum_{j_s = 1}^n T\indices{^{i_1, \dots, i_r}_{j_1, \dots, j_s}} e_{i_1} \otimes \dots \otimes e_{i_r} \otimes \epsilon^{j_1} \otimes \dots \otimes \epsilon^{j_r}.
    \end{equation*}
\end{proposition}
\begin{proof}
    Let \ffamily{a_i}{i=1}{r} and \ffamily{b_i}{i=1}{s} be families of indices in \(\set{1, \dots, n}\). We verify the tensor \(T\) has indeed those components by letting it act on the vector and dual basis elements:
    \begin{align*}
        T\indices{^{a_1, \dots, a_r}_{b_1, \dots, b_s}} &= T(\epsilon^{a_1}, \dots, \epsilon^{a_r}, e_{b_1}, \dots, e_{b_s})\\
                                                        &= \sum_{i_1 = 1}^{n} \dots \sum_{i_s = 1}^{n} \sum_{j_1 = 1}^{n}\dots \sum_{j_s = 1}^n T\indices{^{i_1, \dots, i_r}_{j_1, \dots, j_s}} e_{i_1}(\epsilon^{a_1}) \cdot \dotso \cdot e_{i_r}(\epsilon^{a_r}) \cdot \epsilon^{j_1}(e_{b_1}) \cdot \dotso \cdot \epsilon^{j_r}(e_{b_r})\\
                                                        &= \sum_{i_1 = 1}^{n} \dots \sum_{i_s = 1}^{n} \sum_{j_1 = 1}^{n}\dots \sum_{j_s = 1}^n T\indices{^{i_1, \dots, i_r}_{j_1, \dots, j_s}} \delta^{a_1}_{i_1}\cdot \dotso \cdot \delta^{a_r}_{i_r}\cdot \delta^{j_1}_{b_1}\cdot \dotso \cdot \delta^{j_s}_{b_s}\\
                                                        &= T\indices{^{a_1, \dots, a_r}_{b_1, \dots, b_s}},
    \end{align*}
    and we have recovered the desired components.
\end{proof}

From now on, the \emph{Einstein summation notation} will be used. In this notation, we use the convention that \emph{basis vectors of \(V\) are labeled by lower indices} and \emph{dual basis covectors are labeled by upper indices}, as was used in the previous definition. The summation convention is to omit the sum signs over an index whenever it appears as an upper and lower index in the same product. As an example, instead of writing \(v = v^1e_1 + \dots + v^n e_n\), we simply write
\begin{equation*}
    v = v^i e_i,
\end{equation*}
and the summation over \(i\) from 1 to \(n\) is implied. Likewise, the expression in \cref{prop:tensor_components} is simplified to
\begin{equation*}
    T = T\indices{^{i_1, \dots, i_r}_{j_1, \dots, j_s}} e_{i_1} \otimes \dots \otimes e_{i_r} \otimes \epsilon^{j_1} \otimes \epsilon^{j_r},
\end{equation*}
and the summation over all the indices are implied.

We note however that the convention only works with linear spaces and (multi)linear maps. We consider a \((1,1)\)-tensor \(\phi : V ^{\ast} \times V \linear K\) acting on \(\omega \in V ^{\ast}\) and \(v \in V\), writing each step with and without the summation notation:

\begin{equation*}
    \begin{aligned}
        \phi(\omega, v) &= \phi \left(\sum_{i = 1}^n \omega_i \epsilon^i, \sum_{j = 1}^n v^j e_j\right)&&= \phi\left(\omega_i\epsilon^i, v^j e_j\right)\\
                        &= \sum_{i=1}^n\sum_{j=1}^n \phi\left(\omega_i \epsilon^i, v^je_j\right) &&= \phi\left(\omega_i\epsilon^i, v^j e_j\right)\\
                        &= \sum_{i=1}^n \sum_{j=1}^n \omega_i v^j \phi\left(\epsilon^i, e_j\right)&&=\omega_i v^j \phi\left(\epsilon^i, e_j\right)\\
                        &= \sum_{i=1}^n \sum_{j=1}^n \omega_i v^j \phi\indices{^i_j} &&=\omega_i v^j \phi\indices{^i_j}.
    \end{aligned}
\end{equation*}
Notice that the first step relies on the multilinearity of \(\phi\), but in the summation notation it is not clear \emph{where} the implied summations are happening. This example illustrates the use of the summation convention and serves as a warning that it is well-defined only for linear structures.

\subsection{Change of Basis}

Let \(V\) be an \(n\)-dimensional vector space over a field \(\mathbb{K}\) with a basis \(\mathcal{B} = \set{e_1, \dots, e_n}\) of \(V\). We now consider another basis \(\tilde{\mathcal{B}}=\set{\tilde{e}_1, \dots, \tilde{e}_n}\) of \(V\). Since the elements of \(\tilde{\mathcal{B}}\) are vectors of \(V\), there exists \(A\indices{^i_j} \in \mathbb{K}\) such that
\begin{equation*}
    \tilde{e}_j = A\indices{^i_j}e_i,
\end{equation*}
for \(j = 1, \dots, n\). Likewise, elements of \(\mathcal{B}\) can be expanded in terms of their components in the \(\tilde{\mathcal{B}}\) basis, that is
\begin{equation*}
    e_j = B\indices{^i_j}\tilde{e}_i,
\end{equation*}
for some \(B\indices{^i_j} \in \mathbb{K}\), for \(j = 1, \dots, n\). In this case, we must have \(A\indices{^i_j}B{^j_k} = \delta_{k}^{i}.\) That is, there exists an automorphism \(A : V \linear V\)that relates the basis \(\mathcal{B}\) to the basis \(\tilde{\mathcal{B}}\) with components \(A\indices{^i_j}\) and its inverse \(B = A^{-1} : V \linear V\) has components \(B\indices{^i_j}\).

We now see how the dual basis is modified under this change of basis. Let \(\tilde{\epsilon}^i = C\indices{^i_j}\epsilon^j\) and then determine the components \(C\indices{^i_j}\) with
\begin{align*}
    \tilde{\epsilon}^i(\tilde{e}_j) &= A\indices{^k_j}C\indices{^i_l}\epsilon^l(e_k)\\
    \delta^i_j &= A\indices{^k_j}C\indices{^i_l}\delta^l_k\\
    \delta^i_j &= C\indices{^i_k}A\indices{^k_j},
\end{align*}
which implies \(C\indices{^i_k} = B\indices{^i_k}\). That is, \(\tilde{\epsilon}^i = B\indices{^i_j}\epsilon^j\) and \(\epsilon^i = A\indices{^i_j}\epsilon^j.\)

Let \(\omega = \omega_i \epsilon^i \in V ^{\ast}\) and \(v = v^i e_i\) in the basis \(\mathcal{B}\). We now determine the relation between the components of these objects in the basis \(\tilde{\mathcal{B}}\). For the covector, we have
\begin{equation*}
    \omega_i = \omega(e_i) = \omega(B\indices{^j_i}\tilde{e}_j) = B\indices{^j_i}\tilde{\omega}_j,
\end{equation*}
and for the vector,
\begin{equation*}
    v^i = v(\epsilon^i) = \epsilon^i(v) = \epsilon^i(\tilde{v}^j \tilde{e}_j) = A\indices{^k_j} \tilde{v}^j \epsilon^i(e_k) = A\indices{^i_j}\tilde{v}^j.
\end{equation*}
We see the components of the covector \(\omega\) change just like the basis vectors \(e_i\) do and the components of the vector \(v\) change as the dual basis vectors \(\epsilon^i\) do. More generally, for a (r,s)-tensor, the upper indices change like vector components and lower indices like covector components:
\begin{equation*}
    T\indices{^{a_1,\dots,a_r}_{b_1,\dots,b_s}} = A\indices{^{a_1}_{m_1}}\dots A\indices{^{a_r}_{m_r}} B\indices{^{n_1}_{b_1}} \dots B\indices{^{n_s}_{b_s}} \tilde{T}\indices{^{m_1, \dots, m_r}_{n_1, \dots, n_s}}.
\end{equation*}

\subsection{Determinants}

Let \(V\) be a \(n\)-dimensional vector space over a field \(\mathbb{K}\).

\begin{definition}{\(m\)-form on a vector space}{m-form}
    An \(m\)-form is a \(T_m^0V\) tensor \(\omega\) that is totally anti-symmetric. That is, let \(\pi\) be a permutation of the permutation group \(S^m\), then
    \begin{equation*}
        \omega(v_1, \dots, v_m) = \mathrm{sgn}(\pi)\cdot \omega(v_{\pi(1)}, \dots, v_{\pi(m)}).
    \end{equation*}
\end{definition}
\begin{remark}
    In case \(m = 0\), \(\omega \in \mathbb{K}\) and in the case \(m > d\), \(\omega\) is the null tensor. The special case \(m = n\) is called \emph{volume form} and it can be shown that if \(\omega\) and \(\omega'\) are two non-vanishing volume forms, then there exists \(\lambda \in \mathbb{K}\) such that \(\omega' = \lambda\omega\).
\end{remark}

\begin{definition}{Volume form}{volume_form}
    A choice of a non-vanishing volume form \(\omega\) is called a \emph{choice of volume on \(V\)}. Let \(\mathcal{F} = \ffamily{v_i}{i=1}{n}\) be a family of \(n\) vectors in \(V\), then
    \begin{equation*}
        \mathrm{vol}(v_1, \dots, v_n) = \omega(v_1, \dots, v_n)
    \end{equation*}
    is the \emph{volume spanned by \(\mathcal{F}\)}.
\end{definition}
\begin{remark}
    It follows from the anti-symmetries of the volume form \(\omega\) that \(\mathcal{F}\) is not linearly independent if and only if the volume spanned by \(\mathcal{F}\) is zero. Equivalently, \(\mathcal{F}\) is a basis if and only if the volume spanned by \(\mathcal{F}\) is non-zero.
\end{remark}

\begin{definition}{Determinant of an endomorphism on \(V\)}{determinant}
    Let \(\phi \in \End(V) \cong T_1^1 V\). The determinant of \(\phi\) is defined as
    \begin{equation*}
        \det \phi = \frac{\mathrm{vol}(\phi(e_1), \dots, \phi(e_n))}{\mathrm{vol}(e_1, \dots, e_n)}
    \end{equation*}
    for some choice of volume on \(V\) and some basis \(\set{e_1, \dots, e_n}\) of \(V\).
\end{definition}

\begin{proposition}{Determinant is well-defined}{determinant}
    The determinant is well defined.
\end{proposition}
\begin{proof}
    Let \(\omega\) and \(\omega'\) be two choices of volume. Then, there exists \(\lambda \in \mathbb{K}\) such that \(\omega' = \lambda \omega\). Then, for a basis \(\mathcal{B} = \set{e_1, \dots, e_n}\) and a family of \(n\) vectors \ffamily{v_i}{i=1}{n}, we have
    \begin{equation*}
        \frac{\omega(v_1, \dots, v_n)}{\omega(e_1, \dots, e_n)} = \frac{c\omega'(v_1, \dots, v_n)}{c\omega'(e_1, \dots, e_n)} = \frac{\omega'(v_1, \dots, v_n)}{\omega'(e_1, \dots, e_n)}.
    \end{equation*}
    In particular, the determinant is invariant by choice of volume.

    To show independence from choice of basis, we first compute the determinant of an endomorphism \(\phi : V \to V\) with components \(\phi\indices{^i_j}\) in the basis \(\mathcal{B}.\) We have
    \begin{align*}
        \omega\left(\phi(e_1), \dots, \phi(e_n)\right) &= \omega\left(\phi\indices{^{k_1}_1}e_{k_1}, \dots, \phi\indices{^{k_n}_n} e_{k_n}\right)\\
                                                       &= \phi\indices{^{k_1}_{1}} \cdot \dots \cdot \phi\indices{^{k_n}_{n}} \omega\left(e_{j_1}, \dots, e_{j_n}\right)\\
                                                       &= \sum_{\sigma \in S_n} \sgn(\sigma)\phi\indices{^{\sigma(1)}_{1}}\cdot\dots\cdot\phi\indices{^{\sigma(n)}_{n}} \omega(e_1, \dots, e_n),
    \end{align*}
    since the volume spanned by a linearly dependent family of vectors is zero. As a result, we have
    \begin{equation*}
        \frac{\omega\left(\phi(e_1), \dots, \phi(e_n)\right)}{\omega(e_1, \dots, e_n)} = \sum_{\sigma \in S_n} \sgn(\sigma) \phi\indices{^{\sigma(1)}_1} \cdot \dots \cdot \phi\indices{^{\sigma(n)}_n},
    \end{equation*}
    where the choice of basis is implied by the components of the endomorphism.

    Let \(\tilde{\mathcal{B}} = \set{\tilde{e}_1, \dots, \tilde{e}_n}\) be another basis. Then, there exists a linear map \(A \in \mathrm{Aut}(V)\) such that \(\tilde{e}_i = A\indices{^j_i}e_j,\) and a linear map \(B = A^{-1}\) such that \(e_i = B\indices{^j_i}\tilde{e}_j\) and \(A\indices{^i_j}B\indices{^j_k} = \delta^i_k.\) In the basis \(\tilde{\mathcal{B}}\), the components of the map \(\phi\) are given by
    \begin{equation*}
        \tilde{\phi}\indices{^i_j} = A\indices{^a_j}B\indices{^i_b}\phi\indices{^b_a},
    \end{equation*}
    and by the same computations done in the basis \(\mathcal{B},\) we have
    \begin{align*}
        \frac{\omega\left(\phi(\tilde{e}_1), \dots, \phi(\tilde{e}_n)\right)}{\omega(\tilde{e}_1, \dots, \tilde{e}_n)} &= \sum_{\sigma \in S_n} \sgn(\sigma) \tilde{\phi}\indices{^{\sigma(1)}_{1}} \cdot \dots \cdot \tilde{\phi}\indices{^{\sigma(n)}_{n}}\\
                                                                                                                               &= \sum_{\sigma \in S_n} \sgn(\sigma) A\indices{^{a_1}_1}B\indices{^{\sigma(1)}_{b_1}}\phi\indices{^{b_1}_{a_1}}
                                                                                                                           \cdot \dots \cdot A\indices{^{a_n}_n}B\indices{^{\sigma(n)}_{b_n}}\phi\indices{^{b_n}_{a_n}}.
    \end{align*}
    \todo
\end{proof}

Find a way to calculate determinant of a bilinear map. Tensor density.


\begin{remark}
    Having chosen a basis, it is tempting to consider elements of vector spaces as collections of elements of \(\mathbb{K}\) arranged in some matricial representation. As an example, we may choose the following convention
    \begin{equation*}
        \begin{aligned}
            \omega = \omega_i \epsilon^i \in V ^{\ast}& \leftrightsquigarrow & \omega &\doteq \begin{pmatrix}\omega^1 & \dots & \omega^n\end{pmatrix}\\
            v = v^ie_i \in V & \leftrightsquigarrow & v &\doteq \begin{pmatrix}v^1\\\vdots\\v^n\end{pmatrix}\\
            \phi = \phi\indices{^i_j} e_i \otimes \epsilon^j\in T_1^1V & \leftrightsquigarrow & \phi &\doteq \begin{pmatrix}\phi\indices{^1_1}&\phi\indices{^1_2}&\dots&\phi\indices{^1_n}\\\phi\indices{^2_1}&\phi\indices{^2_2}&\dots&\phi\indices{^2_n}\\\vdots&\vdots&\ddots&\vdots\\\phi\indices{^n_1}&\phi\indices{^n_2}&\dots&\phi\indices{^n_n}\\\end{pmatrix},
        \end{aligned}
    \end{equation*}
    that is, covectors are represented by row matrices, vectors by column matrices, and the (1,1)-tensor \(\phi : V \to V\) has its components \(\phi\indices{^i_j}\) arranged in a square matrix where the first (upper) index \(i\) indicates the row, and second (lower) index \(j\), the column. First, we identify \(\End(V)\) with \(T_1^1V\), that is \(\phi(\epsilon^i, e_j) = \epsilon^i(\phi(e_j))\) by abuse of notation: in the left hand side, \(\phi\) is understood as (1,1)-tensor and on the right hand side, as an endomorphism on \(V\) and therefore
    \begin{equation*}
        \phi\indices{^i_j} = \phi(\epsilon^i, e_j) = \epsilon^i(\phi(e_j)) \implies \phi(e_j) = \phi\indices{^k_j}e_k.
    \end{equation*}
    To motivate this convention, if \(\phi, \psi \in \End(V) \cong T_1^1\), then \(\phi \circ \psi \in \End(V)\) and we have
    \begin{align*}
        (\phi\circ\psi)(\epsilon^i, e_j) &= \epsilon^i(\phi\circ\psi(e_j))\\
        (\phi \circ \psi)\indices{^i_j} &= \epsilon^i(\phi(\psi(e_j)))\\
                                         &= \epsilon^i(\phi(\psi\indices{^k_j}e_k))\\
                                         &= \psi\indices{^k_j}\epsilon^i(\phi(e_k))\\
                                         &= \psi\indices{^k_j}\phi\indices{^i_k}\\
                                         &= \phi\indices{^i_k}\psi\indices{^k_j},
    \end{align*}
    and this last line may be interpreted as the matrix multiplication of the matrix representations of \(\phi\) and \(\psi\), in this order. Similarly, if \(\omega \in V ^{\ast}\) and \(v \in V,\) then \(\omega(v) = \omega_i v^i\) may be interpreted as the matrix multiplication of the row matrix that represents \(\omega\) by the column matrix that represents \(v\). Finally, we consider \(\phi(\omega, v)\)
    \begin{align*}
        \phi(\omega, v) &= \phi(\omega_i \epsilon^i, v^je_j)\\
                        &= \omega_i v^j \phi\indices{^i_j}\\
                        &= \omega_i \phi\indices{^i_j} v^j,
    \end{align*}
    and the result may be interpreted as the matrix multiplication of a row matrix, by the square matrix and finally by the column matrix.

    However, determinants \todo
\end{remark}

\section{Tangent spaces to a manifold}

We suppress the notation of a smooth manifold \manifold{M} to simply its set \(M,\) unless there are more manifolds and their underlying structure need be emphasized.

Let \(M\) be a smooth manifold. We equip the set of smooth functions from \(M\) to \(\mathbb{R}\) with point-wise addition and scalar multiplication, and we claim \((\smooth{M}, +, \cdot)\) is a vector space. That is, if \(f, g \in \smooth{M}\) and \(\lambda \in \mathbb{R}\), then \((f+g)(p) = f(p) + g(p)\) and \((\lambda f)(p) = \lambda\cdot f(p)\), where \(p \in M\).

\begin{definition}{Directional derivative at a point along a curve}{tangent_vector}
    Let \(I = (-\varepsilon, \varepsilon) \subset \mathbb{R}\) be an open interval, for some \(\varepsilon > 0\). Let \(\gamma : I \to M\) be a smooth curve through a point \(p \in M\) such that \(p = \gamma(0)\). The \emph{directional derivative operator at the point \(p\) along the curve \(\gamma\)} is the linear map
    \begin{align*}
        X_{\gamma, p} : \smooth{M} &\linear \mathbb{R}\\
                                            f &\mapsto (f \circ \gamma)'(0).
    \end{align*}
    In differential geometry, the operator \(X_{\gamma, p}\) is usually called the \emph{tangent vector to the curve \(\gamma\) at the point \(p\).}
\end{definition}

A precise intuition is \(X_{\gamma, p}\) is the velocity of the curve \(\gamma\) at the point \(p\). Given a curve \(\gamma : (-\varepsilon, \varepsilon) \to M\), this notion is easily seen with a curve \(\eta : \left(-\frac{\varepsilon}2, \frac{\varepsilon}{2}\right) \to M\) with double the parameter speed \(\eta(\lambda) = \gamma(2 \lambda),\) for \(|\lambda| < \varepsilon\), where one gets \(X_{\eta, p} = 2 X_{\gamma, p}\).

\begin{definition}{Tangent vector space at a point}{tangent_space}
    The \emph{tangent vector space at a point \(p \in M\)} is the set \(T_p M\) of all the directional derivatives operators at the point \(p\) along smooth curves \(\gamma : I \to M\) with \(\gamma(0) = p\) equipped with the addition
    \begin{align*}
        + : T_pM \times T_pM &\to T_pM\\
                    (X, Y)  &\mapsto X+Y
    \end{align*}
    and scalar multiplication
    \begin{align*}
        \cdot : \mathbb{R} \times T_pM &\to T_pM\\
                    (\lambda, X)  &\mapsto \lambda X
    \end{align*}
    defined point-wise, that is, \((X+Y)f = Xf + Yf\) and \((\lambda X)f = \lambda \cdot (Xf)\) for all \(f \in \smooth{M}\).
\end{definition}

We must now verify the claim that \(T_pM\) equipped with the above operations is indeed a vector space. The immediate concern is whether there exists curves such that \(X+Y\) and \(\lambda X\) are their tangent vectors. The vector space axioms follow from letting the tangent vectors act on smooth functions and by using the field properties of \(\mathbb{R}\).

\begin{proposition}{Tangent vector operations are closed in the tangent space}{tangent_vector_operations}
    Let \(I = (-\varepsilon, \varepsilon) \subset \mathbb{R}\) be an interval and let \(\gamma, \eta : I \to M\) be smooth curves with \(\gamma(0) = p\) and \(\eta(0) = p\), where \(p \in M\). Let \(X, Y \in T_pM\) be the directional derivative operator at \(p\) along the curves \(\gamma\) and \(\eta\), respectively. Then, there exists smooth curves \(\phi : I_\phi \to M\) and \(\psi : I_\psi \to M\), where \(I_\phi\) and \(I_\psi\) are intervals on \(\mathbb{R}\), with \(\phi(0) = p\) and \(\psi(0) = p\), such that \(X+Y\) is the tangent vector at \(p\) along \(\phi\) and \(\lambda X\) is the tangent vector at \(p\) along \(\psi\), for some \(\lambda \in \mathbb{R}\). That is, \(X + Y \in T_pM\) and \(\lambda X \in T_pM\).
\end{proposition}
\begin{proof}
    We construct the curve \(\psi\) associated with scalar multiplication. If \(\lambda = 0,\) we set \(I_\psi = \mathbb{R}\) and \(\psi(t) = p\) for all \(t \in I_\psi\), clearly yielding the zero directional derivative operator. For \(\lambda \neq 0,\) we set \(I_\psi = \left(-\frac{\varepsilon}{|\lambda|}, \frac{\varepsilon}{|\lambda|}\right)\) and \(\psi(t) = \gamma(\lambda t)\), for all \(t \in I_\psi\). For any \(f \in \smooth{M}\), the map \(f \circ \gamma : \mathbb{R} \to \mathbb{R}\) is smooth as a real valued function of a single variable. Composing \(f \circ \gamma\) with the smooth map \(t \mapsto \lambda t\) yields \(f \circ \psi\). By the chain rule, we have
    \begin{align*}
        (f \circ \psi)'(0) &= \diff*{\underbrace{f\circ \gamma}_{\mathbb{R} \to \mathbb{R}}\underbrace{(\lambda t)}_{\mathbb{R}\to \mathbb{R}}}{t}[t=0]\\
                           &= (f \circ \gamma)'(0) \cdot (\lambda t)'(0)\\
                           &= \lambda Xf,
    \end{align*}
    as desired.

    Next we construct the curve \(\phi\) associated with vector addition. We consider a chart \((U, x)\) such that \(p \in U\) and take a subset of \(J \subset I\) such that its image by the curves \(\gamma\) and \(\eta\) is contained in \(U\). Let \(\phi : J \to M\) be defined by
    \begin{equation*}
        \phi= x^{-1} \circ \left(x\circ \gamma+ x\circ \eta- x(p)\right),
    \end{equation*}
    where \(x(p)\) is the constant map \(t \mapsto x(p)\) for all \(t \in J\).
    We compute the directional derivative at \(p\) along \(\phi\) of \(f \in \smooth{M}\) with the chain rule
    \begin{align*}
        (f \circ \phi)'(0) &= \left(\underbrace{f \circ x^{-1}}_{\mathbb{R}^n \to \mathbb{R}} \circ \underbrace{\left(x\circ \gamma+ x\circ \eta- x(p)\right)}_{\mathbb{R} \to \mathbb{R}^n}\right)'(0)\\
                           &= \partial_i (f \circ x^{-1})(x(p)) \cdot \left(x^i \circ \gamma + x^i \circ \eta - x^i(p)\right)'(0)\\
                           &= \partial_i (f \circ x^{-1})(x(p)) \cdot (x^i \circ \gamma)'(0) + \partial_j (f \circ x^{-1})(x(p))\cdot (x^j \circ \eta)'(0)\\
                           &= \left(f \circ x^{-1} \circ x \circ \gamma\right)'(0) + \left(f \circ x^{-1} \circ x \circ \eta\right)'(0)\\
                           &= (f \circ \gamma)'(0) + (f \circ \eta)'(0)\\
                           &= Xf + Yf,
    \end{align*}
    as desired. We note the tangent vector to \(\phi\) at \(p\) does not depend of the choice of chart, since the chart map was composed with its inverse.
\end{proof}

\subsection{Algebras and derivations}
\begin{definition}{Algebra}{algebra}
    An \emph{algebra (over a field)} \(\mathcal{A} = (V, +, \cdot, \bullet)\) is a vector space \((V, +, \cdot)\) over a field \(\mathbb{K}\), equipped with a bilinear map \(\bullet : V \times V \linear V\), called a product.
\end{definition}
Defining the product of two smooth functions on a manifold point-wise, we see \(\smooth{M}\) is an algebra over the real numbers, not just a vector space.

\begin{definition}{Derivation on an algebra}{derivation}
    A \emph{derivation} \(D\) on an algebra \(\mathcal{A}\) is a linear map \(D : \mathcal{A} \to \mathcal{A}\) that satisfies the \emph{Leibniz rule}
    \begin{equation*}
        D(f\bullet g) = (Df) \bullet g + f \bullet (Dg),
    \end{equation*}
    for all \(f,g \in \mathcal{A}.\)
\end{definition}
Any \(X \in T_pM\) is a \emph{derivation at a point} on the algebra of smooth functions on a manifold, since the Leibniz rule is satisfied
\begin{equation*}
    X(fg) = (Xf)g(p) + f(p)(Xg).
\end{equation*}
As another example, we may set \(\mathcal{A}\) as the algebra of endomorphisms on a vector space \(V\) with product \([\phi, \psi] = \phi \circ \psi - \psi \circ \phi\). It can be shown that this product satisfies the so called \emph{Jacobi identity},
\begin{equation*}
    [\phi, [\psi, \rho]] + [\psi, [\rho, \phi]] + [\rho, [\phi, \psi]] = 0.
\end{equation*}
In particular, this algebra is called a \emph{Lie algebra}. Given \(H \in \mathcal{A}\), we define a derivation
\begin{align*}
    D_H : \mathcal{A} &\linear \mathcal{A}\\
                \phi &\mapsto [H, \phi].
\end{align*}
Using the Jacobi identity, it's easy to verify this is indeed a derivation. We note that, in classical mechanics, the Poisson bracket is a derivation on the algebra of functions on phase space.

\subsection{Chart induced basis of the tangent space}

We now establish an important theorem that relates the dimension of the tangent space \(T_pM\) with the dimension of the manifold. In this constructive proof, a basis for the tangent space will be determined from a chart. First, we prove a lemma.

\begin{lemma}{Component functions are smooth}{component_smooth}
    Let \manifold{M} be an \(n\)-dimensional smooth manifold. Let \((U, x) \in \mathscr{A}_M\) be a chart. Then, the component functions \(x^i : U \to \mathbb{R}\) are smooth.
\end{lemma}
\begin{proof}
    We recall that a map \(f : M \to \mathbb{R}\) is smooth at a point \(p \in U\) if and only if \(f \circ x^{-1} \in \mathcal{C}^\infty (\mathbb{R}^n, \mathbb{R})\).
    \begin{equation*}
        \begin{tikzcd}[column sep = normal, row sep = large]
            U \arrow{r}{f} \arrow{d}{x} & \mathbb{R}\\
            x(U) \arrow[swap]{ur}{f \circ x^{-1}} &
        \end{tikzcd}
    \end{equation*}
    In particular, \(x^i : U \to \mathbb{R}\) is smooth if and only if \(x^i \circ x^{-1} \in \mathcal{C}^\infty(\mathbb{R}^n, \mathbb{R}).\)
    \begin{equation*}
        \begin{tikzcd}[column sep = normal, row sep = large]
            U \arrow{r}{x^i} \arrow{d}{x} & \mathbb{R}\\
            x(U) \arrow[swap]{ur}{x^i \circ x^{-1}} &
        \end{tikzcd}
    \end{equation*}
    By definition, we have \(x^i \circ x^{-1} = \mathrm{proj}_i\), which is smooth, since it is linear.
\end{proof}

We may now construct a chart-induced basis for the tangent space.

\begin{theorem}{Dimension of the tangent space}{dimension_tangent_space}
    Let \(M\) be an \(n\)-dimensional smooth manifold. Then, for all \(p \in M\), the \(\dim T_pM = n\).
\end{theorem}
\begin{proof}
    Let \((U, x)\) be a chart such that \(p \in U\).% Without loss of generality we may assume \(x(p) = 0\).

    We consider the family of \emph{coordinate curves} \ffamily{\gamma_{i} : I \to U}{i = 1}{n}, where \(I \subset \mathbb{R}\) is an interval, with local expression satisfying
    \begin{equation*}
        (x^j \circ \gamma_i)(\lambda) = x(p) + \delta_i^j \lambda,
    \end{equation*}
    for all \(\lambda \in I\), \(i, j \in \set{1, \dots, n}\) and such that \(\gamma_i(0) = p\).
    Intuitively, each curve is the image of a straight line in \(x(U)\) under the map \(x^{-1}\), where these straight lines are parallel to each axis of \(\mathbb{R}^n\) and meet at \(x(p)\).

    Let \(e_i \in T_pM\) be the tangent vector at \(p\) along \(\gamma_i\) and let \(f \in \smooth{M}.\) By the chain rule,
    \begin{align*}
        e_i f &= (f\circ \gamma_i)'(0)\\
              &= (f \circ x^{-1} \circ x \circ \gamma_i)'(0)\\
              &= \partial_j (f \circ x^{-1})(x(p)) \cdot (x^j \circ \gamma_i)'(0)\\
              &= \partial_j (f \circ x^{-1})(x(p)) \cdot (x(p) + \delta_i^j \lambda)'(0)\\
              &= \delta_i^j \partial_j (f \circ x^{-1})(x(p))\\
              &= \partial_i (f \circ x^{-1})(x(p)).
    \end{align*}

    We write
    \begin{equation*}
        e_i f = \bvec[f]{x^i}{p}
    \end{equation*}
    to denote \(\partial_i (f\circ x^{-1})(x(p)).\) That is, \(e_i = \bvec{x^i}{p}\) and
    \begin{equation*}
        \mathcal{B} = \bset{x}{n}{p}%\set*{\diffp*{}{x^1}, \dots, \diffp*{}{x^n}}
    \end{equation*}
    is the set of tangent vectors to the chart induced curves \(\gamma_i\). We claim \(\mathcal{B}\) is a generating set for \(T_p M\). Indeed, let \(\eta : I \to M\) be a smooth curve with \(\eta(0) = p\) with tangent vector \(X\in T_pM\) at p. For a smooth map \(f \in \mathcal{C} ^\infty (M)\), we have
    \begin{align*}
        Xf &= (f \circ \eta)'(0)\\
           &= (f \circ x^{-1} \circ x \circ \eta)'(0)\\
           &= \partial_i (f \circ x^{-1})(x(p)) \cdot (x^i \circ \eta)'(0)\\
           &= (x^i \circ \eta)'(0) \bvec[f]{x^i}{p}.
    \end{align*}
    This shows \(X\) is a linear combination of the elements of \(\mathcal{B}\), with components \((x^i \circ \eta)'(0).\) Thus, \(\mathcal{B}\) spans \(T_pM\).

    Let \ffamily{\lambda^i}{i=1}{n} be one family of components of a linear combination of the elements of \(\mathcal{B}\) that results in the zero tangent vector. Due to \cref{lem:component_smooth}, we may compute the directional derivative at p of component functions \(x^i : U \to \mathbb{R}\). We have
    \begin{align*}
        \lambda^i \bvec[x^j]{x^i}{p} &= \lambda^i \partial_i \left(x^j \circ x^{-1}\right)(x(p))\\
                                     &= \lambda^i \delta_{i}^j\\
                                     &= \lambda^j.
    \end{align*}
    Since \(\lambda^i \diffp*{}{x^i} = 0,\) we have shown \(\lambda^j = 0,\) for all \(j \in \set{1, \dots, n}\). That is, \(\mathcal{B}\) is linearly independent and, thus, a basis of \(T_pM\).
\end{proof}

Suppose \((U, x), (V, y) \in \mathscr{A}_M\) are charts of \(M\) where \(U\) and \(V\) are neighborhoods of \(p\). Without loss of generality, we assume \(U = U \cap V\) and restrict \(y : U \to \mathbb{R}^n\). Then each chart induces a basis for \(T_pM\), namely \(\mathcal{B}_x = \bset{x}{n}{p}\) and \(\mathcal{B}_y = \bset{y}{n}{p}\).  We have already shown that for \(X \in T_pM\) with an associated curve \(\eta : I \to M,\) then
\begin{equation*}
    X = (y^i \circ \eta)'(0) \bvec{y^i}{p}.
\end{equation*}
In particular, this is valid for the basis vectors in \(\mathcal{B}_x.\) As before, for a given basis vector, we look at its associated coordinate curve \(\gamma_i : I \to M\) with \((x^j \circ \gamma_i)(\lambda) = x^j(p) + \delta_i^j \lambda\) for all \(\lambda \in I\). Then, we have
\begin{align*}
    \bvec{x^i}{p} &= (y^j \circ \gamma_i)'(0) \bvec{y^j}{p}\\
                  &= (y^j \circ x^{-1} \circ x \circ \gamma_i)'(0) \bvec{y^j}{p}\\
                  &= \partial_k (y^j \circ x^{-1})(x(p)) \cdot (x^k \circ \gamma_i)'(0) \bvec{y^j}{p}\\
                  &= \delta_i^k \bvec[y^j]{x^k}{p} \bvec{y^j}{p}\\
                  &= \bvec[y^j]{x^i}{p} \bvec{y^j}{p},
\end{align*}
that is, the linear map with components \(\bvec[y^j]{x^i}{p}\) is the map that governs the change of basis from \(\mathcal{B}_x\) to \(\mathcal{B}_y\). This map is usually called the \emph{Jacobian}.

\subsection{Cotangent spaces and the gradient}

We now define the dual vector space to the tangent space.

\begin{definition}{Cotangent space at a point}{cotangent_space}
    Let \(M\) be a smooth manifold. The \emph{cotangent space \(T_p ^{\ast}M\) at a point \(p \in M\)} is the dual vector space to the tangent space \(T_pM,\) that is \(T_p ^{\ast}M = (T_pM)^{\ast}.\)
\end{definition}
\begin{remark}
    We recall that for a finite-dimensional vector space, its dimension is equal to the dimension of its dual vector space. Therefore, for a finite-dimensional manifold, the dimension of the cotangent space is equal to the tangent space.
\end{remark}

We may now define tensors over a tangent plane on a point of the manifold. As before, the set
\begin{equation*}
    T_s^r(T_pM) = \underbrace{T_pM \otimes \dots \otimes T_pM}_{r\text{ times}} \otimes \underbrace{T_p ^{\ast}M \otimes \dots \otimes T_p ^{\ast} M}_{s \text{ times}}
\end{equation*}
of multilinear maps
\begin{equation*}
    t : \underbrace{T_p^{\ast} \times \dots \times T_p^{\ast}}_{r\text{ times}} \times \underbrace{T_pM \times \dots \times T_pM}_{s \text{ times}} \linear \mathbb{R}
\end{equation*}
is the set (vector space) of all \((r,s)\)-tensors at the point \(p \in M\).

\begin{definition}{Gradient operator at a point}{gradient}
    The \emph{gradient \(d_p\) operator at a point \(p \in M,\)} is a linear map
    \begin{align*}
        d_p : \smooth{M} &\linear T_p ^{\ast}M\\
                                  f &\mapsto d_p f
    \end{align*}
    defined by
    \begin{equation*}
        (d_p f)(X) = Xf,
    \end{equation*}
    for all \(X \in T_pM\).
\end{definition}

Given a smooth map \(f : M \to \mathbb{R}\), we may look at its level set \(\set{p \in M : f(p) = c}\), for some constant \(c \in \mathbb{R}\). Then, if \(X \in T_pM\) is tangent to this level set, it follows that \(d_p f(X) = 0\).

With the notion of the gradient operator, we construct a chart-induced basis for the cotangent space.

\begin{theorem}{Chart-induced covector basis}{dual_basis}
    Let \(M\) be an \(n\)-dimensional smooth manifold. Let \((U, x) \in \mathscr{A}_M\) be a chart with \(p \in U\). Then
    \begin{equation*}
        \mathcal{B} = \set*{d_p x^1, \dots, d_p x^n}
    \end{equation*}
    is the \emph{chart-induced covector basis} for \(T_p ^{\ast} M\).
\end{theorem}
\begin{proof}
    We recall \cref{lem:component_smooth} to show that, indeed, \(d_p x^i \in T_p ^{\ast} M\).

    Consider \(d_p x^i \left(\bvec{x^j}{p}\right)\). We have

    \begin{align*}
        d_p x^i \left(\bvec{x^j}{p}\right) &= \bvec[x^i]{x^j}{p}\\
                                           &= \partial_j \underbrace{\left(x^i \circ x^{-1}\right)}_{\mathrm{proj}_i \colon \mathbb{R}^n \to \mathbb{R}}(x(p))\\
                                           &= \delta_j^i.
    \end{align*}

    From linearity, we have
    \begin{equation*}
        d_px^i\left(\lambda^j \bvec{x^j}{p}\right) = \lambda^i,
    \end{equation*}
    that is, \(\mathcal{B}\) is the dual basis of \bset{x}{n}{p} for the dual space \(T_p ^{\ast}M\).
\end{proof}

\subsection{Pushforward and pullback}

Given a smooth map between smooth manifolds, we may define an object that takes tangent vectors on a manifold to tangent vectors on another manifold.

\begin{definition}{Pushforward at a point}{pf_tangent_space}
    Let \(\phi : M \to N\) be a smooth map between smooth manifolds. The \emph{pushforward \(\pf[p]\phi\) of the map \(\phi\) at the point \(p \in M\)} is the linear map
    \begin{align*}
        \pf[p]\phi : T_pM &\linear T_{\phi(p)}N\\
                        X &\mapsto \pf[p]\phi(X),
    \end{align*}
    defined by
    \begin{equation*}
        \pf[p]{\phi}(X)f = X(f \circ \phi)
    \end{equation*}
    for all \(f \in \smooth{N}.\)
\end{definition}

\begin{remark}
    The pushforward \(\pf[p]{\phi}\) is often called the \emph{differential of \(\phi\) at \(p\)}.
\end{remark}

We note the tangent vector \(X_{\gamma,p}\) at \(p\) of the smooth curve \(\gamma\) is \emph{pushed forward} to the tangent vector \(\pf[p]{\phi}(X)\) of the smooth curve \(\phi \circ \gamma.\) Indeed, let \(f \in \smooth{N}\), then
\begin{align*}
    \pf[p]\phi(X)f &= X(f \circ \phi)\\
                   &= (f \circ \phi \circ \gamma)'(0)\\
                   &= X_{\phi \circ \gamma, \phi(p)} f,
\end{align*}
which implies \(X_{\phi \circ \gamma, \phi(p)} = \pf[p]\phi(X),\) as desired.

Similarly, we define an object that relates elements in cotangent spaces from one manifold to another.

\begin{definition}{Pullback at a point}{pb_cotangent_space}
    Let \(\phi : M \to N\) be a smooth map between smooth manifolds. The \emph{pullback \(\pb[p]\phi\) of the map \(\phi\) at the point \(p\)} is the linear map
    \begin{align*}
        \pb[p]\phi : T_{\phi(p)}^{\ast}N &\linear T_{p}^{\ast}M\\
        \omega &\mapsto \pb[p]\phi(\omega),
    \end{align*}
    defined by
    \begin{equation*}
        \pb[p]\phi(\omega)(Y) = \omega(\pf[p]\phi(Y))
    \end{equation*}
    for all \(Y \in T_pM\).
\end{definition}
\begin{remark}
    Unless the map \(\phi\) is a diffeomorphism, vectors are pushed forward and covectors are pulled back.
\end{remark}

\begin{theorem}{Pushforward and pullback of a composition}{pf_pb_composition}
    Let \(f : M \to N\) and \(g : N \to P\) be smooth maps between smooth manifolds and let \(p \in M\). Then,
    \begin{equation*}
        \pf[p]{h} = \pf[f(p)]{g} \circ \pf[p]{f} \text{ and }\pb[p]{h} = \pb[p]{f} \circ \pb[f(p)]{g},
    \end{equation*}
    where \(h = g \circ f\).
\end{theorem}
\begin{remark}
    The identity for the pushforward is the generalization of the chain rule of elementary calculus.
\end{remark}
\begin{proof}
    We first prove the chain rule. Let \(X \in T_pM\), then for any \(\varphi \in \smooth{P}\),
    \begin{align*}
        \left(\pf[p]{h}{X}\right)\varphi &= X(\varphi \circ g \circ f)\\
                                         &= \left(\pf[p]{f}{X}\right)(\varphi \circ g)\\
                                         &= \pf[f(p)]{g}{\left(\pf[p]{f}{X}\right)}\varphi\\
                                         &= \left(\pf[f(p)]{g}{\circ\pf[p]{f}{(X)}}\right)\varphi
    \end{align*}
    hence \(\pf[p]{h} = \pf[f(p)]{g} \circ \pf[p]{f}\).

    To prove the other identity, we use the above result. Indeed, let \(\omega \in T_{h(p)}P\), then for any \(X \in T_pM\),
    \begin{align*}
        \left(\pb[p]{f}{\circ \pb[f(p)]{g}{\omega}}\right)(X) &= \left(\pb[f(p)]{g}{\omega}\right)\left(\pf[p]{f}{X}\right)\\
                                                              &= \omega \left(\pf[f(p)]{g}{\circ \pf[p]{f}{X}}\right)\\
                                                              &= \omega \left(\pf[p]{h}{X}\right)\\
                                                              &= \pb[p]{h}{\omega}(X),
    \end{align*}
    hence \(\pb[p]{h} = \pb[p]{f} \circ \pb[f(p)]{g}.\)
\end{proof}

\subsection{Ambient space}
So far, the tangent plane and its structure has been done intrinsically.

Decide the question under which circumstance some smooth manifold \(M\) can sit in some \(\mathbb{R}^n\)

Although the following definitions are done between smooth manifolds \(M\) and \(N\) we will turn our attention to the special case \(N=\mathbb{R}^m\), equipped with the standard topology.

\begin{definition}{Immersion}{immersion}
    An \emph{immersion \(\phi\) of \(M\) into \(N\)} is a smooth map \(\phi : M \to N\) such that, for all \(p \in M\), the pushforward \(\pf[p]\phi\) is injective.
\end{definition}
\begin{example}
    \begin{enumerate}[label=(\alph*)]
        \item We consider a map \(\phi : S^1 \to \mathbb{R}^2\) such that \(\phi(M)\) is a closed simple smooth curve in \(\mathbb{R}^2\). Then \(\phi(M)\) is an immersion.
        \item We consider the map \(\tilde\phi : S^1 \to \mathbb{R}^2\) such that \(\tilde\phi(M)\) is a lemniscate. We note \(\tilde\phi\) is not injection, but \(\pf{\tilde\phi}\) is injective, and therefore \(\pf{\tilde\phi}\) is an immersion.
    \end{enumerate}
\end{example}

\begin{definition}{Embedding}{embedding}
    An \emph{embedding} is an immersion with image homeomorphic to its domain, where the image has the subspace topology inherited from its codomain.
\end{definition}
\begin{example}
    The map \(\phi\) defined in the previous example is an embedding, but the map \(\tilde\phi\) is not.
\end{example}

We may now quote two important theorems that answer the question considered in the beginning of this section.

\begin{theorem}{Whitney's immersion and embedding theorem}{immersion_whitney}
    Any \(n\)-dimensional smooth manifold can be
    \begin{enumerate}[label=(\alph*)]
        \item immersed in \(\mathbb{R}^{2n-1}\);
        \item embedded in \(\mathbb{R}^{2n}\).
    \end{enumerate}
\end{theorem}
\begin{example}
    The two-dimensional Klein bottle can be immersed into \(\mathbb{R}^3\) but only embedded in \(\mathbb{R}^4\). This is the "worst case" of the Whitney theorem. The theorem does not state if there is an immersion or embedding in lower dimensions other than the ones stated. One could check, for instance, the 2-sphere may be embedded into \(\mathbb{R}^3\).
\end{example}

With this theorem one could show the intrinsically definitions are equivalent to ones defined extrinsically. For instance, one could embed a manifold into a higher dimensional Euclidean space, named the \emph{ambient space}, and define the tangent space at a point as a hyperplane in the ambient space.

Even though the extrinsic approach is equivalent to the intrinsic approach, in the purely abstract sense, it may not desirable to pursue in some cases. As an example, in General Relativity there would have to be a physical justification for an ambient space where spacetime is embedded.


\section{Tensors over a ring}

\begin{definition}{Ring}{ring}
    A ring \((R, +, \cdot)\) is a set \(R\) equipped with two maps \(+, \cdot : R \times R \to R\) called addition and multiplication that satisfy
    \begin{enumerate}[label=(\alph*)]
        \item Associativity of addition and multiplication: For all \(a,b,c \in R\), \(a + (b + c) = (a + b) + c\) and \(a \cdot (b\cdot c) = (a\cdot b) \cdot c\);
        \item Commutativity of addition: For all \(a,b \in R\), \(a + b = b + a\);
        \item Additive and multiplicative identity: There exists two distinct elements \(0\) and \(1\) in \(R\) such that for all \(a \in R\), \(a + 0 = a\) and \(a \cdot 1 = a\);
        \item Additive inverse: For every \(a \in R\) there exists an element in \(-a \in R\), called the additive inverse of \(a\), such that \(a + (-a) = 0\);
        \item Distributivity of multiplication over addition: For all \(a, b, c \in R\), \(a \cdot (b + c) = (a \cdot b) + (a\cdot c)\).
    \end{enumerate}
    Usually the multiplication \(a \cdot b\) is denoted by \(ab\).
\end{definition}

\begin{theorem}{Existence of Hamel basis}{existence_of_basis}
    Every module over a division ring has a Hamel basis.
\end{theorem}
\begin{corollary}
    Every vector field has a Hamel basis.
\end{corollary}

\section{Differential forms and exterior algebra}

Recall \(S_k\) is the group of permutations on the finite set of natural numbers \(\set{1, \dots, k}\). If \(\pi \in S_k\) is a permutation, then its parity \(\sgn(\pi)\) is \(1\) if \(\pi\) has an even amount of inversions, and \(-1\) otherwise.

\begin{definition}{Differential form}{differential_form}
    Let \(M\) be a smooth manifold. A \emph{(differential) \(k\)-form} is a \((0, k)\)-tensor field \(\omega\) on \(M\) that is totally antisymmetric, that is
    \begin{equation*}
        \omega(X_1, \dots, X_k) = \sgn(\pi)\cdot\omega(X_{\pi(1)}, \dots, X_{\pi(k}),
    \end{equation*}
    where \(X_i \in \sections{TM}\), and \(\pi \in S_k.\)
\end{definition}
\begin{example}
    \begin{enumerate}[label=(\alph*)]
        \item Smooth functions on \(M\) and covector fields are trivially differential forms. We may now refer to covector fields as 1-forms.
        \item If \(M\) is orientable, then there exists a nowhere vanishing volume form.
        \item Electromagnetic field strength \(F\) is a 2-form.
        \item In classical mechanics, if \(Q\) is the configuration space, then its cotangent bundle \(T^\ast Q\) is the phase space. On \(T^\ast Q\) there exists a canonically defined 2-form, called the \emph{symplectic form}.
    \end{enumerate}
\end{example}

We denote the set of all \(k\)-forms on \(M\) by \forms[k]{M}, which can be equipped to be a \smooth{M}-module, as it is a submodule of the set of all \((0, k)\)-tensor fields on \(M\). It is easy to see the tensor product of differential forms does not yield a form. Indeed, let \(\omega, \eta \in \forms[1]{M}\) be 1-forms, then
\begin{equation}
    (\omega\otimes\eta)(X, Y) = \omega(X) \cdot \eta(Y),
\end{equation}
which generically is not the same as \(-(\omega \otimes \eta)(Y, X)\).

We wish to construct a \smooth{M}-algebra of forms, where the algebra over a ring is defined by changing \enquote{field} to \enquote{ring} and \enquote{vector field} to \enquote{module} in \cref{def:algebra}. A product between forms that yields forms must be defined, but we have just seen the tensor product is not enough.

\begin{definition}{Wedge product}{wedge}
    Let \(M\) be a smooth manifold. The \emph{wedge product} between two differential forms is the map
    \begin{align*}
        \wedge : \forms[k]{M}\times \forms[\ell]{M} &\to \forms[k+\ell]{M}\\
                                (\omega, \eta) &\mapsto \omega \wedge \eta,
    \end{align*}
    defined by
    \begin{equation*}
        (\omega \wedge \eta)(X_1, \dots X_{k+\ell}) = \frac{1}{k! \ell!} \sum_{\pi \in S_{k+\ell}} \sgn(\pi)\cdot(\omega \otimes \eta)\left(X_{\pi(1)}, \dots, X_{\pi(k+\ell)}\right),
    \end{equation*}
    where \(X_i \in \sections{TM}.\)
\end{definition}
\begin{example}
    In the case of 1-forms, we have \(\omega \wedge \eta = \omega \otimes \eta - \eta \otimes \omega,\) and it is easy to see \(\omega \wedge \eta\) indeed is a 2-form and that \(\omega \wedge \eta = - \eta \wedge \omega\). A similar result is generalized for any pair of differential forms, which is shown in \cref{prop:wedge_properties}.
\end{example}

We now check the assertion made in the previous definition. That is, we must check if the wedge product yields a differential form.
\begin{proposition}{Wedge product yields a differential form}{wedge_forms}
    Let \(\omega \in \forms[k]{M}\) and \(\eta \in \forms[\ell]{M}\). Then \(\omega \wedge \eta \in \forms[k+\ell]{M}\).
\end{proposition}
\begin{proof}
    Consider the differential forms \(\omega \in \forms[k]{M}\) and \(\eta \in \forms[\ell]{M}\), we have. It is easy to see \(\omega \wedge \eta\) is a multilinear map, since \(\omega\) and \(\eta\) are tensors and the wedge product is defined with the tensor product. That is, \(\omega \wedge \eta\) is a \((0, k + \ell)\)-tensor field, and we must now show it is a differential form.

    Let \(\tau \in S_{k+\ell}\) be a permutation and let \(X_i \in \sections{TM}\) be vector fields for \(i \in \set{1, \dots, k+\ell}\). From the definition of the wedge product, we have
    \begin{equation*}
        (\omega \wedge \eta)\left(X_{\tau(1)}, \dots, X_{\tau(k + \ell)}\right) = \frac{1}{k!\ell!} \sum_{\pi \in S_{k+\ell}} \sgn(\pi)\cdot(\omega \otimes \eta)\left(X_{\pi \circ \tau(1)}, \dots, X_{\pi \circ \tau(k+\ell)}\right).
    \end{equation*}
    For any \(\pi \in S_{k+\ell},\) \(\pi \circ \tau \in S_{k+\ell},\) since the permutation group is closed under composition. It is clear that \(\sgn(\pi \circ \tau) = \sgn(\pi) \cdot \sgn(\tau)\), and we have
    \begin{align*}
        \sgn(\tau)\cdot(\omega \wedge \eta)\left(X_{\tau(1)}, \dots, X_{\tau(k+\ell)}\right) &= \frac{1}{k!\ell!} \sum_{\pi \in S_{k+\ell}} \sgn(\pi\circ\tau)\cdot (\omega \otimes\eta)\left(X_{\pi \circ \tau(1)}, \dots, X_{\pi \circ \tau(k+\ell)}\right)\\
                                                                                             &= \frac{1}{k!\ell!}\sum_{\sigma \in S_{k+\ell}} \sgn(\sigma)\cdot(\omega\otimes \eta)\left(X_{\sigma(1)}, \dots, X_{\sigma(k+\ell)}\right)\\
                                                                                             &= (\omega \wedge \eta)\left(X_1, \dots, X_{k+\ell}\right).
    \end{align*}
    This shows \(\omega \wedge \eta\) is an alternating tensor, and this proves our claim.
\end{proof}

% cite munkres analysis on manifolds
\begin{proposition}{Wedge product properties}{wedge_properties}
    The wedge product is associative and graded anticommutative, meaning
    \begin{equation*}
        \omega \wedge \eta = (-1)^{k\ell} \eta \wedge \omega,
    \end{equation*}
    and
    \begin{equation*}
        (\omega \wedge \eta) \wedge \alpha = \omega \wedge (\eta \wedge \alpha),
    \end{equation*}
    for all \(\omega \in \forms[k]{M}\), \(\eta \in \forms[\ell]{M}\), and \(\alpha \in \forms[m]{M}\).
\end{proposition}
\begin{proof}
    As the proof for associativity is cumbersome, we refer to \cite{munkres_analysis}. To show graded anticommutativity, we consider the permutation \(\tau \in S_{k+\ell}\) such that
    \begin{equation*}
        (\tau(1), \dots, \tau(k+\ell)) = (k+1, \dots, k+\ell, 1, \dots, k).
    \end{equation*}
    We verify the parity of this permutation by counting the number of inversions of \(\tau.\) It is easy to see that the only inversions are between elements \(k+i\) and \(j\), where \(1 \leq i \leq \ell\) and \(1 \leq j \leq k\). Moreover, for a given element \(k + i,\) there are exactly \(k\) inversions. Then, there are \(k\ell\) inversions, hence  \(\sgn(\tau) = (-1)^{k\ell}.\)

    Then, given vector fields \(X_i \in \sections{TM}\), for \(i \in \set{1, \dots, k+\ell}\), it follows that
    \begin{equation*}
        (\omega \otimes \eta)\left(X_1, \dots, X_{k+\ell}\right) = (\eta \otimes \omega)\left(X_{\tau(1)}, \dots, X_{\tau(k+\ell)}\right).
    \end{equation*}
    From the definition of the wedge product we have
    \begin{align*}
        (\omega \wedge \eta)\left(X_{1}, \dots, X_{k+\ell}\right) &= \frac{1}{k!\ell!} \sum_{\pi \in S_{k+\ell}} \sgn(\pi)\cdot (\omega \otimes \eta)\left(X_{\pi(1)}, \dots X_{\pi(k+\ell)}\right)\\
                                                                  &= \frac{1}{k!\ell!}\sum_{\pi \in S_{k+\ell}} \frac{\sgn(\tau \circ \pi)}{\sgn(\tau)} \cdot (\eta\otimes\omega)\left(X_{\tau\circ\pi(1)}, \dots, X_{\tau\circ\pi(k+\ell)}\right)\\
                                                                  &= \frac{(-1)^{k\ell}}{k!\ell!}\sum_{\sigma \in S_{k+\ell}} \sgn(\sigma) (\eta\otimes\omega)\left(X_{\sigma(1)}, \dots, X_{\sigma(k+\ell)}\right)\\
                                                                  &= (-1)^{k\ell} (\eta \wedge \omega) \left(X_1, \dots, X_{k+\ell}\right),
    \end{align*}
    that is, \(\omega \wedge \eta = (-1)^{k\ell} \eta \wedge \omega,\) as desired.
\end{proof}

Recall that a product in an algebra must not only be closed in the algebra, but it must be a bilinear map.
\begin{proposition}{Wedge product is \smooth{M}-bilinear}{wedge_bilinear}
    Let \(\alpha,\beta \in \forms[k]{M}\), \(\omega\in \forms[\ell]{M}\) and \(f \in \smooth{M}\). Then
    \begin{equation*}
        (\alpha + f \beta) \wedge \omega = \alpha \wedge \omega + f (\beta\wedge\omega),
    \end{equation*}
    that is, the wedge product is bilinear.
\end{proposition}
\begin{proof}
    It is clear this property follows from the bilinearity of the tensor product, that is, we have \((\alpha + f \beta) \otimes \omega = \alpha \otimes \omega + f \beta \otimes \omega\). Let \(X_i \in \sections{TM}\) be vector fields, with \(i \in \set{1, \dots, k+\ell},\) then
    \begin{align*}
        \left((\alpha + f \beta) \wedge \omega\right)\left(X_1, \dots, X_{k+\ell}\right) &= \frac{1}{k!\ell!} \sum_{\pi \in S_{k+\ell}} \sgn(\pi)\cdot \left((\alpha + f \beta) \otimes \omega\right)\left(X_{\pi(1)}, \dots, X_{\pi(k+\ell)}\right)\\
                                                                                         &= \frac{1}{k!\ell!} \sum_{\pi \in S_{k+\ell}} \sgn(\pi)\cdot (\alpha \otimes \omega + f \beta \otimes \omega)\left(X_\pi(1), \dots, X_{\pi(k+\ell)}\right)\\
                                                                                         &= (\alpha \wedge \omega + f \beta \wedge \omega)\left(X_1, \dots, X_{k+\ell}\right),
    \end{align*}
    as desired.
\end{proof}

We generalize the pullback of covectors to differential forms and show it is compatible with the wedge product.

\begin{definition}{Pullback of a differential form}{pullback_form}
    Let \(h : M \to N\) be a smooth map between smooth manifolds \(M\) and \(N\). The \emph{pullback of a \(k\)-form \(\omega \in \forms[k]{N}\) on \(N\) is the \(k\)-form \(\pb{h}\omega \in \forms[k]{M}\) on \(M\)} defined by
    \begin{equation*}
        (\pb{h}\omega)(p) (X_1(p), \dots, X_k(p)) = \omega(h(p))\left(\pf[p]{h}(X_1(p)), \dots, \pf[p]{h}(X_k(p))\right),
    \end{equation*}
    for all points \(p \in M\) and vector fields \(X_i \in \sections{TM}\).
\end{definition}
It is clear that \(\pb{h}\omega \in \forms[k]{M},\) since it inherits both multilinearity and antisymmetry from the differential form \(\omega \in \forms[k]{N}\).

\begin{proposition}{Pullback is compatible with the wedge product}{pullback_wedge}
    Let \(h : M \to N\) be a smooth map between smooth manifolds \(M\) and \(N\). Then
    \begin{equation*}
        \pb{h}(\omega \wedge \eta) = (\pb{h}\omega) \wedge (\pb{h}\eta),
    \end{equation*}
    for all \(\omega \in \forms[k]{N}\) and \(\eta \in \forms[\ell]{N}.\)
\end{proposition}
\begin{proof}
    We consider vector fields \(X_i \in \sections{TM},\) for \(i \in \set{1, \dots, k+\ell}\) and differential forms \(\omega \in \forms[k]{N}\) and \(\eta\in\forms[\ell]{M}\). Then, for each point \(p \in M\) we assign \(Y_i = X_i(p) \in T_pM\) for aesthetic purposes. Then,
    \begin{align*}
        (\pb{h}\omega \otimes \pb{h}\eta)(p)(Y_1, \dots, Y_{k+\ell}) &= \pb{h}\omega(p)(Y_1, \dots, Y_{k})\cdot \pb{h}\eta(p)(Y_{k+1}, \dots, Y_{k+\ell})\\
                                                                     &= \omega(h(p))(\pf[p]{h}Y_1, \dots, \pf[p]{h}Y_k)\cdot \eta(h(p))(\pf[p]{h}Y_{k+1}, \dots, \pf[p]{H}Y_{k+\ell})\\
                                                                     &= (\omega \otimes \eta)(h(p))(\pf[p]{h}Y_1, \dots, \pf[p]{h}Y_{k+\ell}),
    \end{align*}
    and as a result,
    \begin{align*}
        (\pb{h}\omega\wedge\pb{h}\eta)(p)(X_1(p), \dots, X_{k+\ell}(p)) &= \frac{1}{k!\ell!} \sum_{\pi \in S_{k+\ell}} \sgn(\pi) (\pb{h}\omega \otimes \pb{h}\eta)(p) (Y_{\pi(1)}, \dots, Y_{\pi(k+\ell)})\\
                                                                  &= \frac{1}{k!\ell!} \sum_{\pi \in S_{k+\ell}} \sgn(\pi) (\omega \otimes \eta)(h(p)) (\pf[p]{h}Y_{\pi(1)}, \dots, \pf[p]{h}Y_{\pi(k+\ell)})\\
                                                                  &= (\omega \wedge \eta)(h(p))(\pf[p]{h}Y_1, \dots, \pf[p]{h}Y_{k+\ell})\\
                                                                  &= \pb{h}(\omega\wedge \eta )(p)(X_1(p), \dots, X_{k+\ell}(p)),
    \end{align*}
    and we conclude \(\pb{h}\omega \wedge \pb{h}\eta = \pb{h}(\omega\wedge \eta)\) as desired.
\end{proof}

We may now define the algebra with the \smooth{M}-module of the direct sum of all differential forms, that is closed under the linear continuation of the wedge product.
\begin{definition}{Exterior algebra}{exterior_algebra}
    Let \(M\) be an \(n\)-dimensional smooth manifold. The \emph{exterior algebra \forms{M}} is the \smooth{M}-module
    \begin{equation*}
        \forms{M} = \bigoplus_{k = 0}^n \forms[k]{M}
    \end{equation*}
    equipped with the \emph{exterior product \(\wedge\)}, the bilinear map

    \begin{align*}
        \wedge : \forms{M} \times \forms{M} &\to \forms{M}\\
                             (\omega, \eta) &\mapsto \omega \wedge \eta,
    \end{align*}
    defined by linear continuation of the wedge product.
\end{definition}
\begin{remark}
    Any differential form of rank greater than \(n\) is the zero tensor. To see this, consider a local chart, where the dimension of the tangent space is equal to \(n,\) that is, there are at most \(n\) linearly independent vectors, and thus any differential form of rank greater than \(n\) would only yield zero for any family of tangent vectors.
\end{remark}
\begin{example}
    We illustrate the meaning of the linear continuation of the wedge product. Recall an element \(u\) of a direct sum module \(V \oplus W\) has a unique decomposition \(u = v + w,\) where \(v \in V\) and \(w \in W\). A differential form \(\omega \in \forms{M}\) may be expressed as
    \begin{equation*}
        \omega = \sum_{k = 0}^{n} \omega^k,
    \end{equation*}
    where \(\omega^k \in \forms[k]{M}.\) Let \(\eta \in \forms[\ell]{M} \subset \forms{M}\). The exterior product \(\omega \wedge \eta\) is then
    \begin{equation*}
        \omega \wedge \eta = \sum_{k = 0}^{n} \omega^k \wedge \eta,
    \end{equation*}
    where \(\omega^k \wedge \eta\) is defined by the wedge form between a \(k\)-form and an \(\ell\)-form, as before.

    We warn, however, that the graded anticommutativity shown in \cref{prop:wedge_properties} does not hold for a general element of the exterior algebra, and we may only use it for elements whose ranks are known, that is
    \begin{equation*}
        \eta \wedge \omega = \sum_{k = 0}^{n} \eta \wedge \omega^k = \sum_{k = 0}^n (-1)^{k\ell} \omega^k \wedge \eta,
    \end{equation*}
    which is not, in general, related to \(\omega \wedge \eta\).
\end{example}

Recall we have already defined the gradient operator
\begin{align*}
    d : \forms[0]{M} &\to \forms[1]{M}\\
                   f &\mapsto df
\end{align*}
where \((df)(p) = d_p f \in T_p ^{\ast}M\), for all \(p\in M\). This can be extended to differential forms of higher ranks.

\begin{definition}{Exterior derivative}{exterior_derivative}
    The \emph{exterior derivative} is the linear operator
    \begin{align*}
        d : \forms[k]{M} &\to \forms[k+1]{M}\\
                  \omega &\mapsto d\omega,
    \end{align*}
    defined by

    \begin{align*}
        (d\omega)\left(X_1, \dots, X_{k+1}\right) &= \sum_{i=1}^{k+1}\; (-1)^{i+1} X_i\left(\omega(X_1, \dots, \widehat{X_i}, \dots, X_{k+1})\right) \\
                                                  &+ \sum_{i < j} (-1)^{i+j} \omega\left([X_i, X_j], X_1, \dots, \widehat{X_i}, \dots, \widehat{X_j}, \dots, X_{k+1}\right),
    \end{align*}
    where \(X_i \in \sections{TM}\) are vector fields, with \(\widehat{X_m}\) denoting the omission of \(X_m\).
\end{definition}

A few results must be obtained in order to show the exterior derivative is well-defined. Namely, the definition claims the operator is linear and that \(d\omega\) is a \((k+1)\)-form. {\color{Red} These are assumed without proof.}

\begin{theorem}{Exterior derivative is an antiderivation}{exterior_antiderivation}
    Let \(\omega \in \forms[k]{M}\) and \(\sigma \in \forms[\ell]{M}\), then
    \begin{equation*}
        d(\omega \wedge \sigma) = d\omega \wedge \sigma + (-1)^{k} \omega \wedge d\sigma.
    \end{equation*}
\end{theorem}

\begin{theorem}{Exterior differentiation commutes with the pullback}{exterior_pullback}
    Let \(h : M \to N\) be a smooth map between smooth manifolds \(M\) and \(N\). Then
    \begin{equation*}
        \pb{h}(d\omega) = d(\pb{h}\omega),
    \end{equation*}
    for all \(\omega \in \forms[k]{N}.\)
\end{theorem}

By linear continuation, we may extend the exterior derivative to any differential form \(\omega \in \forms{M}.\)

\begin{example}
    We continue the examples given before.
    \begin{enumerate}[label=(\alph*)]
        \item In Maxwell electrodynamics, the homogeneous equations may be expressed as \(dF = 0\). In \(\mathbb{R}^4,\) this means \(F = dA,\) where \(A\) is the gauge potential.
        \item In classical mechanics, there exists a symplectic form \(\omega \in \forms[2]{T ^{\ast}Q}\) with \(d\omega = 0.\) There is a natural 1-form defined by \(\theta = p_i dq^i,\) from which one defines the symplectic form \(\omega = d\theta\). With this definition, one finds that \(d\theta = 0.\)
    \end{enumerate}
\end{example}


\begin{theorem}{d² = 0}{d2}
    The operator \(d \circ d : \forms[k]{M} \to \forms[k+2]{M}\) is the null operator. More succinctly, \(d^2 = 0.\)
\end{theorem}
\begin{proof}
    In local coordinates,
    \begin{equation*}
    d\omega = \partial_b\omega\indices{_{a_1\dots a_k}} dx^b \wedge dx^{a_1} \wedge \dots \wedge dx^{a_k},
    \end{equation*}
    then
    \begin{align*}
        d(d\omega) &= \partial_c\partial_b\omega\indices{_{a_1\dots a_k}} dx^c\wedge dx^b \wedge dx^{a_1} \wedge \dots \wedge dx^{a_k}\\
                   &= \partial_{(cb)}\omega\indices{_{[a_1\dots a_k]}} dx^{[c}\wedge dx^b \wedge dx^{a_1} \wedge \dots \wedge dx^{a_k]}\\
                   &= 0.
    \end{align*}
    This result is extended for the entire manifold.
\end{proof}

\subsection{deRham cohomology}
The last result implies that given the sequence of maps
\begin{equation*}
    \begin{tikzcd}[column sep = small, row sep = large]
        \forms[0]{M} \arrow{r}{d} & \forms[1]{M} \arrow{r}{d} & \dots  \arrow{r}{d} & \forms[k-1]{M} \arrow{r}{{\color{Peach}d}} & \forms[k]{M} \arrow{r}{{\color{Mauve}d}} & \forms[k+1]{M} \dots \arrow{r}{d} & \forms[n]{M},
    \end{tikzcd}
\end{equation*}
then the kernel of \({\color{Mauve} d} : \forms[k]{M} \to \forms[k+1]{M}\) contains the image of \({\color{Peach} d} : \forms[k-1]{M} \to \forms[k]{M}.\) A \(k\)-form is called \emph{exact} if it lies in the image of {\color{Peach}\(d\)}, and it is called \emph{closed} if it lies in the kernel of {\color{Mauve}\(d\)}.

The submodule of exact \(k\)-forms is denoted by \(B^k\) and the submodule of closed \(k\)-forms is denoted by \(Z^k.\) The Poincaré lemma states that if the underlying manifold is \(\mathbb{R}^n\), then \(Z^k = B^k\), for \(n > 0.\) In cases where \(N^k\) is a proper subset of \(Z^k,\) how to relate these submodules?

\begin{definition}{deRham cohomology group}{deRham}
    The \emph{\(k\)-th deRham cohomology group} is the quotient space
    \begin{equation*}
        H^k(M) = Z^k/B^k,
    \end{equation*}
    where the equivalence relation is given by
    \begin{equation*}
        \omega \sim \sigma \iff \omega - \sigma \in B^k,
    \end{equation*}
    for \(\omega,\sigma \in Z^k.\)
\end{definition}

Remarkably, the cohomology group \(H^k(M)\) depends only on the \emph{global} topology of \(M\), by deRham's theorem. As an example, \(H^0(M)\) is isomorphic to \(\mathbb{R}^d,\) where \(d\) is the number of connected pieces of \(M.\)



\printbibliography
\end{document}
