\documentclass[12pt,oneside,a4paper]{book}

% Language and formatting
\usepackage{polyglossia}
\usepackage[strict=false,autostyle=true,english=american]{csquotes} %fvextra to avoid warning?
\setmainlanguage[variant=us]{english}

\usepackage[backend=biber]{biblatex}
\addbibresource{bibliography.bib}

% \setmainfont{Palatino Linotype}
% \setmathfont{Palatino Linotype}
\usepackage[a4paper, margin=2cm]{geometry}
\usepackage{booktabs}

% title header
\usepackage{titleps}% http://ctan.org/pkg/titleps
\makeatletter
\newpagestyle{main}{% Define page style main
    \sethead%
    [\textbf\thepage][][\thechapter.\ \chaptertitle]% [<even-left>][<even-center>][<even-right>]
    {\thesection.\ \sectiontitle}{}{\textbf\thepage}% {<odd-left>}{<odd-center>}{<odd-right>}
    \setfoot{}{}{}% {<left>}{<center>}{<right>}
}
\pagestyle{main}% Use page style main

% Images
\usepackage{tikz}
\usetikzlibrary{cd}
\usepackage{graphicx, caption, subcaption}
\usepackage{float}

% Math tools
\usepackage{amsfonts, mathtools, amssymb, amsmath, amsthm, enumitem}
\usepackage{newpxtext, newpxmath}
\numberwithin{equation}{section}
\usepackage[ISO]{diffcoeff}
\usepackage{tensor}
\usepackage{siunitx}

% Misc
\usepackage{xcolor}
\usepackage[breakable]{tcolorbox}

\DeclareMathOperator\sech{sech}
\DeclarePairedDelimiter\abs{\lvert}{\rvert}
\DeclarePairedDelimiter\norm{\lVert}{\rVert}
\DeclarePairedDelimiterX\inner[2]{\langle}{\rangle}{#1,\mathopen{}#2}
\DeclarePairedDelimiter\set{\{}{\}}
\newcommand\family[2]{\ensuremath{\set{#1}_{#2}}}
\newcommand\vetor[1]{\ensuremath{\boldsymbol{#1}}}
\newcommand\topology[1]{\ensuremath{\left(#1, \mathcal{O}_{#1}\right)}}
\newcommand\restrict[2]{\ensuremath{\left.#1\right\rvert_{#2}}}

% catppuccin (latte)
\definecolor{Rosewater}{RGB}{220,138,120}
\definecolor{Flamingo}{RGB}{221,120,120}
\definecolor{Pink}{RGB}{234,118,203}
\definecolor{Mauve}{RGB}{136,57,239}
\definecolor{Red}{RGB}{210,15,57}
\definecolor{Maroon}{RGB}{230,69,83}
\definecolor{Peach}{RGB}{254,100,11}
\definecolor{Yellow}{RGB}{223,142,29}
\definecolor{Green}{RGB}{64,160,43}
\definecolor{Teal}{RGB}{23,146,153}
\definecolor{Sky}{RGB}{4,165,229}
\definecolor{Sapphire}{RGB}{32,159,181}
\definecolor{Blue}{RGB}{30,102,245}
\definecolor{Lavender}{RGB}{114,135,253}
\definecolor{Text}{RGB}{76,79,105}
\definecolor{Subtext1}{RGB}{92,95,119}
\definecolor{Subtext0}{RGB}{108,111,133}
\definecolor{Overlay2}{RGB}{124,127,147}
\definecolor{Overlay1}{RGB}{140,143,161}
\definecolor{Overlay0}{RGB}{156,160,176}
\definecolor{Surface2}{RGB}{172,176,190}
\definecolor{Surface1}{RGB}{188,192,204}
\definecolor{Surface0}{RGB}{204,208,218}
\definecolor{Base}{RGB}{239,241,245}
\definecolor{Mantle}{RGB}{230,233,239}
\definecolor{Crust}{RGB}{220,224,232}

% References
\usepackage{hyperref}
\usepackage[capitalize, nameinlink, noabbrev]{cleveref}
\makeatletter
\hypersetup{
    pdftitle=\@title,
    pdfauthor=\@author,
    colorlinks=true,
    linkcolor=Mauve,
    citecolor=pink,
    filecolor=red,
    urlcolor=blue,
    bookmarksdepth=4
}
\makeatother

% tcolorbox environments
\tcbuselibrary{theorems}
% theorem
\newtcbtheorem[number within=chapter]{theorem}{Theorem}%
{breakable,colback=Mauve!5,colframe=Mauve!95!black,fonttitle=\bfseries}{thm}
\crefname{tcb@cnt@theorem}{Theorem}{Theorems}

% definition
\newtcbtheorem[number within=chapter]{definition}{Definition}%
{breakable, colback=Pink!5,colframe=Pink!95!black,fonttitle=\bfseries}{def}
\crefname{tcb@cnt@definition}{Definition}{Definitions}

% proposition
\newtcbtheorem[number within=chapter]{proposition}{Proposition}%
{breakable,colback=Rosewater!5,colframe=Rosewater!95!black,fonttitle=\bfseries}{prop}
\crefname{tcb@cnt@proposition}{Proposition}{Propositions}

% lemma
\newtcbtheorem[number within=chapter]{lemma}{Lemma}%
{breakable,colback=Flamingo!5,colframe=Flamingo!95!black,fonttitle=\bfseries}{lem}
\crefname{tcb@cnt@lemma}{Lemma}{Lemmas}

% example
\newtheorem{example}{Example}[chapter]

% amsthm environments
% \newtheorem{definition}{Definition}[chapter]
% \newtheorem{theorem}{Theorem}[chapter]
% \newtheorem{proposition}{Proposition}[chapter]
\newtheorem{remark}{Remark}[chapter]
% \newtheorem{lemma}{Lemma}[chapter]
\newtheorem{corollary}{Corollary}[chapter]

\title{Notes on \textit{Riemannian Geometry}}
\author{Louis Bergamo Radial}

\setcounter{chapter}{0}

\begin{document}
\maketitle

\tableofcontents

\chapter{Topological Manifolds}
The theory of smooth manifolds is a very useful generalization of the differential calculus on \(\mathbb{R}^n\). Namely, a smooth manifold is a topological space endowed with a differentiable structure such that it locally resembles Euclidean space.

% \begin{definition}{Differentiable Manifolds}{manifold}
%     A \emph{differentiable manifold} of dimension \(n\) is a set \(M\) and \emph{system of coordinates}, a family \(\set{(U_\alpha, \varphi_\alpha)}_{\alpha\in J}\) of injective mappings \(\varphi_\alpha: U_\alpha \to M\) of open sets \(U_\alpha\) of \(\mathbb{R}^n\) into \(M\), such that
%     \begin{enumerate}
%         \item the union of the \emph{coordinate neighborhoods} \(\varphi_\alpha(U_\alpha)\) cover the set \(M\), that is, \[\bigcup_{\alpha\in J} \varphi_\alpha(U_\alpha) = M;\]
%         \item for any pair \(\alpha, \beta\), with \(\varphi_\alpha(U_\alpha) \cap \varphi_\beta(U_\beta) = W \neq \emptyset\), the sets \(\varphi_\alpha ^{-1}(W)\) and \(\varphi_\beta^{-1}(W)\) are open sets in \(\mathbb{R}^n\) and the mappings \(\varphi_\beta^{-1}\circ\varphi_\alpha\) are differentiable;
%         \item the \emph{system of coordinates} \(\set{U_\alpha, \varphi_\alpha}_{\alpha\in J}\) is maximal relative to the conditions above.
%     \end{enumerate}
% \end{definition}

\section{Topology}
In order to define the notion of smooth manifolds, we must first begin with some building blocks, such as topology and topological manifolds.

\begin{definition}{Topology}{topology}
    A \emph{topology} on the set \(M\) is a family \(\mathcal{O}\) of subsets of \(M\) satisfying
    \begin{enumerate}[label=(\alph*)]
        \item the empty set and the set \(M\) belong to \(\mathcal{O}\);
        \item a finite intersection of elements of \(\mathcal{O}\) is a member of \(\mathcal{O}\); and
        \item an arbitrary union of members of \(\mathcal{O}\) belongs to \(\mathcal{O}\).
    \end{enumerate}

    The pair \topology{M} is named a \emph{topological space}, elements of \(\mathcal{O}\) are called \emph{open sets} and elements of \(M\smallsetminus\mathcal{O}\) are called \emph{closed sets}. Additionally, given an element \(p \in M\) an open set \(U\) that contains \(p\) is called a \emph{neighborhood} of \(p\).
\end{definition}

In \cref{prop:standard_topology,prop:subspace_topology,prop:product_topology} we show a couple of important examples that illustrate how the axioms of topological spaces given in \cref{def:topology} are used.

\begin{proposition}{Standard topology in \(\mathbb{R}^n\)}{standard_topology}
    We define the \emph{open ball} \(B_n(r,p) \subset \mathbb{R}^n\) of radius \(r > 0\) centered at \(p = (p^1, \dots, p^n)\) as the set
    \begin{equation*}
        B_n(r, p) = \set*{q = (q^1, \dots, q^n) \in \mathbb{R}^n : \sum_{i=1}^{n}{(q^i - p^i)^2} < r^2}.
    \end{equation*}
    Next, we define the \emph{standard topology} \(\mathcal{O}_\text{standard}\) of \(\mathbb{R}^n\). A subset \(U \subset \mathbb{R}^n\) is an open set if for every point \(p \in U\) there exists \(r > 0\) such that \(B_n(r, p) \subset U\). Then, \((\mathbb{R}^n, \mathcal{O}_\text{standard})\) is a topological space.
\end{proposition}
\begin{proof}
    It is easy to see \(\mathbb{R}^n\in\mathcal{O}_{\text{standard}}\) and \(\emptyset \in \mathcal{O}_{\text{standard}}\).

    Suppose \(U, V \in \mathcal{O}_{\text{standard}}\) and let \(p \in U \cap V \neq \emptyset\).Then, there exists \(r_U > 0\) and \(r_V > 0\) such that \(B_n(r_U, p) \subset U\) and \(B_n(r_V, p) \subset V\). Setting \(r = \min\set{r_U, r_V} > 0\) we have \(B_n(r, p)\) as subset of both \(U\) and \(V\), that is, \(B_n(r, p) \subset U\cap V\). It follows that \(U\cap V\in\mathcal{O}_\text{standard}\).

    Let \family{U_\alpha}{\alpha\in J} be a family of sets in \(\mathcal{O}_\text{standard}\). Let \(p \in \bigcup_{\alpha\in J}U_\alpha\), that is, there exists \(\beta \in J\) such that \(p \in U_\beta\). Since \(U_\beta \in \mathcal{O}_\text{standard}\), there exists \(r_\beta > 0\) such that \(B_n(r, p) \subset U_\beta \subset \bigcup_{\alpha \in J} U_\alpha\).
\end{proof}

\begin{proposition}{Subspace topology is a topology}{subspace_topology}
    Given a topological space \topology{M} and a subset \(S\) of \(M\), we define the \emph{subspace topology} \restrict{\mathcal{O}_M}{S} as
    \begin{equation*}
        \restrict{\mathcal{O}_M}{S} = \set{U \cap S : U \in \mathcal{O}_M}.
    \end{equation*}
    Then \((S, \restrict{\mathcal{O}_M}{S})\) is a topological space.
\end{proposition}
\begin{proof}
    We must show the conditions (a), (b), and (c) of \cref{def:topology} are satisfied.
    \begin{enumerate}[label=(\alph*)]
        \item Since \(S = M \cap S\) and \(\emptyset = \emptyset \cap S\), we have \(S \in \restrict{\mathcal{O}_M}{S}\) and \(\emptyset \in \restrict{\mathcal{O}_M}{S}\).
        \item Let \(U, V \in \restrict{\mathcal{O}_M}{S}\). Then, there exists \(\tilde{U}, \tilde{V} \in \mathcal{O}_M\) such that \(U = \tilde{U} \cap S\) and \(V = \tilde{V} \cap S\).Then, \(U \cap V = (\tilde{U}\cap S) \cap (\tilde{V} \cap S) = (\tilde{U}\cap\tilde{V})\cap S\). Since \(\tilde{U} \cap \tilde{V} \in \mathcal{O}_M\), we have \(U \cap V \in \restrict{\mathcal{O}_M}{S}\).
        \item Let \family{U_\alpha}{\alpha \in J} be a family of open sets in \(\restrict{\mathcal{O}_M}{S}\). For each \(\alpha \in J\), there exists a \(\tilde{U}_\alpha\in\mathcal{O}_M\) such that \(U_\alpha = \tilde{U}_\alpha \cap S\). Then
            \begin{align*}
                \bigcup_{\alpha \in J} U_\alpha &= \bigcup_{\alpha \in J} \tilde{U}_\alpha \cap S\\
                                                &= \set{m \in S : \exists \alpha \in J \text{ such that } m \in \tilde{U}_\alpha}\\
                                                &= \set{m \in M : \exists \alpha \in J \text{ such that } m \in \tilde{U}_\alpha} \cap S\\
                                                &= S\cap\bigcup_{\alpha\in J}\tilde{U}_\alpha.
            \end{align*}
        Since arbitrary unions of open sets is an open set, it follows that \(\bigcup_{\alpha\in J}U_\alpha \in \restrict{\mathcal{O}_M}{S}\).
    \end{enumerate}
\end{proof}

\begin{proposition}{Product topology}{product_topology}
    Let \topology{M} and \topology{N} be topological spaces. Define the \emph{product topology} \(\mathcal{O}_{M\times N}\) as the collection of subsets \(U \subset M \times N\) such that for all \((m,n) \in U\), there exists neighborhoods \(S \subset M\) and \(T \subset N\) of \(m \in M\) and \(n\in N\) such that \(S \times T \subset U\). Then \topology{M\times N} is a topological space.
\end{proposition}
\begin{proof}
    Clearly, \(M\times N\) and \(\emptyset\) are open sets in the product topology.

    Next, we consider open sets \(U, V \in \mathcal{O}_{M\times N}\) and an element \(p \in U \cap V\). Let \(p = (m, n) \in M \times N\), then there exists neighborhoods \(S_U, S_V\subset M\) of \(m\) and \(T_U, T_V \subset N\) of \(n\) such that \(S_U \times T_U \subset U\) and \(S_V \times T_V \subset V\). Let \(S = S_U \cap S_V\) and \(T = T_U \cap T_V\), then \(S \in \mathcal{O}_M\) and \(T \in \mathcal{O}_N\) are neighborhoods of \(m\) and \(n\), respectively. Moreover, \(S \times T \subset U \cap V\) is a neighborhood of \(p\), from which follows \(U \cap V \in \mathcal{O}_{M\times N}\).

    Let \family{U_\alpha}{\alpha\in J} be a family of open sets in the product topology. Let \(p\in \bigcup_{\alpha\in J}U_\alpha\), then there exists \(\beta \in J\) such that \(p \in U_{\beta}\). By definition, there exists open sets \(S \in \mathcal{O}_M\) and \(T \in \mathcal{O}_N\) such that \(S \times T \subset U_\beta \subset \bigcup_{\alpha\in J} U_\alpha\). Therefore, \(\bigcup_{\alpha\in J}U_\alpha\) is an open set.
\end{proof}

Along with the axioms of topological spaces described in \cref{def:topology} one might add further restrictions to specify the space considered. Some common restrictions are called the \emph{separation axioms}. Among these, we will make use of the T2 axiom, namely the Hausdorff property. Historically, Felix Hausdorff used this axiom in his original definition of a topological space, although the formulation of his other axioms was not exactly as those of \cref{def:topology}, but an equivalent one.
\begin{definition}{Hausdorff space}{hausdorff}
    A topological space \topology{M} is called a \emph{Hausdorff space} if for any \(p,q\in M\) with \(p\neq q\), there exists a neighborhood \(U\) of \(p\), i.e. \(p \in U \in \mathcal{O}_M\), and a neighborhood \(V\) of \(q\) such that \(U \cap V = \emptyset\).
\end{definition}

\section{Convergence}

In analysis on \(\mathbb{R}^n\) with the standard topology, we often consider sequences \(x : \mathbb{N} \to \mathbb{R}^n\) and study whether it converges to a value. We say the sequence \(x\) converges to \(y \in \mathbb{R}^n\) if for all \(\varepsilon > 0\) there exists \(N \in \mathbb{N}\) such that \(x(i) - y \in B_n(\varepsilon, 0)\) for all \(i > N\). We generalize the notion of a convergent sequence to any topological space in \cref{def:convergence}.

\begin{definition}{Convergence of a sequence}{convergence}
    A sequence \(x : \mathbb{N} \to M\) on a topological space \topology{M} is said to \emph{converge} to a \emph{limit point} \(p \in M\) if for every neighborhood \(U \in \mathcal{O}_M\) of \(p\) there exists \(N \in \mathbb{N}\) such that \(x(n) \in U\) for all \(n > N\).
\end{definition}

\begin{theorem}{Unique limit on Hausdorff spaces}{unique_limit}
    Let \topology{M} be a Hausdorff space. If a sequence \(x\) converges on \(M\), its limit point is unique.
\end{theorem}
\begin{proof}
    Let \(p, q \in M\) be limit points of the sequence \(x\). Suppose, by contradiction, that \(p \neq q\). By the Hausdorff property, there exists neighborhoods \(U, V \in \mathcal{O}_M\) of \(p\) and \(q\), respectively, such that \(U \cap V = \emptyset\). From the definition of convergence, there exists \(N_p, N_q \in \mathbb{N}\) such that \(x(n) \in U\) for all \(n > N_p\) and \(x(n) \in V\) for all \(n > N_q\). Let \(N = \mathrm{min}\set{N_p, N_q}\), then for all \(n > N\), \(x(n) \in U\) and \(x(n) \in V\), that is, \(x(n) \in U \cap V = \emptyset\). This contradiction proves the statement.
\end{proof}

% open/closed in terms of sequences, closure

\section{Homeomorphisms}

With the notion of topological spaces, we may ask ourselves whether certain maps between topological spaces can preserve the topology. That is, a map that takes open sets in the domain topology into open sets in the target topology. To define such a map we define \emph{continuity}.

\begin{definition}{Continuous map}{continuity}
    Let \topology{M} and \topology{N} be topological spaces. Then a map \(f : M \to N\) is \emph{continuous} (with respect to \(\mathcal{O}_M\) and \(\mathcal{O}_N\)) if, for all \(V \in \mathcal{O}_N\), the preimage \(f^{-1}(V)\) is an open set in \(\mathcal{O}_M\).
\end{definition}

In short, a map is continuous if and only the preimages of (all) open sets are open sets. Now a map that preserves the topology is called a \emph{homeomorphism}, which is defined as a continuous bijection with continuous inverse. We now prove such a map satisfies the condition required.

\begin{proposition}{Homeomorphism maps open sets to open sets}{homeomorphism}
    Let \topology{M} and \topology{N} be topological spaces. Suppose a map \(f : M \to N\) is a homeomorphism, then \(f\) maps open sets in \(\mathcal{O}_M\) into open sets in \(\mathcal{O}_N\).
\end{proposition}
\begin{proof}
    Given a subset \(U \in \mathcal{O}_M\), we must show the image \(V = f(U)\) is open in \topology{N}. Taking our attention to the inverse map \(g = f^{-1} : N \to M\), we see the preimage \(g^{-1}(U) = V\) must be open in \topology{N}, due to continuity.
\end{proof}

If there exists a homeomorphism between two topological spaces, they are said to be homeomorphic to each other. This begs the question: if \topology{M} is homeomorphic to \topology{N} and \topology{N} is homeomorphic to \topology{P}, are \topology{M} and \topology{P} homeomorphic? To answer this we must show whether the composition of continuous maps is itself continuous.

\begin{theorem}{Composition of continuous maps}{continuous_composition}
    Let \topology{M}, \topology{N}, and \topology{P} be topological spaces. If the maps \(f: M \to N\) and \(g : N \to P\) are continuous (with respect to the appropriate topologies), then the map \(g \circ f : M \to P\) is continuous with respect to \(\mathcal{O_M}\) and \(\mathcal{O_P}\).
\end{theorem}
\begin{proof}
    Let \(V\) be an open set of \topology{P}. We must show the preimage \((g \circ f)^{-1}(V)\) is an open set of \topology{M}. We have
    \begin{align*}
        (g\circ f)^{-1}(V) &= \set{m \in M : g\circ f(m) \in V}\\
                           &= \set{m \in M : f(m) \in g^{-1}(V)}\\
                           &= f^{-1}\left(g^{-1}(V)\right).
    \end{align*}
    Since the map \(g\) is continuous and \(V\) is an open set in \topology{P}, it follows that \(g^{-1}(V)\) is open in \topology{N}. By the same argument, \(f^{-1}\left(g^{-1}(V)\right)\) is an open set in \topology{M}.
\end{proof}

\begin{corollary}
    If \topology{M} is homeomorphic to \topology{N} and \topology{N} is homeomorphic to \topology{P}, then \topology{M} is homeomorphic to \topology{P}.
\end{corollary}
\begin{proof}
    Let \(f : M \to N\) and \(g : N \to P\) be homeomorphisms from \topology{M} to \topology{N} and \topology{N} to \topology{P}, respectively. Consider the composition \(g\circ f : M \to P\).
    \begin{equation*}
        \begin{tikzcd}[column sep = normal, row sep = large]
            M \arrow{r}{f} \arrow[swap]{dr}{g\circ f} & N \arrow{d}{g} \\
                                                      & P
        \end{tikzcd}
    \end{equation*}
    By \cref{thm:continuous_composition}, the map \(g\circ f\) is a homeomorphism from \topology{M} to \topology{P}.
\end{proof}

As was done for the subspace topology, we prove a similar result for continuous maps.

\begin{proposition}{Restriction of a continuous map}{restriction_map}
    Let \topology{M} and \topology{N} be topological spaces and let \(f : M \to N\) be a continuous map. Let \(S\) be a subset of \(M\) and let \topology{S} be the subspace topology, then \(\restrict{f}{S} : S \to N\) is a continuous map with respect to \(\mathcal{O}_S\) and \(\mathcal{O}_N\).
\end{proposition}
\begin{proof}
    Let \(V \in \mathcal{O}_N\). Then, by the definition of preimage, we have
    \begin{align*}
        \restrict{f}{S}^{-1}(V) &= \set{s \in S : \restrict{f}{S}(s) \in V}\\
                                &= \set{s \in S : f(s) \in V}\\
                                &= f^{-1}(V) \cap S.
    \end{align*}
    By hypothesis, the preimage \(f^{-1}(V)\) is an open set in \topology{M}, so \(\restrict{f}{S}^{-1}(V)\) is an open set in the subspace topology.
\end{proof}

We can now define the notion of a topological space locally resembling Euclidean space.
\begin{definition}{Locally Euclidean topological space}{locally_euclidean}
    A topological space \topology{M} is \emph{locally Euclidean} of dimension \(n\) if for all \(m \in M\) there exists an open subset \(U \in \mathcal{O}_M\) about \(m\) that is homeomorphic to \(\mathbb{R}^n\) with respect to the subspace topology and the standard topology of \(\mathbb{R}^n\).
\end{definition}
It is sufficient to show the subspace topology \(\topology{U}\) is homeomorphic to an open ball in \(\mathbb{R}^n\), due to \cref{prop:ball_homeomorphic_euclidean}.
\begin{proposition}{Open ball is homeomorphic to the Euclidean space}{ball_homeomorphic_euclidean}
    Let \(r > 0\), then the map \(f : B_n(r, 0)\subset\mathbb{R}^n\to\mathbb{R}^n\) given by
    \[f(x) = \frac{x}{r - \norm{x}}\]
    is a homeomorphism with respect to the standard topology.
\end{proposition}
\begin{proof}
    We begin by checking \(f\) is one-to-one and onto.

    Suppose there exists \(x_1, x_2 \in B_n(r, 0)\) such that \(f(x_1) = f(x_2)\). It follows from
    \begin{align*}
        f(x_2) - f(x_1) &= \frac{x_2}{r - \norm{x_2}} - \frac{x_1}{r - \norm{x_1}}\\
                        &= \frac{\left(r - \norm{x_1}\right)x_2 - \left(r - \norm{x_2}\right)x_1}{\left(r - \norm{x_2}\right)\left(r - \norm{x_1}\right)}
    \end{align*}
    that \(\left(r - \norm{x_1}\right)x_2 = \left(r - \norm{x_2}\right)x_1\). Taking the norm on both sides, we have \(\norm{x_1} = \norm{x_2}\). Substituting back, we have \(x_1 = x_2\), proving \(f\) is injective.

    Suppose \(y \in \mathbb{R}^n\) and consider \(\xi = \frac{ry}{1 + \norm{y}}\). Clearly, \(\xi \in B_n(r,0)\). We have
    \begin{align*}
        f(\xi) &= f\left(\frac{ry}{1 + \norm{y}}\right)\\
               &= \frac{ry}{1 + \norm{y}} \frac{1}{r - \norm*{\frac{ry}{1 + \norm{y}}}}\\
               &= \frac{1}{\left(1 + \norm{y}\right)\left(1 - \frac{\norm{y}}{1 + \norm{y}}\right)} y\\
               &= y,
    \end{align*}
    so \(f\) is onto.

    We have shown \(f\) is a bijection with inverse \(f^{-1} : \mathbb{R}^n \to B_n(r, 0)\) defined by
    \begin{equation}
        f^{-1}(x) = \frac{rx}{1 + \norm{x}}.
    \end{equation}
    With the standard topology, continuity of \(f\) and \(f^{-1}\) follows from techniques of elementary calculus, and we conclude \(f\) is a homeomorphism.
\end{proof}

\section{Compactness and paracompactness}

\begin{definition}{Compactness}{compact}
    A topological space \topology{M} is \emph{compact} if every \emph{open cover} of \(M\) has a finite subcover. That is, the topological space is compact if for every family of open sets \(C\) that covers \(M\), i.e. \(\bigcup_{U \in C}U = M\) with \(U \in \mathcal{O}_M\), there exists a finite family of open sets \(F \subset C\) such that \(\bigcup_{U\in F} U = M\).

    Additionally, in a topological space \topology{N}, a subset \(S\subset N\) is called compact if the subspace topology is compact.
\end{definition}

\begin{theorem}{Heine-Borel theorem}{heine_borel}
    A subset \(S\subset\mathbb{R}^n\) with the standard topology is compact if it is closed and bounded.
\end{theorem}
\begin{proof}
    Refer to \cite{babyrudin}.
\end{proof}

\begin{definition}{Locally finite cover}{locally_finite}
    A cover \(C\) of a topological space \topology{M} is called \emph{locally finite} if each point in the space has a neighborhood that intersects only finitely many sets in \(C\). More precisely, for all \(p \in M\) there exists a neighborhood \(U \in \mathcal{O}_M\) about \(p\) such that \(U \cap V \neq \emptyset\) only for finitely many \(V \in C\).
\end{definition}

\begin{definition}{Refinement}{refinement}
    A \emph{refinement} of a cover \(C\) of a topological space \topology{M} is a cover \(D\) such that every set in \(D\) is contained in some set in \(C\). Precisely, let \(C = \family{U_\alpha}{\alpha \in A}\) and \(D = \family{V_\beta}{\beta \in B}\) such that \(\bigcup_{\alpha \in A} U_\alpha = M\) and \(\bigcup_{\beta \in B} V_\beta = M\), then \(D\) is a refinement of \(C\) if for all \(\beta \in B\) there exists \(\alpha \in A\) such that \(V_\beta \subset U_\alpha\).
\end{definition}

\begin{definition}{Paracompactness}{paracompact}
    A topological space \topology{M} is called \emph{paracompact} if every open cover \(C\) has an \emph{open refinement} \(\tilde{C}\) that is \emph{locally finite}.
\end{definition}

\begin{theorem}{Stone's theorem}{stone}
    Any \emph{metrizable space} is paracompact.
\end{theorem}
\begin{proof}
    Refer to \cite{munkres_topology}.
\end{proof}

\begin{definition}{Partition of unity subordinate to a cover}{partition_of_unity}
    Let \(C = \family{U_\alpha}{\alpha \in J}\) be an open cover of the topological space \topology{M}. The family of functions \(\mathcal{F}_C = \family{f_\alpha}{\alpha \in J}\) from \(M\) to \([0, 1]\subset \mathbb{R}\) is a (continuous) \emph{partition of unity subordinate to \(U_\alpha\)} if
    \begin{enumerate}[label=(\alph*)]
        \item each \(f_\alpha : X \to [0,1]\) is continuous;
        \item for each \(\alpha \in J\), the support of \(f_\alpha\) is contained in \(U_\alpha\), that is, for every \(f \in \mathcal{F}_C\) there exists \(U \in C\) such that \(f(p) \neq 0 \implies p \in U\);
        \item the family \family{\mathrm{supp} f_\alpha}{\alpha\in J} is locally finite, i.e. for any \(p \in M\) there exists a neighborhood \(U \in \mathcal{O}_M\) about \(p\) where all but finitely many functions \(f_\alpha\) vanish on \(U\);
        \item for any point \(p \in M\), \(\sum_{\alpha \in J} f_\alpha = 1\).
    \end{enumerate}
    % A \emph{partition of unity} of a topological space \topology{M} is a set \(\mathcal{F}\) of continuous functions from \(M\) to \([0,1]\subset\mathbb{R}\) such that for every point \(p \in M\)
    % \begin{enumerate}[label=(\alph*)]
    %     \item there exists a neighborhood \(U \in \mathcal{O}_M\) about \(p\) where all but finitely many functions of \(\mathcal{F}\) vanish on \(U\);
    %     \item the sum of all function values at \(p\) is 1, that is, \(\sum_{f \in \mathcal{F}} f(p) = 1\).
    % \end{enumerate}
    %
    % Moreover, let \(C = \family{U_\alpha}{\alpha \in J}\) be an open cover of \(M\). A \emph{partition of unity subordinate to the open cover \(C\)} is a family \(\mathcal{F}_C\) of continuous maps \(f_\alpha : p \to [0,1] \subset\mathbb{R}\) indexed over the same set \(J\) such that the support of \(f_\alpha\) is contained in \(U_\alpha\), for all \(\alpha \in J\). That is, for every \(f \in \mathcal{F}_C\) there exists an open set \(U \in C\) such that \(f(p) \neq 0 \implies p \in U\).
\end{definition}

\begin{theorem}{Paracompactness and partitions of unity}{hausdorff_paracompact}
    Let \topology{M} be a Hausdorff space. Then it is paracompact if and only if every open cover \(C\) admits a partition of unity subordinate to that cover.
\end{theorem}
\begin{proof}
    Refer to \cite{munkres_topology}.
\end{proof}

\section{Connectedness and path-connectedness}
\begin{definition}{Connectedness}{connectedness}
    A topological space \topology{M} is \emph{connected} unless there exists two non-empty, non-intersecting open sets \(A, B \in \mathcal{O}_M\) such that \(M = A \cup B\).
\end{definition}

\begin{theorem}{Interval is connected}{interval}
    Every interval \(I \subset \mathbb{R}\) is connected with respect to the standard topology.
\end{theorem}
\begin{proof}
    Suppose \(I\) is not connected, then \(I = A \cup B\), where \(A, B \subset I\) are non-empty, non-intersecting open sets. Let \(a \in A\) and \(b \in B\). Without loss of generality, we assume \(a < b\).

    Consider \(\alpha = \sup\set{x \in \mathbb{R} : [a, x) \cap I \subset A}\). Then \(a \leq \alpha \leq b\), so \(\alpha \in I\). Since \(B = I \smallsetminus A\) is open, we have \(A\) closed, hence \(\alpha \in A\). Since \(A\) is open, there exists \(r > 0\) such that \((\alpha - r, \alpha + r)\cap I \subset A\). We conclude \((a, \alpha+r)\cap I \subset A\), which is a contradiction.
\end{proof}

\begin{theorem}{Open and closed subsets in a connected space}{clopen}
    A topological space \topology{M} is connected if and only if \(\emptyset\) and \(M\) are the only subsets that are both open and closed.
\end{theorem}
\begin{proof}
    Suppose the topological space is connected. Suppose, by contradiction, the non-empty subset \(U \subsetneq M\) is open and closed. It follows from \(M = U \cup (M\smallsetminus U)\) that the topological space is not connected, a contradiction.

    Suppose the empty set and \(M\) are the only subsets that are both open and closed. Suppose, by contradiction, that the topological space is not connected. Then there exists two non-empty non-intersecting open sets \(A,  B \in \mathcal{O}_M\) such that \(M = A \cup B\). Clearly, \(B = M\smallsetminus A\) is closed, and likewise, \(A\) is closed. By hypothesis, \(A = B = M\) since they are both open and closed and are non-empty. We have thus arrived at a contradiction, since \(A\cap B = M\) is non-empty, which proves the topological space is connected.
\end{proof}

\begin{definition}{Path-connectedness}{pathconnected}
    A topological space \topology{M} is called \emph{path-connected} if for every pair of points \(p, q\in M\) there exists a continuous curve \(\gamma : [0, 1] \to M\) such that \(\gamma (0) = p\) and \(\gamma(1) = q\).
\end{definition}

\begin{theorem}{Path-connectedness implies connectedness}{pathconnected_implies_connectedness}
    A path-connected topological space is connected.
\end{theorem}
\begin{proof}
    Suppose, by contradiction, a topological space \topology{M} is path-connected but not connected. Then, there exists non-empty, non-intersecting open sets \(A, B \in \mathcal{O}_M\) such that \(M = A \cup B\). Choose \(a \in A\) and \(b \in B\). Since \topology{M} is path-connected, there exists a continuous curve \(\gamma: [0,1] \to M\) such that \(\gamma(0) = a\) and \(\gamma(1) = b\). Consider the preimage \([0,1] = \gamma^{-1}(M)\). If follows from
    \begin{align*}
        \gamma^{-1}(M) &= \gamma^{-1}(A\cup B)\\
                       &= \set*{x \in [0,1] : \gamma(x) \in A \cup B}\\
                       &= \gamma^{-1}(A) \cup \gamma^{-1}(B)
    \end{align*}
    that \([0,1]\) is the union of two non-empty sets, since \(0 \in \gamma^{-1}(A)\) and \(1 \in \gamma^{-1}(B)\). Suppose \(\gamma^{-1}(A) \cap \gamma^{-1}(B)\) is non-empty. Then there exists \(t \in [0,1]\) such that \(\gamma(t) \in A\) and \(\gamma(t) \in B\), that is, \(\gamma(t) \in A \cap B\). This contradiction shows \(\gamma^{-1}(A)\) and \(\gamma^{-1}(B)\) are non-intersecting. By continuity, these sets are both open and closed. Therefore, \([0,1]\) is not connected. By \cref{thm:interval}, this is a contradiction, and the theorem follows.
\end{proof}

\section{Homotopic curves and the fundamental group}

\begin{definition}{Group}{group}
    A \emph{group} is a non-empty set \(G\), called the \emph{underlying set}, closed under a map \(\cdot : G \times G \to G\), named \emph{group operation}, such that the \emph{group axioms} are satisfied:
    \begin{enumerate}[label=(\alph*)]
        \item Associativity: For all \(a,b,c \in G\), on has \((a\cdot b)\cdot c = a \cdot (b\cdot c)\).
        \item Identity element: There exists a unique element \(e \in G\), called the \emph{identity element} or \emph{neutral element} of the group, such that for every \(g \in G\), \(e \cdot g = g\) and \(g \cdot e = g\).
        \item Inverse element: For each \(g \in G\), there exists a unique element \(g^{-1} \in G\) such that \(g \cdot g^{-1} = e\) and \(g^{-1} \cdot g = e\).
    \end{enumerate}
    If the group operation is commutative, \(G, \cdot\) is named an \emph{abelian group}.
\end{definition}
\begin{remark}
     If the group operation is notated as addition, the inverse element of \(g\) is typically denoted by \(-g\), the identity element denoted by 0, and the group may be called an \emph{additive group}. Similarly, a group may be called \emph{multiplicative group} if the group operation is notated as multiplication, and the identity element is denoted by 1. Additionally, in a multiplicative group, the group operation may be denoted simply by juxtaposition, that is, \(f \cdot g = fg\).
\end{remark}

\begin{definition}{Homotopic curves}{homotopy}
    Let \topology{M} be a topological space and let \(p, q \in M\). Two curves \(\gamma,\eta: [0,1]\to M\) from \(p\) to \(q\), i.e. \(\gamma(0) = \eta(0) = p\) and \(\gamma(1) = \eta(1) = q\), are \emph{homotopic} if there exists a continuous map \(h : [0,1] \times [0,1] \to M\) such that \(h(0, \lambda) = \gamma(\lambda)\) and \(h(1, \lambda) = \eta(\lambda)\), for all \(\lambda \in [0,1]\).
\end{definition}

It is easy to check that homotopic curves form a equivalence relation. (do that)

\begin{definition}{Loops at a point}{loops}
    Let \topology{M} be a topological space and let \(p \in M\). We define the \emph{space of loops at \(p\)} as the set \(\mathscr{L}_p\) of continuous curves \(\gamma : [0,1] \to M\) such that \(\gamma(0) = \gamma(1) = p\).

    A \emph{concatenation} is a binary operation \(\ast_p : \mathscr{L}_p \times \mathscr{L}_p \to \mathscr{L}_p\) defined by
    \begin{equation*}
        (\gamma \ast_p \eta)(\lambda) = \begin{cases}\gamma(2\lambda), & \text{ for } \lambda \in \left[0,\frac12\right]\\\eta(2\lambda -1), &\text{ for }\lambda \in \left(\frac12, 1\right]\end{cases}
    \end{equation*}
    for all \(\gamma,\eta \in \mathscr{L}_p\).
\end{definition}

\begin{definition}{Fundamental group}{fundamental_group}
    The \emph{fundamental group \((\pi_{1,p}, \cdot)\)} of a topological space is the set
    \begin{equation*}
        \pi_{1,p} = \mathscr{L}_p / \mathrm{homotopy} = \set{[\gamma]_{\mathrm{homotopy}} : \gamma \in \mathscr{L}_p}
    \end{equation*}
    together with the product \(\cdot : \pi_{1,p} \times \pi_{1,p} \to \pi_{1,p}\) defined by
    \begin{equation*}
        [\gamma] \cdot [\eta] = [\gamma \ast_p \eta].
    \end{equation*}
\end{definition}

\begin{example}
    \begin{enumerate}[label=(\alph*)]
        \item On the two-sphere \(S^2\), the fundamental group has a single element, represented by the constant loop.
        \item For the cylinder \(C = \mathbb{R}\times S^1\), the fundamental group is homomorphic to the group \((\mathbb{Z}, +)\). There exists a bijection \(f : \pi_1 \to \mathbb{Z}\) with the property \(f(\alpha\cdot\beta) = f(\alpha) + f(\beta)\).
        \item On the torus \(T^2 = S^1 \times S^1\), we have \(\pi_1\) isomorphic to \(\mathbb{Z}\times\mathbb{Z}\).
    \end{enumerate}
\end{example}

\begin{definition}{Simply connected}{simply_connected}
    A topological space \topology{M} is \emph{simply connected} if it is path-connected and if for every point \(p \in M\) the fundamental group \((\pi_{1,p}, \cdot)\) is the trivial group.
\end{definition}
\begin{remark}
    Building up on the previous examples, we see that the two-sphere is simply connected, while the cylinder and the torus are not, although path-connected.
\end{remark}

\section{Topological manifolds}

We now finally define the notion of a manifold, which is a "well-behaving" topological space that \emph{locally} looks like Euclidean space.

% second countable?
\begin{definition}{Topological manifold}{topological_manifold}
    A paracompact Hausdorff space \topology{M} locally Euclidean of dimension \(n\) is called a \emph{\(n\)-dimensional topological manifold}.
\end{definition}

\begin{example}
    We list a few examples of constructions of new manifolds from other manifolds.
    \begin{enumerate}[label=(\alph*)]
        \item As with topological spaces, it is clear that the subspace topology \topology{N} is in its own right a manifold. This construction takes the name of \emph{submanifold}.
        \item Let \topology{M} be an \(m\)-dimensional manifold and \topology{N} be an \(n\)-dimensional manifold. Then the product topology \topology{M\times N} is an \((n+m)\)-dimensional manifold. As an example, the circle \(S^1\) is a 1-dimensional manifold, so the torus \(T^2 = S^1 \times S^1\) is a 2-dimensional manifold.
    \end{enumerate}
\end{example}

\subsection{Bundles}

It is clear we may not always construct a manifold as a product manifold. To see this, take the Möbius strip. %I have no idea what I'm doing, one day I'll learn this properly.
We generalize this concept with \emph{bundles}.

\begin{definition}{Bundles of topological manifolds}{bundle}
    A \emph{bundle of topological manifolds} is a triple \((E, \pi, M)\), where \topology{E} is a topological manifold called the \emph{total space}, \topology{M} is a topological manifold called the \emph{base space}, and \(\pi : E \to M\) is a continuous surjective map called the \emph{projection}. Let \(p \in M\), then \(\pi^{-1}(\set{p}) = F_p\) is the \emph{fiber at \(p\).}
\end{definition}
\begin{example}
    \begin{enumerate}[label=(\alph*)]
        \item The first example is when the total space \(E = M \times F\) is a product manifold of the base space \(M\) and a fiber \(F\). We take the projection \(\pi : M \times F \to M\) as the continuous map \((p, f) \mapsto p\).
        \item We now take the Möbius strip as the total space and the base space as \(S^1\). For any point in \(S^1\), the preimage of the projection is some interval on the real line.  % I'm not sure I understood this properly.
        \item We consider \(M = \mathbb{R}\) as the base space. We begin construct the total space by attaching to each point of \(M\) a circle. Then, we continuously deform the circles such that they collapse to a point on the positive numbers of the real line. Finally, we continuously stretch the point on that half of the line to increasing intervals. The fiber at any given point may be homeomorphic to a circle or to a point, or to an interval.
        % I don't see how this total space is a manifold
    \end{enumerate}
\end{example}

In the last example, a bundle was constructed with a fiber space that is not the same at different points. This motivates the following definition.

\begin{definition}{Fiber bundle}{fiber_bundle}
    A bundle \((E, \pi, M)\) is a \emph{fiber bundle with typical fiber \(F\)} if for all \(p \in M\), the preimage \(\pi^{-1}(\{p\})\) is homeomorphic to the manifold \(F\). The diagram
    \begin{equation*}
        \begin{tikzcd}[row sep = large, column sep = normal]
            F \arrow{r}{} & E \arrow{r}{\pi} & M
        \end{tikzcd}
    \end{equation*}
    denotes a fiber bundle.
\end{definition}

As an example, we consider the \emph{\(\mathbb{C}\)-line bundle over \(M\)} as the bundle \((E, \pi, M)\) with typical fiber \(\mathbb{C}\).

\begin{definition}{Section of the bundle}{bundle_section}
    Let \((E, \pi, M)\) be a bundle. A map \(\sigma : M \to E\) is called a section of the bundle if \(\pi \circ \sigma = \mathrm{id}_M\).
\end{definition}

As a special case we consider the "product bundle"
\begin{equation*}
    \begin{tikzcd}[column sep = normal, row sep = large]
        F \arrow{r}{} & M \times F \arrow{r}{\pi} & M,
    \end{tikzcd}
\end{equation*}
where \(\pi : M \times F \to M\) is the projection \((p, f) \mapsto p\). Given any function \(s : M \to F\), we may construct a section \(\sigma : M \to M \times F\) defined by \(p\mapsto (p, s(p))\). % As an example, we consider the \(\mathbb{C}\)-line bundle over \(M\) in quantum mechanics, and notice a wave function is a section of the \(\mathbb{C}\)-line bundle over the physical space.

Given a bundle \((E, \pi, M)\), we may consider the submanifolds \(E' \subset E\) and \(M' \subset M\) and the restriction \(\pi' = \restrict{\pi}{E'}\), then we call \((E', \pi', M')\) a \emph{subbundle}. Similarly, we consider another submanifold \(N \subset M\) and the preimage \(G = \pi^{-1}(N)\), then \((G, \restrict{\pi}{G}, N)\) is called the \emph{restricted bundle.}

Given two bundles \((E, \pi, M)\) and \((E', \pi', M')\) and a pair of maps \(u: E \to E'\) and \(f: M \to M'\). Then the pair \((u,f)\) is called a \emph{bundle morphism} if the diagram
\begin{equation*}
    \begin{tikzcd}[column sep = normal, row sep = large]
        E \arrow{r}{u} \arrow{d}{\pi} & E' \arrow{d}{\pi'}\\
        M \arrow{r}{f} & M'
    \end{tikzcd}
\end{equation*}
commutes. The bundles are called \emph{isomorphic as bundles} if there exists bundle morphisms \((u, f)\) and \((u^{-1}, f^{-1})\). In this case \((u,f)\) are called \emph{bundle isomorphisms} and are the structure-preserving maps for bundles.

We may weaken this condition by not requiring it globally. Two bundles \((E, \pi, M)\) and \((E', \pi', M')\) are \emph{locally isomorphic as bundles} if for every \(p \in M\) there exists a neighborhood \(U \in \mathcal{O}_M\) such that the restricted bundle \((\pi^{-1}(U), \restrict{\pi}{\pi^{-1}(U)}, U)\) is isomorphic to \((E', \pi', M')\).

A bundle is called \emph{(locally) trivial} if it is (locally) isomorphic to a product bundle. As an example, the cylinder is trivial and thus locally trivial and a Möbius strip is not trivial, but it is locally trivial. From now on we will disregard bundles that are not locally trivial. Then, locally, any section can be represented as a map from the base space to fiber.

\begin{definition}{Pullback bundle}{pullback_bundle}
    Let \((E, \pi, M)\) be a bundle, \(M'\) be a manifold and \(f : M' \to M\) be a continuous map.
    \begin{equation*}
        \begin{tikzcd}[column sep = normal, row sep = large]
            f^\ast E \arrow{d}{\pi'} \arrow{r}{u} & E \arrow{d}{\pi} \\
            M' \arrow{r}{f} & M
        \end{tikzcd}
    \end{equation*}
    Let \(f^\ast E = \set{(m', e) \in M'\times E : \pi(e) = f(m')} \subset M' \times E\) equipped with the subspace topology, define the map \(u : f^\ast E \to E\) by \( (m', e) \mapsto e\) and the projection \(\pi' : f^\ast E \to M'\) by \((m',e)\mapsto m'\) such that the diagram above commutes. The bundle \((f^\ast E, \pi', M')\) is called the \emph{pullback bundle by \(f\)} or the \emph{bundle induced by \(f\)}.
\end{definition}

Given a bundle \((E, \pi, M)\) with section \(\sigma : M \to E\) and a continuous map \(f : M' \to M\), where \(M'\) is a manifold, we may define the pullback section \(f^\ast \sigma : M' \to f^\ast E\) on the bundle induced by \(f\) by the map \( m' \mapsto (m', \sigma\circ f(m'))\). Since \(\pi \circ \sigma = \mathrm{id}_M\), it is clear the image of the pullback section is contained in \(f^\ast E\), so the map is well-defined.

\subsection{Atlas}

Since a topological manifold is locally Euclidean, for any point there is a neighborhood homeomorphic to Euclidean space. That is, there exists a homeomorphism from one such neighborhood to Euclidean space. Moreover, the collection of all such neighborhoods must cover the manifold. These remarks motivate the \cref{def:chart,def:atlas}.

\begin{definition}{Chart of a manifold}{chart}
    Let \topology{M} be a topological manifold of dimension \(n\). Then a pair \((U, x)\) where \(U \in \mathcal{O}_M\) and \(x: U \to x(U) \subset \mathbb{R}^n\) is called a \emph{chart} of the manifold.

    The component functions of x, the maps \(x^i : U \to \mathbb{R}\) defined by \(p \mapsto \mathrm{proj}_i(x(p))\), are called the \emph{coordinates of the point \(p \in U\) with respect to the chart \((U, x)\).}
\end{definition}

\begin{definition}{Atlas of a manifold}{atlas}
    Let \topology{M} be a topological manifold. The \emph{atlas} \(\mathscr{A} = \family{(U_\alpha, x_\alpha)}{\alpha \in J}\) is a family of charts of the manifold such that \(\bigcup_{\alpha\in J}{U_{\alpha}} = M\).
\end{definition}

It is easy to see that the domains of different charts may overlap. Let \((U, x)\) and \((V, y)\) be charts of a manifold \topology{M} with \(U\cap V \neq \emptyset\). Since \(U \cap V \in \mathcal{O}_M\) is a non-empty open set, the pairs \((U\cap V, x)\) and \((U \cap V, y)\) are charts on the manifold.
\begin{equation*}
    \begin{tikzcd}[column sep = normal, row sep = large]
        & U\cap V \arrow[swap]{ld}{x} \arrow{rd}{y} & \\
        x(U\cap V) \arrow{rr}{y\circ x^{-1}} && y(U\cap V)
    \end{tikzcd}
\end{equation*}
As the maps \(x\) and \(y\) are bijections, the map \(y \circ x^{-1} : x(U\cap V) \to y(U \cap V)\) is well defined and it is called the \emph{chart transition map}. By \cref{thm:continuous_composition}, the transition map is a homeomorphism.

The following definitions will seem redundant, but will serve as a framework of definitions as we require more and more structure on manifolds. Namely, later on we will replace the continuity requirement with a differentiability class.

\begin{definition}{\(\mathcal{C}^0\)-compatible charts}{c0_compatible}
    Two charts \((U, x)\) and \((V, y)\) of a \(n\)-dimensional manifold \topology{M} are \emph{\(\mathcal{C}^0\)-compatible} if either
    \begin{enumerate}[label=(\alph*)]
        \item \(U \cap V = \emptyset\); or
        \item \(U \cap V \neq \emptyset\) and the transition map \(y \circ x^{-1}\) is continuous as a map \(\mathbb{R}^n \to \mathbb{R}^n\).
    \end{enumerate}
\end{definition}

As discussed above, it is clear that any two charts on a manifold are \(\mathcal{C}^0\)-compatible. However, as we can study, for example, the differentiability class of the transition map as a map \(\mathbb{R}^n \to \mathbb{R}^n\), we may not conclude any such thing from the chart maps before adding structure to the manifold.

\begin{definition}{\(\mathcal{C}^0\)-atlas}{c0_atlas}
    A \emph{\(\mathcal{C}^0\)-atlas} \(\mathscr{A}\) is an atlas whose charts are pairwise \(\mathcal{C}^0\)-compatible.
\end{definition}

\begin{definition}{Maximal \(\mathcal{C}^0\)-atlas}{max_c0_atlas}
    A \(\mathcal{C}^0\)-atlas \(\mathscr{A}\) is \emph{maximal} if any chart \((U, x)\) that is \(\mathcal{C}^0\)-compatible with any \((V, y) \in \mathscr{A}\) is already contained in \(\mathscr{A}\).
\end{definition}

Even though every atlas is a \(\mathcal{C}^0\)-atlas, not every atlas is maximal. As an example we consider \topology{M} as the real line equipped with the standard topology. Then, the family \(\set{(\mathbb{R}, \mathrm{id}_{\mathbb{R}})}\) is an atlas, but the chart \(((-\infty, 0), \mathrm{id}_{\mathbb{R}})\) is not contained in it.

With an atlas and charts, one may study objects on an \(n\)-dimensional topological manifold \topology{M} from different points of view. As an example, we consider a curve \(\gamma : \mathbb{R} \to M\) and ask ourselves whether the curve is continuous. Clearly, by definition, we can check whether the preimage of open sets in \(M\) are open. From another perspective, we may employ charts.
\begin{equation*}
    \begin{tikzcd}[column sep = normal, row sep = large]
        & y(U) \subset \mathbb{R}^n \\
        \gamma^{-1}(U) \subset \mathbb{R} \arrow{r}{\gamma} \arrow{ur}{y\circ \gamma} \arrow[swap]{dr}{x\circ \gamma} &  U \subset M\arrow[swap]{u}{y} \arrow{d}{x}\\
        & x(U) \subset \mathbb{R}^n \arrow[bend right = 60, swap]{uu}{y\circ x^{-1}}
    \end{tikzcd}
\end{equation*}
Let \(U \subset M\) be an open set that contains the image of the curve, and let \((U, x)\) be a chart. The \emph{expression} of \(\gamma\) in this chart is the map \(x\circ \gamma: \gamma^{-1}(U) \subset \mathbb{R} \to x(U) \subset \mathbb{R}^n\). Then, the curve is continuous if its expression is continuous as a function from \(\mathbb{R} \to \mathbb{R}^n\), that is, if its components are continuous real-valued functions of a single variable. Additionally, if \((U, y)\) is another chart, then the expression of the curve in this chart \(y \circ \gamma : \gamma^{-1}(U) \to y(U)\) is continuous if and only if the expression in the other chart \(x\circ \gamma\) is continuous, because the chart transition map \(y\circ x^{-1} : x(U) \to y(U)\) is continuous. In this perspective, it is possible to ignore altogether the inner workings of the manifolds and use only the coordinate systems given by the charts.

Analogously, a map \(\phi : M \to N\), where \topology{M} is an \(m\)-dimensional topological manifold and \topology{N} is an \(n\)-dimensional topological manifold, is continuous if preimages of open sets in \(N\) are open sets in \(M\). By employing charts \((U, x)\) on \(M\) and \((V, y)\) on \(N\), the expression of the map is \(y \circ \phi \circ x^{-1} : x(U) \subset \mathbb{R}^n \to y(V) \subset \mathbb{R}^n\), and its continuity can be determined as a function of \(\mathbb{R}^m \to \mathbb{R}^n\).
\begin{equation*}
    \begin{tikzcd}[column sep = large, row sep = large]
        \tilde{x}(U) \subset \mathbb{R}^m \arrow{r}{\tilde{y}\circ \phi \circ \tilde{x}^{-1}} & \tilde{y}(V) \subset \mathbb{R}^n\\
        U \subset M \arrow{d}{x} \arrow[swap]{u}{\tilde{x}} \arrow{r}{\phi} & V \subset N \arrow{d}{y} \arrow[swap]{u}{\tilde{y}}\\
        x(U) \subset \mathbb{R}^m \arrow{r}{y\circ \phi \circ x^{-1}} \arrow[bend left=60]{uu}{\tilde{x} \circ x^{-1}} & y(V) \subset \mathbb{R}^n \arrow[bend right=60, swap]{uu}{\tilde{y}\circ y^{-1}}
    \end{tikzcd}
\end{equation*}
Taking another pair of charts \((U, \tilde{x})\) and \((V, \tilde{y})\), it follows from the continuity of the chart transition maps that the expression of \(\phi\) in these charts \(\tilde{y} \circ \phi \circ \tilde{x}^{-1} : \tilde{x}(U) \subset \mathbb{R}^n \to \tilde{y}(V) \subset \mathbb{R}^n\) is a continuous function if and only if the expression \(y \circ \phi\circ{x}^{-1}\) is continuous.


The theory of smooth manifolds is a very useful generalization of the differential calculus on \(\mathbb{R}^n\). Namely, a smooth manifold is a topological space endowed with a differentiable structure such that it locally resembles Euclidean space.

% \begin{definition}{Differentiable Manifolds}{manifold}
%     A \emph{differentiable manifold} of dimension \(n\) is a set \(M\) and \emph{system of coordinates}, a family \(\set{(U_\alpha, \varphi_\alpha)}_{\alpha\in J}\) of injective mappings \(\varphi_\alpha: U_\alpha \to M\) of open sets \(U_\alpha\) of \(\mathbb{R}^n\) into \(M\), such that
%     \begin{enumerate}
%         \item the union of the \emph{coordinate neighborhoods} \(\varphi_\alpha(U_\alpha)\) cover the set \(M\), that is, \[\bigcup_{\alpha\in J} \varphi_\alpha(U_\alpha) = M;\]
%         \item for any pair \(\alpha, \beta\), with \(\varphi_\alpha(U_\alpha) \cap \varphi_\beta(U_\beta) = W \neq \emptyset\), the sets \(\varphi_\alpha ^{-1}(W)\) and \(\varphi_\beta^{-1}(W)\) are open sets in \(\mathbb{R}^n\) and the mappings \(\varphi_\beta^{-1}\circ\varphi_\alpha\) are differentiable;
%         \item the \emph{system of coordinates} \(\set{U_\alpha, \varphi_\alpha}_{\alpha\in J}\) is maximal relative to the conditions above.
%     \end{enumerate}
% \end{definition}

\section{Topology}
In order to define the notion of smooth manifolds, we must first begin with some building blocks, such as topology and topological manifolds.

\begin{definition}{Topology}{topology}
    A \emph{topology} on the set \(M\) is a family \(\mathcal{O}\) of subsets of \(M\) satisfying
    \begin{enumerate}[label=(\alph*)]
        \item the empty set and the set \(M\) belong to \(\mathcal{O}\);
        \item a finite intersection of elements of \(\mathcal{O}\) is a member of \(\mathcal{O}\); and
        \item an arbitrary union of members of \(\mathcal{O}\) belongs to \(\mathcal{O}\).
    \end{enumerate}

    The pair \topology{M} is named a \emph{topological space}, elements of \(\mathcal{O}\) are called \emph{open sets} and elements of \(M\smallsetminus\mathcal{O}\) are called \emph{closed sets}. Additionally, given an element \(p \in M\) an open set \(U\) that contains \(p\) is called a \emph{neighborhood} of \(p\).
\end{definition}

In \cref{prop:standard_topology,prop:subspace_topology,prop:product_topology} we show a couple of important examples that illustrate how the axioms of topological spaces given in \cref{def:topology} are used.

\begin{proposition}{Standard topology in \(\mathbb{R}^n\)}{standard_topology}
    We define the \emph{open ball} \(B_n(r,p) \subset \mathbb{R}^n\) of radius \(r > 0\) centered at \(p = (p^1, \dots, p^n)\) as the set
    \begin{equation*}
        B_n(r, p) = \set*{q = (q^1, \dots, q^n) \in \mathbb{R}^n : \sum_{i=1}^{n}{(q^i - p^i)^2} < r^2}.
    \end{equation*}
    Next, we define the \emph{standard topology} \(\mathcal{O}_\text{standard}\) of \(\mathbb{R}^n\). A subset \(U \subset \mathbb{R}^n\) is an open set if for every point \(p \in U\) there exists \(r > 0\) such that \(B_n(r, p) \subset U\). Then, \((\mathbb{R}^n, \mathcal{O}_\text{standard})\) is a topological space.
\end{proposition}
\begin{proof}
    It is easy to see \(\mathbb{R}^n\in\mathcal{O}_{\text{standard}}\) and \(\emptyset \in \mathcal{O}_{\text{standard}}\).

    Suppose \(U, V \in \mathcal{O}_{\text{standard}}\) and let \(p \in U \cap V \neq \emptyset\).Then, there exists \(r_U > 0\) and \(r_V > 0\) such that \(B_n(r_U, p) \subset U\) and \(B_n(r_V, p) \subset V\). Setting \(r = \min\set{r_U, r_V} > 0\) we have \(B_n(r, p)\) as subset of both \(U\) and \(V\), that is, \(B_n(r, p) \subset U\cap V\). It follows that \(U\cap V\in\mathcal{O}_\text{standard}\).

    Let \family{U_\alpha}{\alpha\in J} be a family of sets in \(\mathcal{O}_\text{standard}\). Let \(p \in \bigcup_{\alpha\in J}U_\alpha\), that is, there exists \(\beta \in J\) such that \(p \in U_\beta\). Since \(U_\beta \in \mathcal{O}_\text{standard}\), there exists \(r_\beta > 0\) such that \(B_n(r, p) \subset U_\beta \subset \bigcup_{\alpha \in J} U_\alpha\).
\end{proof}

\begin{proposition}{Subspace topology is a topology}{subspace_topology}
    Given a topological space \topology{M} and a subset \(S\) of \(M\), we define the \emph{subspace topology} \restrict{\mathcal{O}_M}{S} as
    \begin{equation*}
        \restrict{\mathcal{O}_M}{S} = \set{U \cap S : U \in \mathcal{O}_M}.
    \end{equation*}
    Then \((S, \restrict{\mathcal{O}_M}{S})\) is a topological space.
\end{proposition}
\begin{proof}
    We must show the conditions (a), (b), and (c) of \cref{def:topology} are satisfied.
    \begin{enumerate}[label=(\alph*)]
        \item Since \(S = M \cap S\) and \(\emptyset = \emptyset \cap S\), we have \(S \in \restrict{\mathcal{O}_M}{S}\) and \(\emptyset \in \restrict{\mathcal{O}_M}{S}\).
        \item Let \(U, V \in \restrict{\mathcal{O}_M}{S}\). Then, there exists \(\tilde{U}, \tilde{V} \in \mathcal{O}_M\) such that \(U = \tilde{U} \cap S\) and \(V = \tilde{V} \cap S\).Then, \(U \cap V = (\tilde{U}\cap S) \cap (\tilde{V} \cap S) = (\tilde{U}\cap\tilde{V})\cap S\). Since \(\tilde{U} \cap \tilde{V} \in \mathcal{O}_M\), we have \(U \cap V \in \restrict{\mathcal{O}_M}{S}\).
        \item Let \family{U_\alpha}{\alpha \in J} be a family of open sets in \(\restrict{\mathcal{O}_M}{S}\). For each \(\alpha \in J\), there exists a \(\tilde{U}_\alpha\in\mathcal{O}_M\) such that \(U_\alpha = \tilde{U}_\alpha \cap S\). Then
            \begin{align*}
                \bigcup_{\alpha \in J} U_\alpha &= \bigcup_{\alpha \in J} \tilde{U}_\alpha \cap S\\
                                                &= \set{m \in S : \exists \alpha \in J \text{ such that } m \in \tilde{U}_\alpha}\\
                                                &= \set{m \in M : \exists \alpha \in J \text{ such that } m \in \tilde{U}_\alpha} \cap S\\
                                                &= S\cap\bigcup_{\alpha\in J}\tilde{U}_\alpha.
            \end{align*}
        Since arbitrary unions of open sets is an open set, it follows that \(\bigcup_{\alpha\in J}U_\alpha \in \restrict{\mathcal{O}_M}{S}\).
    \end{enumerate}
\end{proof}

\begin{proposition}{Product topology}{product_topology}
    Let \topology{M} and \topology{N} be topological spaces. Define the \emph{product topology} \(\mathcal{O}_{M\times N}\) as the collection of subsets \(U \subset M \times N\) such that for all \((m,n) \in U\), there exists neighborhoods \(S \subset M\) and \(T \subset N\) of \(m \in M\) and \(n\in N\) such that \(S \times T \subset U\). Then \topology{M\times N} is a topological space.
\end{proposition}
\begin{proof}
    Clearly, \(M\times N\) and \(\emptyset\) are open sets in the product topology.

    Next, we consider open sets \(U, V \in \mathcal{O}_{M\times N}\) and an element \(p \in U \cap V\). Let \(p = (m, n) \in M \times N\), then there exists neighborhoods \(S_U, S_V\subset M\) of \(m\) and \(T_U, T_V \subset N\) of \(n\) such that \(S_U \times T_U \subset U\) and \(S_V \times T_V \subset V\). Let \(S = S_U \cap S_V\) and \(T = T_U \cap T_V\), then \(S \in \mathcal{O}_M\) and \(T \in \mathcal{O}_N\) are neighborhoods of \(m\) and \(n\), respectively. Moreover, \(S \times T \subset U \cap V\) is a neighborhood of \(p\), from which follows \(U \cap V \in \mathcal{O}_{M\times N}\).

    Let \family{U_\alpha}{\alpha\in J} be a family of open sets in the product topology. Let \(p\in \bigcup_{\alpha\in J}U_\alpha\), then there exists \(\beta \in J\) such that \(p \in U_{\beta}\). By definition, there exists open sets \(S \in \mathcal{O}_M\) and \(T \in \mathcal{O}_N\) such that \(S \times T \subset U_\beta \subset \bigcup_{\alpha\in J} U_\alpha\). Therefore, \(\bigcup_{\alpha\in J}U_\alpha\) is an open set.
\end{proof}

Along with the axioms of topological spaces described in \cref{def:topology} one might add further restrictions to specify the space considered. Some common restrictions are called the \emph{separation axioms}. Among these, we will make use of the T2 axiom, namely the Hausdorff property. Historically, Felix Hausdorff used this axiom in his original definition of a topological space, although the formulation of his other axioms was not exactly as those of \cref{def:topology}, but an equivalent one.
\begin{definition}{Hausdorff space}{hausdorff}
    A topological space \topology{M} is called a \emph{Hausdorff space} if for any \(p,q\in M\) with \(p\neq q\), there exists a neighborhood \(U\) of \(p\), i.e. \(p \in U \in \mathcal{O}_M\), and a neighborhood \(V\) of \(q\) such that \(U \cap V = \emptyset\).
\end{definition}

\section{Homeomorphisms}

With the notion of topological spaces, we may ask ourselves whether certain maps between topological spaces can preserve the topology. That is, a map that takes open sets in the domain topology into open sets in the target topology. To define such a map we define \emph{continuity}.

\begin{definition}{Continuous map}{continuity}
    Let \topology{M} and \topology{N} be topological spaces. Then a map \(f : M \to N\) is \emph{continuous} (with respect to \(\mathcal{O}_M\) and \(\mathcal{O}_N\)) if, for all \(V \in \mathcal{O}_N\), the preimage \(f^{-1}(V)\) is an open set in \(\mathcal{O}_M\).
\end{definition}

In short, a map is continuous if and only the preimages of (all) open sets are open sets. Now a map that preserves the topology is called a \emph{homeomorphism}, which is defined as a continuous bijection with continuous inverse. We now prove such a map satisfies the condition required.

\begin{proposition}{Homeomorphism maps open sets to open sets}{homeomorphism}
    Let \topology{M} and \topology{N} be topological spaces. Suppose a map \(f : M \to N\) is a homeomorphism, then \(f\) maps open sets in \(\mathcal{O}_M\) into open sets in \(\mathcal{O}_N\).
\end{proposition}
\begin{proof}
    Given a subset \(U \in \mathcal{O}_M\), we must show the image \(V = f(U)\) is open in \topology{N}. Taking our attention to the inverse map \(g = f^{-1} : N \to M\), we see the preimage \(g^{-1}(U) = V\) must be open in \topology{N}, due to continuity.
\end{proof}

If there exists a homeomorphism between two topological spaces, they are said to be homeomorphic to each other. This begs the question: if \topology{M} is homeomorphic to \topology{N} and \topology{N} is homeomorphic to \topology{P}, are \topology{M} and \topology{P} homeomorphic? To answer this we must show whether the composition of continuous maps is itself continuous.

\begin{theorem}{Composition of continuous maps}{continuous_composition}
    Let \topology{M}, \topology{N}, and \topology{P} be topological spaces. If the maps \(f: M \to N\) and \(g : N \to P\) are continuous (with respect to the appropriate topologies), then the map \(g \circ f : M \to P\) is continuous with respect to \(\mathcal{O_M}\) and \(\mathcal{O_P}\).
\end{theorem}
\begin{proof}
    Let \(V\) be an open set of \topology{P}. We must show the preimage \((g \circ f)^{-1}(V)\) is an open set of \topology{M}. We have
    \begin{align*}
        (g\circ f)^{-1}(V) &= \set{m \in M : g\circ f(m) \in V}\\
                           &= \set{m \in M : f(m) \in g^{-1}(V)}\\
                           &= f^{-1}\left(g^{-1}(V)\right).
    \end{align*}
    Since the map \(g\) is continuous and \(V\) is an open set in \topology{P}, it follows that \(g^{-1}(V)\) is open in \topology{N}. By the same argument, \(f^{-1}\left(g^{-1}(V)\right)\) is an open set in \topology{M}.
\end{proof}

\begin{corollary}
    If \topology{M} is homeomorphic to \topology{N} and \topology{N} is homeomorphic to \topology{P}, then \topology{M} is homeomorphic to \topology{P}.
\end{corollary}
\begin{proof}
    Let \(f : M \to N\) and \(g : N \to P\) be homeomorphisms from \topology{M} to \topology{N} and \topology{N} to \topology{P}, respectively. Consider the composition \(g\circ f : M \to P\).
    \[
    \begin{tikzcd}
        M \arrow{r}{f} \arrow[swap]{dr}{g\circ f} & N \arrow{d}{g} \\
                                            & P
    \end{tikzcd}
    \]
    By \cref{thm:continuous_composition}, the map \(g\circ f\) is a homeomorphism from \topology{M} to \topology{P}.
\end{proof}

As was done for the subspace topology, we prove a similar result for continuous maps.

\begin{proposition}{Restriction of a continuous map}{restriction_map}
    Let \topology{M} and \topology{N} be topological spaces and let \(f : M \to N\) be a continuous map. Let \(S\) be a subset of \(M\) and let \topology{S} be the subspace topology, then \(\restrict{f}{S} : S \to N\) is a continuous map with respect to \(\mathcal{O}_S\) and \(\mathcal{O}_N\).
\end{proposition}
\begin{proof}
    Let \(V \in \mathcal{O}_N\). Then, by the definition of preimage, we have
    \begin{align*}
        \restrict{f}{S}^{-1}(V) &= \set{s \in S : \restrict{f}{S}(s) \in V}\\
                                &= \set{s \in S : f(s) \in V}\\
                                &= f^{-1}(V) \cap S.
    \end{align*}
    By hypothesis, the preimage \(f^{-1}(V)\) is an open set in \topology{M}, so \(\restrict{f}{S}^{-1}(V)\) is an open set in the subspace topology.
\end{proof}

We can now define the notion of a topological space locally resembling Euclidean space.
\begin{definition}{Locally Euclidean topological space}{locally_euclidean}
    A topological space \topology{M} is \emph{locally Euclidean} of dimension \(n\) if for all \(m \in M\) there exists an open subset \(U \in \mathcal{O}_M\) about \(m\) that is homeomorphic to \(\mathbb{R}^n\) with respect to the subspace topology and the standard topology of \(\mathbb{R}^n\).
\end{definition}
It is sufficient to show the subspace topology \(\topology{U}\) is homeomorphic to an open ball in \(\mathbb{R}^n\), due to \cref{prop:ball_homeomorphic_euclidean}.
\begin{proposition}{Open ball is homeomorphic to the Euclidean space}{ball_homeomorphic_euclidean}
    Let \(r > 0\), then the map \(f : B_n(r, 0)\subset\mathbb{R}^n\to\mathbb{R}^n\) given by
    \[f(x) = \frac{x}{r - \norm{x}}\]
    is a homeomorphism with respect to the standard topology.
\end{proposition}
\begin{proof}
    We begin by checking \(f\) is one-to-one and onto.

    Suppose there exists \(x_1, x_2 \in B_n(r, 0)\) such that \(f(x_1) = f(x_2)\). It follows from
    \begin{align*}
        f(x_2) - f(x_1) &= \frac{x_2}{r - \norm{x_2}} - \frac{x_1}{r - \norm{x_1}}\\
                        &= \frac{\left(r - \norm{x_1}\right)x_2 - \left(r - \norm{x_2}\right)x_1}{\left(r - \norm{x_2}\right)\left(r - \norm{x_1}\right)}
    \end{align*}
    that \(\left(r - \norm{x_1}\right)x_2 = \left(r - \norm{x_2}\right)x_1\). Applying the norm to both sides, we have \(\norm{x_1} = \norm{x_2}\). Substituting back, we have \(x_1 = x_2\), proving \(f\) is injective.

    Suppose \(y \in \mathbb{R}^n\) and consider \(\xi = \frac{ry}{1 + \norm{y}}\). Clearly, \(\xi \in B_n(r,0)\). We have
    \begin{align*}
        f(\xi) &= f\left(\frac{ry}{1 + \norm{y}}\right)\\
               &= \frac{ry}{1 + \norm{y}} \frac{1}{r - \norm*{\frac{ry}{1 + \norm{y}}}}\\
               &= \frac{1}{\left(1 + \norm{y}\right)\left(1 - \frac{\norm{y}}{1 + \norm{y}}\right)} y\\
               &= y,
    \end{align*}
    so \(f\) is onto.

    We have shown \(f\) is a bijection with inverse \(f^{-1} : \mathbb{R}^n \to B_n(r, 0)\) defined by
    \begin{equation}
        f^{-1}(x) = \frac{rx}{1 + \norm{x}}.
    \end{equation}
    With the standard topology, continuity of \(f\) and \(f^{-1}\) follows from techniques of elementary calculus, and we conclude \(f\) is a homeomorphism.
\end{proof}

\section{Compactness and paracompactness}

\begin{definition}{Hausdorff space}{hausdorff}
    A topological space \topology{M} is called a \emph{Hausdorff space} if for any \(p,q\in M\) with \(p\neq q\), there exists a neighborhood \(U\) of \(p\), i.e. \(p \in U \in \mathcal{O}_M\), and a neighborhood \(V\) of \(q\) such that \(U \cap V = \emptyset\).
\end{definition}
\begin{remark}
    The Hausdorff property is one of the \emph{separation axioms} of topological spaces. Namely, a Hausdorff space is also called a \emph{T2 space}.
\end{remark}

\begin{definition}{Compactness}{compact}
    A topological space \topology{M} is \emph{compact} if every \emph{open cover} of \(M\) has a finite subcover. That is, the topological space is compact if for every family of open sets \(C\) that covers \(M\), i.e. \(\bigcup_{U \in C}U = M\) with \(U \in \mathcal{O}_M\), there exists a finite family of open sets \(F \subset C\) such that \(\bigcup_{U\in F} U = M\).

    Additionally, in a topological space \topology{N}, a subset \(S\subset N\) is called compact if the subspace topology is compact.
\end{definition}

\begin{theorem}{Heine-Borel theorem}{heine_borel}
    A subset \(S\subset\mathbb{R}^n\) with the standard topology is compact if it is closed and bounded.
\end{theorem}
\begin{proof}
    Refer to \cite{babyrudin}.
\end{proof}

\begin{definition}{Locally finite collection}{locally_finite}
    A collection of subsets \(C\) of a topological space \topology{M} is called \emph{locally finite} if each point in the space has a neighborhood that intersects only finitely many sets in \(C\). More precisely, for all \(p \in M\) there exists a neighborhood \(U \in \mathcal{O}_M\) about \(p\) such that \(U \cap V \neq \emptyset\) only for finitely many \(V \in C\).
\end{definition}

\begin{definition}{Refinement}{refinement}
    A \emph{refinement} of a cover \(C\) of a topological space \topology{M} is a cover \(D\) such that every set in \(D\) is contained in some set in \(C\). Precisely, let \(C = \family{U_\alpha}{\alpha \in A}\) and \(D = \family{V_\beta}{\beta \in B}\) such that \(\bigcup_{\alpha \in A} U_\alpha = M\) and \(\bigcup_{\beta \in B} V_\beta = M\), then \(D\) is a refinement of \(C\) if for all \(\beta \in B\) there exists \(\alpha \in A\) such that \(V_\beta \subset U_\alpha\).
\end{definition}

\begin{definition}{Paracompactness}{paracompact}
    A topological space \topology{M} is called \emph{paracompact} if every open cover \(C\) has an \emph{open refinement} \(\tilde{C}\) that is \emph{locally finite}.
\end{definition}

\begin{definition}{Partition of unity}{partition_of_unity}
    A \emph{partition of unity} of a topological space \topology{M} is a set \(\mathcal{F}\) of continuous functions from \(M\) to \([0,1]\subset\mathbb{R}\) such that for every point \(p \in M\)
    \begin{enumerate}[label=(\alph*)]
        \item there exists a neighborhood \(U \in \mathcal{O}_M\) about \(p\) where all but finitely many functions of \(\mathcal{F}\) vanish on \(U\);
        \item the sum of all function values at \(p\) is 1, that is, \(\sum_{f \in \mathcal{F}} f(p) = 1\).
    \end{enumerate}

    Moreover, let \(C = \family{U_\alpha}{\alpha \in J}\) be an open cover of \(M\). A \emph{partition of unity subordinate to the open cover \(C\)} is a family \(\mathcal{F}_C\) of continuous maps \(f_\alpha : p \to [0,1] \subset\mathbb{R}\) indexed over the same set \(J\) such that the support of \(f_\alpha\) is contained in \(U_\alpha\), for all \(\alpha \in J\). That is, for every \(f \in \mathcal{F}_C\) there exists an open set \(U \in C\) such that \(f(p) \neq 0 \implies p \in U\).
\end{definition}

\begin{theorem}{Paracompactness and partitions of unity}{hausdorff_paracompact}
    Let \topology{M} be a Hausdorff space. Then it is paracompact if and only if every open cover \(C\) admits a partition of unity subordinate to that cover.
\end{theorem}
\begin{proof}
    Refer to \cite{munkres_topology}.
\end{proof}

\input{chapter01/s4-connectedness}



\printbibliography
\end{document}
